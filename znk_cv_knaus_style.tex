\documentclass{article}
\pagestyle{plain}
\usepackage{natbib}
\usepackage{bibentry}
\usepackage{hyperref}
\usepackage{graphicx}
\usepackage[svgnames]{xcolor} 
\usepackage{fullpage}

\hypersetup{
    colorlinks=true,       % false: boxed links; true: colored links
    urlcolor=blue           % color of external links
}





\newcommand{\bibverse}[1]{\begin{verse} \bibentry{#1}. \end{verse} }


\newlength{\spacebox} 
\settowidth{\spacebox}{8888888888}


\usepackage{sectsty} 
    \sectionfont{% 
        \usefont{OT1}{phv}{medium weight}{n}% 
        \sectionrule{0pt}{0pt}{-5pt}{1pt} }

\newcommand{\MyName}[1]{
		\begin{center}
		\Huge \usefont{OT1}{phv}{b}{n} #1 % Name
		\end{center}
		\par \normalsize \normalfont
}



\newcommand{\MySlogan}[1]{
		\large \usefont{OT1}{phv}{m}{n}
		\textit{#1} % Slogan (optional)
		\par \normalsize \normalfont}

\newcommand{\NewPart}[1]{\section*{\uppercase{#1}}}

\newcommand{\PersonalEntry}[2]{
		\noindent\hangindent=2em\hangafter=0 		% Indentation
		\parbox{\spacebox}{						% Box to align text
		\textit{#1}}								% Entry name (birth, address, etc.)
		\hspace{1.5em} #2 \par}					% Entry value

\newcommand{\EducationEntry}[4]{
		\noindent \textbf{#1} % Study
		\hfill
		\colorbox{White}{\parbox{6em}{\hfill\color{Black}#2}}\par % Duration
		\noindent \textit{#3} \par % School
		\noindent\ \small #4 	% Description
		\normalsize
		\par
}

\newcommand{\EducationEntryB}[4]{
		\noindent \textbf{#1} % Study
		\hfill
		\colorbox{White}{\parbox{6em}{\hfill\color{Black}#2}}\par % Duration
		\noindent \textit{#3} % School
		\hfill
		\colorbox{White}{\parbox{34em}{\hfill\color{Black}#4}}
		\normalsize
		\par
}



\newcommand{\WorkEntry}[4]{
		\noindent \textbf{#1} \hfill % Study
		\colorbox{White}{%
			\parbox{24em}{%
			\hfill\color{Black}#2}} \par	% Duration
		\noindent \textit{#3}\par % School
		\noindent \small #4 \par % Description.
		\normalsize \par
		\vspace{6 pt}}

\newcommand{\ResearchInterest}[3]{
		\noindent\hangindent=0.3in
		\textbf{#1:} \emph{#2}. #3
		\vspace{12 pt}
}


\newcommand{\AwardEntry}[3]{
		\noindent\hangindent=0.3in
		\textbf{#1}. #2. #3.
}

\newcommand{\TwoEntry}[2]{
		\noindent
		\hangindent=0.5in
		\textbf{#1} #2.
}

\newcommand{\AddReferences}{
  \NewPart{References}{}
  (Please contact me to arrange letters.)
  %\vspace{6 pt}
  %\textbf{Nik Gr\"{u}nwald, Ph.D.}
  %\PersonalEntry{Title}{Research Plant Pathologist}
  %\PersonalEntry{Organization}{Horticultural Crops Research Lab, USDA, Agricultural Research Service}
  %\PersonalEntry{Address}{3420 NW Orchard Avenue, Corvallis, OR, 97331}

  \vspace{6 pt}
  \raggedright
  \noindent
  \textbf{Richard C. Cronn, Ph.D.}\\
  \PersonalEntry{Title}{Research Geneticist}
  \PersonalEntry{Organization}{Pacific Northwest Research Station, USDA Forest Service}
  \PersonalEntry{Address}{3200 SW Jefferson Way, Corvallis, OR, 97331}
  %\PersonalEntry{E-mail}{rcronn at fs dot fed dot us}
  \PersonalEntry{E-mail}{rcronn@fs.fed.us}
  \PersonalEntry{Phone}{541-750-7291}

  \vspace{6 pt}
  \textbf{Aaron Liston, Ph.D.}\\
  \PersonalEntry{Title}{Professor}
  \PersonalEntry{Organization}{Oregon State University}
  \PersonalEntry{Department}{Botany and Plant Pathology}
  \PersonalEntry{Address}{2082 Cordley Hall, Corvallis, OR, 97331}
  %\PersonalEntry{E-mail}{listona at science dot oregonstate dot edu}
  \PersonalEntry{E-mail}{listona@science.oregonstate.edu}
  \PersonalEntry{Phone}{541-737-5301}


  \vspace{6 pt}
  \textbf{Peter C. Dolan, Ph.D.}\\
  \PersonalEntry{Title}{Assistant Professor}
  \PersonalEntry{Organization}{University of Minnesota, Morris}
  \PersonalEntry{Department}{Mathematics and Computer Science}
  \PersonalEntry{Address}{600 East 4th Street, Morris, MN, 56267}
  \PersonalEntry{E-mail}{dolan118@morris.umn.edu}
  \PersonalEntry{Phone}{320-589-6307}
}


\begin{document}

\nobibliography*

\MyName{Brian J. Knaus}
\begin{center}
\Large
\emph{Curriculum vitae}
\normalsize
\end{center}


\pagestyle{empty}

\NewPart{Personal details}{}

\PersonalEntry{Title}{Research Plant Pathologist}
\PersonalEntry{Organization}{USDA Agricultural Research Service}
\PersonalEntry{Department}{Horticultural Crops Research Unit}
\PersonalEntry{Address}{3420 NW Orchard Avenue Corvallis, OR 97330}
\PersonalEntry{E-mail}{brian.knaus@gmail.com}
\PersonalEntry{Website}{\url{http://brianknaus.com}}


\NewPart{Education}{} 

\EducationEntryB{Ph.D. Botany and Plant Pathology}{2008}{Oregon State University}{}
\bibverse{knaus_phd}
\small
\begin{center}
\emph{
Co-advised by Drs. Rich Cronn (USDA PNW) and Aaron Liston (OSU).\\
}
\end{center}
\par
\normalsize
\EducationEntryB{B.S. Ecology and Evolutionary Biology}{1997}{The University of Arizona}
{With an incorporated minor in chemistry, math and physics}


\NewPart{Research Interests}{}

\ResearchInterest{Population genetics}{The study of the structure and demographics of populations}{Inference of subdivision, effective size and migration are central topics to understanding the dynamics of populations.  My research has sought to infer the structure and demographic processes which may have led to the structure we observe in populations.}


\ResearchInterest{Evolutionary biology}{The synthesis of biological pattern and process}{Evolution provides the theoretical background to describe the patterns of diversity we observe.  My research has employed evolutionary models to identify processes which may have led to the patterns we observe in nature.}


\ResearchInterest{Bioinformatics}{The union of biology, computer science and statistics}{
As the cost of genome scale sequencing has plummeted the practicality of assaying these genomes has become increasingly feasible.  We are currently standing on the threshold of connecting the phenotypes we observe with their genetic basis.  In biology, much of our work has recently become largely computational.  My work ranges from the quality control of large datasets to the visualization of these data and their analysis.}




\NewPart{Work experience}{}

\WorkEntry{Research Plant Pathologist}{2013-present}{Horticultural Crops Research Unit, USDA Agricultural Research Service}{Genomic architecture of fungicide resistance and mating type as well as the community ecology of plant pathogens in the genus \emph{Phytophthora}.}

\WorkEntry{Post-doctoral scholar}{2012-2013}{Department of Botany and Plant Pathology, Oregon State University}{Modeling dispersal of the causative agent of sudden oak death (\emph{Phytophthora ramorum}).}

\WorkEntry{Research geneticist}{2008-2012}{Pacific Northwest Research Station, USDA Forest Service}{Transcriptome \emph{de novo} assembly and differential expression analysis (RNA-Seq) in Douglas-fir (\emph{Pseudotsuga menzeisii}) to identify transcripts associated with spring bud-burst and the transition from dormancy to physiological growth.}

\WorkEntry{Botanist}{2006, 2007, 2008}{Pacific Northwest Research Station, USDA Forest Service}{Developed an R package for microsatellite analyses. Analysed population genetic diversity for antelope bitterbrush (\emph{Purshia tridentata}).}

\WorkEntry{Graduate teaching assistant}{2002-2003, 2006-2008}
{Oregon State University}{See teaching experience.}



\WorkEntry{Graduate research assistant}{2003-2006}
{Oregon State University}{Generated and analyzed molecular genetic datasets (AFLP, CpSSR) for several intermountain plant taxa.}

\WorkEntry{Ecologist}{2003}
{USDA Forest Service, Sierra Nevada Forest Plan Ammendment}{Led a team of field plant community data collectors in meadows of the Sierra Nevada Mountains of California, USA.}


\WorkEntry{Biological science technician}{1998, 1999, 2000, 2000, 2000-2001}{US Geological Survey, Sequoia and Kings Canyon Field Station; Sequoia and Kings Canyon National Parks; Pacific Southwest Research Station, Riverside Fire Lab, USDA Forest Service; Dorena Tree Improvement Center, USDA Forest Service; Death Valley National Park}{Participated in field collection teams for demographic monitoring, fire history reconstruction, relocation of historic study sites, breeding of disease resistant trees and surveys for rare plants.}








\WorkEntry{Student research assistant}{1996-1998}{Laboratory of Tree-Ring Research, The University of Arizona}{Field collection and dendrochronological data processing to infer historic tree-lines in the Sierra Nevada mountains of California, USA.}




\NewPart{Publications}{}

Google scholar site: \url{http://scholar.google.com/citations?user=19wcuSwAAAAJ&hl=en}

\bibverse{parke2014}
\bibverse{gilmore_etal_2014}
\bibverse{hayden_etal_2014}
\bibverse{wilson_etal_2014}
\bibverse{eckert_etal_2013}
\bibverse{wall_etal_2013}
\bibverse{miller_etal_2013}
\bibverse{jennings_etal_2013}
\bibverse{bushley_etal_2013}
\bibverse{ross-davis_etal_2013}
\bibverse{rai_etal_2013}
\bibverse{howe_etal_2013}
\bibverse{gilmore_etal_2012}
\bibverse{cronn_etal_2012}
\bibverse{jennings_etal_2011a}
\bibverse{jennings_etal_2011b}
\bibverse{knaus_etal_2011}
\bibverse{haig_etal_2011}
\bibverse{knaus_2010}
\bibverse{haig_etal_2006}
\bibverse{knaus_2005}
\bibverse{knaus_2005b}



\NewPart{Presentations}{}

\vspace{6 pt}

\textbf{Invited presentations:}

\bibverse{knaus_etal_tws2012}
\bibverse{knaus_lc2010}
\bibverse{knaus_ps2009}

\vspace{6 pt}

\noindent
\textbf{Contributed presentations:}

\bibverse{knaus_etal_bot2012}
\bibverse{knaus_etal_pag2012}
\bibverse{knaus_etal_wfga2011}
\bibverse{yu_etal_pag2011}
\bibverse{knaus_etal_evol2010}
\bibverse{knaus_cnps2009}
\bibverse{knaus_def2008}
\bibverse{knaus_etal_bot2008}
\bibverse{knaus_etal_bot2007}
\bibverse{knaus_etal_evol2006}
\bibverse{knaus_etal_wibo2006}
\bibverse{knaus_etal_bot2006}


\NewPart{Awards and funding}{}

\AwardEntry{George R. Cooley Award}{2008}{Award for best contributed presentation by a researcher in the early part of their career (within five years of defense).  Awarded by the American Society of Plant Taxonomists at the internationally attended Botany meetings in Vancouver Canada. \url{http://www.inhs.uiuc.edu/~kenr/ASPT/current.html}}

\AwardEntry{Bonnie C. Templeton Award for Plant Systematics}{2007}{An endowment to support research in plant systematics. \$1,500}

\AwardEntry{Anita Summers Travel Grant}{2006}{An award to help facilitate travel of graduate students to academic meetings.  \$300}

\AwardEntry{The Hardman Foundation, Inc}{2004}{Support for graduate student research concerning the native plants of Oregon. \$1,500}

\AwardEntry{Native Plant Society of Oregon Field Research Grant}{2004}{Award to finance the field study of varieties of \emph{Astragalus lentiginosus} Douglas ex Hooker native to Oregon. \$750}

\AwardEntry{Nevada Native Plant Society}{2004}{Grant for the field study of varieties of \emph{Astragalus lentiginosus} Douglas ex Hooker native to Nevada. \$500}


\NewPart{Teaching Experience}{}

\AwardEntry{Introductory Biology for Majors}{(OSU BI21X)}{Lead laboratories of $\sim$48 students in organismal diversity, organ systems, plant and animal physiology, genetics, evolution and ecology.  Terms taught: Fall 2002, Winter 2003, Fall 2006, Winter 2007, Spring 2007, Fall 2007, Winter 2008}

\AwardEntry{Plant Systematics}{(OSU BOT321)}{Lead laboratories in vascular plant identification, diversity, and evolutionary relationships. Terms taught: Spring 2003}

\AwardEntry{Botany for Non-Majors}{(OSU BOT101)}{Lead laboratories of $\sim$24 students in the relevance of botany to everyday life. Terms taught: Spring 2008}


\NewPart{Bioinformatics}{}

Proficient in \textbf{R} and \textbf{Perl} in the Windows and \textbf{Linux} operating systems.  Experienced in the use of queueing systems such as the Sun Grid Engine (SGE) as well as the Portable Batch System (PBS) as implemented on the XSEDE national cyberinfrastructure system (i.e., the Pittsburgh Super Computing Center).  Active projects maintained at GitHub: \url{https://github.com/knausb}.

\vspace{6 pt}

\noindent
\textbf{Authored software:}
\begin{description} \itemsep0pt \parskip0pt
  \item[vcfR] An R package to view and analyze chromosomal variant data.
  \item[Short Read Toolbox] A set of perl scripts (including R and shell scripts) for quality control and barcode sorting of Illumina short-read sequences.
  \item[Genomatic] An R package which automates fragment analysis projects.
\end{description}




\NewPart{Service}{}

\TwoEntry{Peer reviewer of research journals:}{
American Journal of Botany,
Botanical Journal of the Linnaean Society,
Journal of the American Society for Horticultural Science,
Madro{\~n}o,
Molecular Ecology
and
Molecular Phylogenetics and Evolution}

\TwoEntry{Organizer of the RNA-Seq workgroup.}{A group exploring the methods of differential expression using RNA-Seq. \url{http://people.oregonstate.edu/~knausb/rna_seq/}}

\TwoEntry{Organizer of the R Statistical Programming Group.}{A group of Oregon State University Biologists interested in use of the R programing environment. \url{http://oregonstate.edu/~knausb/R_group/R_User.html}}

\vspace{6 pt}
\noindent
\textbf{Presentations:}

\bibverse{knaus_etal_ssfwg2011}
\bibverse{knaus_npso2008}
\bibverse{knaus_lug2007}
\bibverse{knaus_npso2004}







\NewPart{Other Affiliations}{}



\textbf{Professional/Societal membership:}\\
American Association for the Advancement of Science\\
American Phytopathological Society\\
American Society of Plant Taxonomists\\
Botanical Society of America\\
Society for the Study of Evolution\\




\global\let\includereferences\relax

\ifx\includereferences\undefined

\AddReferences

\fi



\bibliographystyle{plain}
\nobibliography{knaus_cv}


\end{document}
