%%%%%%%%%%%%%%%%%%%%%%%%%%%%%%%%%%%%%%%%%
% "ModernCV" CV and Cover Letter
% LaTeX Template
% Version 1.1 (9/12/12)
%
% This template has been downloaded from:
% http://www.LaTeXTemplates.com
%
% Original author:
% Xavier Danaux (xdanaux@gmail.com)
%
% License:
% CC BY-NC-SA 3.0 (http://creativecommons.org/licenses/by-nc-sa/3.0/)
%
% Important note:
% This template requires the moderncv.cls and .sty files to be in the same
% directory as this .tex file. These files provide the resume style and themes
% used for structuring the document.
%
%%%%%%%%%%%%%%%%%%%%%%%%%%%%%%%%%%%%%%%%%

%----------------------------------------------------------------------------------------
%	PACKAGES AND OTHER DOCUMENT CONFIGURATIONS
%----------------------------------------------------------------------------------------

\documentclass[$fontsize$,letterpaper,sans]{moderncv} % Font sizes: 10, 11, or 12; paper sizes: a4paper, letterpaper, a5paper, legalpaper, executivepaper or landscape; font families: sans or roman

\moderncvstyle{classic} % CV theme - options include: 'casual' (default), 'classic', 'oldstyle' and 'banking'
\moderncvcolor{blue} % CV color - options include: 'blue' (default), 'orange', 'green', 'red', 'purple', 'grey' and 'black'
%
% % LAYOUT
% %--------------------------------
% Margins
\usepackage{geometry}
\geometry{$geometry$}
%
% Do not indent paragraphs
\setlength\parindent{0in}
%
% % Enable multicolumns
% \usepackage{multicol}
% \setlength{\columnsep}{-3.5cm}

% Uncomment to suppress page numbers
% \pagenumbering{gobble}
%
% LANGUAGE
%--------------------------------
% Set the main language
$if(lang)$
\usepackage{polyglossia}
\setmainlanguage{$lang$}
$endif$

% TYPOGRAPHY
%--------------------------------
\usepackage{fontspec}
\usepackage{xunicode}
\usepackage{xltxtra}
% converts LaTeX specials (quotes, dashes etc.) to Unicode
\defaultfontfeatures{Mapping=tex-text}
\setromanfont [Ligatures={Common}, Numbers={OldStyle}]{$mainfont$}
% Cool ampersand
\newcommand{\amper}{{\fontspec[Scale=.95]{$mainfont$}\selectfont\itshape\&}}

\usepackage{enumitem}
\usepackage{lipsum} % Used for inserting dummy 'Lorem ipsum' text into the template
\usepackage{natbib}
\usepackage{bibentry}
% \usepackage[scale=0.75]{geometry} % Reduce document margins
%\setlength{\hintscolumnwidth}{3cm} % Uncomment to change the width of the dates column
%\setlength{\makecvtitlenamewidth}{10cm} % For the 'classic' style, uncomment to adjust the width of the space allocated to your name

\newcommand{\bibverse}[1]{\begin{verse} \bibentry{#1}. \end{verse} }
%----------------------------------------------------------------------------------------
%	NAME AND CONTACT INFORMATION SECTION
%----------------------------------------------------------------------------------------

\firstname{$name.first$} % Your first name
\middlename{$name.middle$}
\familyname{$name.last$} % Your last name
\email{$name.email$}
% All information in this block is optional, comment out any lines you don't need
\title{Curriculum Vitae}
\address$for(address)${$address$}$endfor$
\mobile{$phone$}



% Uncomment for black and white links
%\AfterPreamble{\hypersetup{colorlinks,urlcolor=black,linkcolor=black}}
%----------------------------------------------------------------------------------------
\begin{document}

% If you want a cover letter to go with it. Note that this cover letter will be included
% in the pagination.
% 
%----------------------------------------------------------------------------------------
%	COVER LETTER
%----------------------------------------------------------------------------------------

% To remove the cover letter, comment out this entire block

\clearpage
\begin{flushright}
  Dr. Zhian N. Kamvar\\
  Department of Infectious Disease Epidemiology\\
  Imperial College London, W2 1NY, UK\\
  \today
\end{flushright}
% \recipient{HR Departmnet}{EcoHealth Alliance\\460 West 34th Street - 17th Floor\\New York, NY 10001-2320} % Letter recipient
% \date{\today} % Letter date
% \opening{Dear Dr. Ross, members of the search committee} % Opening greeting
\closing{Sincerely,} % Closing phrase
\enclosure[Attached]{curriculum vit\ae{}, references} % List of enclosed documents

% \makelettertitle % Print letter title
Dear Dr. Ross, members of the search committee,

\vspace{1ex}

I am applying for the position of Software Research Scientist. My
training as a computational biologist, passion for robust test-driven software
developement, and practical experience in cross-discipline collaboration and
communication makes me the ideal candidate for this position. As a developer
of software for analysis of both population genetic and epidemiological data, I
am keenly aware of how the lack of standards impacts scientists\footnote{one
recent example:
\url{https://twitter.com/reneecatullo/status/1133900861397016578}}. I have both
the understanding to know how a well-tested package can still give incorrect
results and the technical expertise to collaborate with a team to deliver
robust and useable software that produces correct results. Having worked as
both a research scientist and research software engineer, I am excited for the
opportunity to collaborate with the rOpenSci team. 

\vspace{1ex}

I have all of the preferred skills and qualifications necessary for this
position including a degree in a quantitative field (population genetics),
expertise in open-source R package development\footnote{examples: population
genetics---\href{https://grunwaldlab.github.io/poppr}{poppr},
epidemiology---\href{https://www.repidemicsconsortium.org/aweek}{aweek}, and
\href{https://github.com/R4EPI/sitrep}{sitrep}}, and strong technical writing
skills\footnote{scientific: \href{https://peerj.com/articles/4152/}{Kamvar
\textit{et al.} 2017}, technical:
\href{https://grunwaldlab.github.io/poppr/reference/psex.html}{Documentation
for \texttt{poppr::psex()}}}.
As the co-creator of an
\href{https://blogs.oregonstate.edu/inspiration}{award-winning sci-comm radio
program}, I have unique experience in interdisciplinary communication.  
Moreover I have experience in leading collaborative projects\footnote{RECON + MSF:
\href{https://R4EPIs.netlify.com}{The R4EPIs project}} which employ respectful
peer review\footnote{\href{https://github.com/R4EPI/sitrep/pull/76}{Example
code review for the sitrep package}} with collaborators of varying expertise.
Most importantly, I have a passion for robustness in statistical
software\footnote{\href{https://zkamvar.netlify.com/post/2017-09-23-squish/squish}{Blog
post about correcting a statistical error in my own package}}.

\vspace{1ex}


To give a more concrete picutre of my skillset, I have been developing in R and
C since 2012 (version control since 2013, tests and continous integration since 
2014) and currently maintain five packages on CRAN. All of these packages are
tested under continuous integration and one
(\href{https://grunwaldlab.github.io/poppr}{\textit{poppr}}) is featured in
\textgreater500 scientific publications. All of my published papers have the
code and data needed to reproduce the results. My most recent first-author 
analytical work, \href{https://peerj.com/articles/4152/}{Kamvar \textit{et al.}
2017}, is fully reproducible in a Docker container under continuous integration. 

\vspace{1ex}

Collaboaration is an important part of my research in that I have never worked
on a single project without contributing to someone else's. In the case of
\textit{poppr}, I contributed heavily to the \textit{adegenet} package during
and after a 2015 NESCENT Population Genetics in R hackathon. In the case of
Kamvar \textit{et al.} 2017, I was able to track down and fix a
\href{https://github.com/slowkow/ggrepel/issues/72}{hidden bug} in the
\textit{ggrepel} package. Most recently, I am one of the technical leads on a
collaboration between the R Epidemics Consortium (RECON) and M\'{e}decins Sans
Fronti\`{e}res (MSF) called R4EPIs where we work on standardizing analyses and
training for field epidemiologists.  

\vspace{1ex}

As my carreer focus has shifted from investigative research to scientific
software engineering, I believe this project will be immensely beneficial to my
trajectory as a research software engineer. My background in quantitative
science, robust software development, collaborative capacity, and passion for
both equity and fairness in peer review makes me well-suited to join the team
as a software research scientist. Thank you for your time and consideration, I
look forward to hearing from you. 

\vspace{2ex}

Sincerely,

\vspace{5ex}

\textbf{Zhian N. Kamvar}\\
\textit{Attached: curriculum vit\ae{}, references}
\clearpage

%----------------------------------------------------------------------------------------


% EcoHealth Alliance seeks a creative, dedicated, and collaborative software
% research scientist to support a two-year project in launching a new software
% peer-review initiative. The software research scientist will work on the
% Sloan Foundation supported rOpenSci project, with rOpenSci staff and a
% statistical methods editorial board. They will research and develop standards
% and review guidelines for statistical software, publish findings, and develop
% R software to test packages against those standards. The software research
% scientist will work with staff and the board to collaborate broadly with the
% statistical and software communities to gather input, refine and promote the
% standards, and recruit editors and peer reviewers. The candidate must be
% self-motivated, proactive, collaborative and comfortable working openly and
% reproducibly with a broad online community.

% DESCRIPTION AND RESPONSIBILITIES

% - Research standards and protocols for evaluating statistical software
%     quality and correctness, and the extent of their adoption
% - Create new testing frameworks for R packages
% - Assist rOpenSci staff and project board members in drafting new
%     peer-review evaluation standards, guidelines, and documentation
% - Write technical and scientific papers, documentation, and blog posts
% - Assist in organizing peer-review system for scientific software and
%     managing the review board
% - Participate in and represent the rOpenSci project in person and via
%     on-line fora
% - Participate in other projects and tasks as required or assigned by
%     supervisor

% MINIMUM QUALIFICATIONS

% - A Master's degree in statistics, computer science, or a related field with
%     a focus on quantitative methodologies, or equivalent experience in
%     statistical methods evaluation and development
% - Expertise in open-source R package development, including collaborative
%     development using Git and GitHub, testing frameworks, and continuous
%     integration
% - Strong writing skills
% - Experience in collaborative team projects and consensus building
% - A passion for improving scientific reproducibility

% DESIRED QUALIFICATIONS

% - A PhD in statistics, computer science, or a related field with a focus on
%     quantitative methodologies, or equivalent experience in statistical
%     methods evaluation and development
% - Published scientific or technical articles or software documentation





% \makecvtitle % Print the CV title

$if(software)$
\section{Authored Software}

\subsection{R Packages}
$for(software.R)$
\cvitemwithcomment{$software.R.role$}{\href{$software.R.url$}{$software.R.name$}}{$software.R.description$}
$endfor$

\subsection{Genomics}
$for(software.Genomics)$
\cvitemwithcomment{$software.Genomics.role$}{\href{$software.Genomics.url$}{$software.Genomics.url$}}{$software.Genomics.description$}
$endfor$

\subsection{Websites}
$for(software.Websites)$
\cvitemwithcomment{$software.Websites.role$}{\href{$software.Websites.url$}{$software.Websites.url$}}{$software.Websites.description$}
$endfor$

$endif$

$if(awards)$
\section{Awards}
$for(awards)$
\cvitem{$awards.year$}{$awards.award$ -- $awards.amount$}
$endfor$
$endif$



% \section{Education}
% 
%----------------------------------------------------------------------------------------
%	EDUCATION SECTION
%----------------------------------------------------------------------------------------

\cventry{2017--Present}{Postdoctoral Scholar}{University of Nebraska-Lincoln (UNL)}{Lincoln, NE}{}{Department of Plant Pathology}
\cventry{2011--2016}{Ph.D. Plant Pathology}{Oregon State University (OSU)}{Corvallis, OR}{}{Defended: 6 December 2016}  % Arguments not required can be left empty
\cventry{2007}{B.S. Biology}{Truman State University (TSU)}{Kirksville, MO}{}{Minor: Chemistry}


%
% \section{Employment}
% 
%----------------------------------------------------------------------------------------
%	WORK EXPERIENCE SECTION
%----------------------------------------------------------------------------------------


\cventry{2012--Present}{Thesis Research}{Gr\"unwald Lab}{OSU, Corvallis}{OR}{
	My goal is to determine pattern and process in the evolution of the plant 
	pathogen \textit{Phytophthora syringae} by utilizing population genomic 
	tools to analyze genetic differentiation within and among nursery 
	populations.
	\newline{}\newline{}
	Details:
	\begin{itemize}
		\item Designed simulation analyses for populations of partially clonal diploid organisms
		\item Authored R package for genetic analysis of organisms with mixed reproduction (sexual/clonal)\\ (\url{https://github.com/grunwaldlab/poppr})
		\item Isolated, maintained, and extracted DNA of \textit{Phytophthora syringae} for the purposes of Genotyping By Sequencing.
		\item \textbf{Research Advisor: Dr. Niklaus J. Gr\"unwald}
	\end{itemize}
}

\cventry{Aug--Dec 2011}{Rotation}{Jaiswal Lab}{OSU, Corvallis}{OR}{
	Engaged in various research projects combining bioinformatic-based text mining of databases, wet lab, and greenhouse work. \textbf{Research Advisor: Dr. Pankaj Jaiswal}	
}

\cventry{2006--2007}{Undergraduate Research Assistant}{Biology Discipline}{TSU}{Kirksville, MO}{
	As part of a team of undergraduate students, contributed to the annotation of over 2,000 maize genes determined by microarray hybridization analysis to be differentially regulated in the \textit{Zea mays} shoot apical meristem.
	\newline{}\newline{}
	Details:
	\begin{itemize}
		\item Became proficient in performing and interpreting BLAST and InterProScan searches on sequences, identifying and assessing pertinent primary literature, and using a variety of databases to determine the putative function of maize genes.
		\item Collaborated with other researchers on the same project.
		\item \textbf{Research Mentors: Drs. Brent Buckner and Diane Janick-Buckner}
	\end{itemize}
}


%
% \section{Computational Experience}
% 
I am an experienced software developer of several packages for population
genetics in the R programming language listed below. My goal in software
development is to ensure that my applications are \textbf{open},
\textbf{reproducible}, and \textbf{accessible} to researchers regardless of
computational skills.



\subsection{Developer Practices}

\cvitem{}{Version Control (git), Continuous Integration, Unit Testing, Memory Profiling}

\subsection{Frameworks}

\cvitem{}{Sun Grid Engine, SLURM workload manager, Docker, Open Science Framework}

\subsection{Languages}

\cvitem{Expert}{\textsc{R}}
\cvitem{Competent}{\textsc{Make}, \LaTeX, \textsc{bash}, \textsc{python}, \textsc{C}, \textsc{perl}}
\cvitem{Can Patch}{\textsc{JavaScript}}

%
% \section{Authored Software}
% 
%----------------------------------------------------------------------------------------
%	SOFTWARE SECTION
%----------------------------------------------------------------------------------------

\section{Software}

\subsection{Main Author}

\cvitem{poppr}{R package for analysis of populations with mixed reproductive modes}

\subsection{Contributor}

\cvitem{adegenet}{R package for multivariate analysis of population genetics}
\cvitem{phytophthora-id}{Web application for identification of clonal lineages of two \textit{Phytophthora} species}

%
% \section{Teaching}
% 
%----------------------------------------------------------------------------------------
%	TEACHING EXPERIENCE SECTION
%----------------------------------------------------------------------------------------




% Insert something about LENGTH of workshop. THanks Brian.


\cventry{2014}{Population Genetics in R}{Workshop}{}{}{
	I wrote and instructed a 4 hour workshop with Drs. Niklaus Gr\"unwald and Sydney Everhart. This workshop introduces tools and concepts that allow researchers to easily perform population genetic analyses in the R statistical environment. \url{http://grunwaldlab.cgrb.oregonstate.edu/popgen}
	\newline{}\newline{}
	Sessions:
	\begin{itemize} 
		\item \textit{May 17, 2014} Oregon State University
		\item \textit{August 9, 2014} American Phytopathology Society (APS) 2014 National Conference
	\end{itemize}
}

\cventry{Spring 2012}{Graduate Teaching Assistant}{Biology Dept.}{OSU}{Corvallis, OR}{
	Lead laboratories of $\sim$48 students in organismal diversity, organ systems, plant and animal physiology, genetics, evolution and ecology.
	\newline{}\newline{}
	Responsibilities:
	\begin{itemize}
		\item Developed introductory presentations for quizzes and labs
		\item Proctored all tests and quizzes
		\item Graded assignments and provided students with timely feedback
		\item Held office hours once a week
	\end{itemize}
}

\cventry{2009--2011}{English Instructor}{Herald NIE}{Joong-Dong}{Daegu, South Korea}{
	Taught basic to intermediate English to Korean students ranging from elementary to middle school with an emphasis on task-based learning techniques. 
	\newline{}\newline{}
	Details:
	\begin{itemize}
		\item Took charge of 18 different classes per week
		\item Monitored language acquisition of each student via monthly evaluations based on interviews and speaking tests
		\item Wrote tests, assigned and graded homework pertinent to the level of the students. Initiated and mediated interesting topics for discussion courses
	\end{itemize}
}

\cventry{2008--2009}{English Instructor}{GnB English}{Sangin-2-Dong}{Daegu, South Korea}{
	Taught basic to intermediate English to Korean students ranging from elementary to middle school in tandem with one of the nine Korean English teachers at the academy. 
	\newline{}\newline{}
	Details:
	\begin{itemize}
		\item Assisted with at least 30 different classes per week
		\item Monitored language acquisition of students throughout the year
		\item Gained the ability to be prepared for sudden changes in cirriculum and classroom size. 
	\end{itemize}
}

\cventry{Fall 2006/07}{Undergraduate Teaching Assistant}{Biology Discipline}{TSU}{Kirksville, MO}{
	Appointed as teaching assistant for undergraduate cell biology course.
	\newline{}\newline{}
	Details:
	\begin{itemize}
		\item Helped prepare instructional labs for students of Dr. Diane Janick-Buckner's Cell Biology class
		\item Responded to student lab questions and referred to professor questions outside of my expertise/knowledge base
	\end{itemize}
}

%
% \section{Spoken Languages}
% 
\cvitemwithcomment{English}{Native}{}
\cvitemwithcomment{Korean}{Intermediate}{Can manage basic conversation}
%
% \section{Outreach, Service, and Extracurricular Activities}
% 
%----------------------------------------------------------------------------------------
%	OUTREACH SECTION
%----------------------------------------------------------------------------------------


\cventry{2012--Present}{Radio Host}{Inspiration Dissemination}{KBVR FM, OSU}{Corvallis}{Co-created, produced, and hosted a weekly radio show interviewing graduate students in STEM fields about their research and experiences in graduate school.
\newline{}\newline{}
Details:
\begin{itemize}
	\item Provided opportunity for graduate students to present their research in a unique form of outreach.
	\item Actively worked with graduate students to improve their science communication skills.
	\item Assisted undergraduate media students in gaining real world audio post-production experience.
\end{itemize}
}

\cventry{2012--Present}{Active Contributor}{Bioinformatics Users Group}{OSU}{}{
	Contributed presentations and discussions relevant to use of bioinformatics tools such as workflows in the R statistical environment.
}

% \cventry{2014--Present}{Steward}{Botany and Plant Pathology}{Coalition of Graduate Employees}{}{
% 	Represented and informed Botany and Plant Pathology graduate student workers about 
% }
\cventry{2012--2014}{Treasurer}{Graduate Student Association}{Department of Botany and Plant Pathology}{OSU}{
	Balanced the budget, served on bi-annual travel awards committee, helped organize and coordinate group social functions.
}

\cventry{Summer 2005}{Summer Station Manager}{}{KTRM FM, TSU}{Kirksville, MO}{
	I was the primary authority on personnel decisions, after input from team members. I organized the weekly schedule of DJs, determined the salaries of station directors and balanced a budget.
}

\cventry{2004--2007}{Radio Announcer}{}{KTRM FM, TSU}{Kirksville, MO}{
	I ensured successful operation of the transmitter, covered extra scheduled shifts to ensure KTRM stayed on air, and selected appropriate play-lists for listeners.
}


%
% \section{Awards}
% \cvitem{2016}{Botany and Plant Pathology Grad. Student Assoc. Travel Award -- \$200}
\cvitem{2016}{APS Foundation Student Travel Award -- \$500}
\cvitem{2015}{APS Pacific Division Travel Award -- \$500}
\cvitem{2015}{OSU Graduate School Travel Award -- \$500}
\cvitem{2015}{NESCent Travel award for Population Genetics in R Hackathon -- \$650}
\cvitem{2014}{OSU Botany and Plant Pathology Anita Summers Travel Award -- \$1000}
\cvitem{2014}{Most Innovative [Radio] Program -- Intercollegiate Broadcasting System}
\cvitem{2013}{Seattle Institute For Statistical Genetics Travel Award -- \$450}
\cvitem{2006}{Truman State University Summer Research Stipend -- \$3000}
\cvitem{2003}{Truman State University Presidential Leadership Scholarship -- \$2000}

%
% % Sections in files
% \section{Peer Review}

Molecular Ecology, Methods in Ecology and Evolution

\section{Peer Reviewed Publications}

\begin{enumerate}[leftmargin = 14pt]

	\item Jombart T, Archer F, Schliep K, \textbf{Kamvar Z}, Harris R, Paradis
	E, Goudet J, Lapp H (2016). apex: phylogenetics with multiple genes.
	Molecular Ecology Resources. doi:
	\href{http://dx.doi.org/10.1111/1755-0998.12567}{10.1111/1755-0998.12567}

	\item \textbf{Kamvar ZN}, L\'opez-Uribe MM, Coughlan S, Gr\"unwald NJ, Lapp
	H, Manel S (2016). Developing educational resources for population genetics
	in R: an open and collaborative approach. Molecular ecology resources. doi:
	\href{http://dx.doi.org/10.1111/1755-0998.12558}{10.1111/1755-0998.12558}

	\item Gr\"unwald NJ, Larsen MM, \textbf{Kamvar ZN}, Reeser PW, Kanaskie A,
	Laine J, Wiese R (2015) First report of the EU1 clonal lineage of
	\textit{Phytophthora ramorum} on tanoak in an OR forest. 
	\textit{Plant Disease}. 100:5, 1024-1024. doi:
	\href{http://dx.doi.org/10.1094/PDIS-10-15-1169-PDN}{10.1094/PDIS-10-15-1169-PDN}

	\vspace{6pt}

	\item \textbf{Kamvar ZN}, Brooks JC and Gr\"unwald NJ (2015) Novel R tools for
	analysis of genome-wide population genetic data with emphasis on clonality.
	\textit{Front. Genet.} \textbf{6}: 208. doi: 
	\href{http://dx.doi.org/10.3389/fgene.2015.00208}{10.3389/fgene.2015.00208}

	\vspace{6pt}

	\item \textbf{Kamvar ZN}, Larsen MM, Kanaskie AM, Hansen EM, and Gr\"unwald
	NJ. (2015) Spatial and temporal analysis of populations of the sudden oak
	death pathogen in Oregon forests. \textit{Phytopathology}. \textbf{105}:
	982-989. doi: 
	\href{http://dx.doi.org/10.1094/PHYTO-12-14-0350-FI}{10.1094/PHYTO-12-14-0350-FI}.
	
	\vspace{6pt}

	\item Weiland JE, Garrido PA, \textbf{Kamvar ZN}, Marek SM, Gr\"unwald NJ, and
	Garz\'on CD. (2015) Population structure of \textit{Pythium irregulare}, \textit{P.
	sylvaticum}, and \textit{P. ultimum} in forest nursery soils of Oregon and
	Washington. \textit{Phytopathology}. \textbf{105}: 684-694. doi: 
	\href{http://dx.doi.org/10.1094/PHYTO-05-14-0147-R}{10.1094/PHYTO-05-14-0147-R}

	\vspace{6pt}

    \item \textbf{Kamvar ZN}, Tabima JF, Gr\"unwald NJ. (2014) \textit{Poppr}: an
	R package for genetic analysis of populations with clonal, partially clonal,
	and/or sexual reproduction. PeerJ \textbf{2}: e281.
	doi: \href{http://dx.doi.org/10.7717/peerj.281}{10.7717/peerj.281}
	
	\vspace{6pt}

	\item Buckner B, Beck J, Browning, K, Hoxha E, Grantham L, \textbf{Kamvar
	ZN}, Lough A, Nikolova O, and Schnable PS, Scanlon MJ, and Janick-Buckner D.
	(2007) Involving undergraduates in the annotation and analysis of global
	gene expression studies: creation of a maize shoot apical meristem
	expression database. \textit{Genetics}
	\textbf{176}: 741-747. doi: 
	\href{http://dx.doi.org/10.1534/genetics.106.066472}{10.1534/genetics.106.066472}

\end{enumerate}



\section{Contributed Presentations}


\begin{enumerate}[leftmargin = 14pt]

	\item \textbf{Kamvar ZN}, Larsen MM, Kanaskie AM, Hansen EM, Gr\"unwald NJ.
	(2015) Evidence for at least two introductions of the sudden oak death
	pathogen into Oregon forests. American Phytopathological Society National
	Conference, Pasadena, CA.

	\vspace{6pt}

	\item \textbf{Kamvar ZN}, Tabima JF, Gr\"unwald NJ. (2014) Application of
	the R package poppr for analysis of population genetic data. American
	Phytopathological Society National Conference, Minneapolis, MN.

	\vspace{6pt}

	\item \textbf{Kamvar ZN} (2013) Ph.D. Proposal Seminar: Determination of
	pattern and process in the evolution of the plant pathogen
	\textit{Phytophthora syringae}. Department of Botany and Plant Pathology,
	Oregon State University, Corvallis, OR.

	\vspace{6pt}

	\item \textbf{Kamvar ZN} (2013) \textit{Poppr}: An R package for genetic
	analysis of populations with mixed (clonal/sexual) reproduction. Biology
	Graduate Student Symposium, Hatfield Marine Science Center, Newport OR.

	\vspace{6pt}

	\item \textbf{Kamvar ZN}, Tabima JF, Gr\"unwald NJ (2013) \textit{Poppr}: An
	R package for genetic analysis of populations with mixed (clonal/sexual)
	reproduction. Fungal Genetics Conference, Asilomar, CA.

	\vspace{6pt}

	\item \textbf{Kamvar ZN}, Gr\"unwald NJ (2012) \textit{Poppr}: An R package
	for popualtion genetic analysis. OSU Fall CGRB Conference, Oregon State
	University, Corvallis, OR.

	\vspace{6pt}

	\item Browning K, Fritz A, Hoxha E, and \textbf{Kamvar ZN} (2007) Annotation
	and analysis of global gene expression studies: creation of a maize shoot
	apical meristem expression database, Maize Genetics Conference, St. Charles,
	IL.

	\vspace{6pt}

	\item Browning K, Fritz A, Hoxha E, and \textbf{Kamvar ZN} (2007) Annotation
	and analysis of global gene expression studies: creation of a maize shoot
	apical meristem expression database, Truman Student Research Conference,
	Truman State University, Kirksville, MO.

\end{enumerate}

%----------------------------------------------------------------------------------------
%	COVER LETTER
%----------------------------------------------------------------------------------------

% To remove the cover letter, comment out this entire block

% \clearpage

% \recipient{HR Departmnet}{Corporation\\123 Pleasant Lane\\12345 City, State} % Letter recipient
% \date{\today} % Letter date
% \opening{Dear Sir or Madam,} % Opening greeting
% \closing{Sincerely yours,} % Closing phrase
% \enclosure[Attached]{curriculum vit\ae{}} % List of enclosed documents

% \makelettertitle % Print letter title

% \lipsum[1-3] % Dummy text

% \makeletterclosing % Print letter signature

%----------------------------------------------------------------------------------------

\end{document}

\end{document}
