%%%%%%%%%%%%%%%%%%%%%%%%%%%%%%%%%%%%%%%%%
% "ModernCV" CV and Cover Letter
% LaTeX Template
% Version 1.1 (9/12/12)
%
% This template has been downloaded from:
% http://www.LaTeXTemplates.com
%
% Original author:
% Xavier Danaux (xdanaux@gmail.com)
%
% License:
% CC BY-NC-SA 3.0 (http://creativecommons.org/licenses/by-nc-sa/3.0/)
%
% Important note:
% This template requires the moderncv.cls and .sty files to be in the same
% directory as this .tex file. These files provide the resume style and themes
% used for structuring the document.
%
%%%%%%%%%%%%%%%%%%%%%%%%%%%%%%%%%%%%%%%%%

%----------------------------------------------------------------------------------------
%	PACKAGES AND OTHER DOCUMENT CONFIGURATIONS
%----------------------------------------------------------------------------------------

\documentclass[$fontsize$,letterpaper,sans]{moderncv} % Font sizes: 10, 11, or 12; paper sizes: a4paper, letterpaper, a5paper, legalpaper, executivepaper or landscape; font families: sans or roman

\moderncvstyle{classic} % CV theme - options include: 'casual' (default), 'classic', 'oldstyle' and 'banking'
\moderncvcolor{blue} % CV color - options include: 'blue' (default), 'orange', 'green', 'red', 'purple', 'grey' and 'black'
%
% % LAYOUT
% %--------------------------------
% Margins
\usepackage{geometry}
\geometry{$geometry$}
%
% Do not indent paragraphs
\setlength\parindent{0in}
%
% % Enable multicolumns
% \usepackage{multicol}
% \setlength{\columnsep}{-3.5cm}

% Uncomment to suppress page numbers
% \pagenumbering{gobble}
%
% LANGUAGE
%--------------------------------
% Set the main language
$if(lang)$
\usepackage{polyglossia}
\setmainlanguage{$lang$}
$endif$

% TYPOGRAPHY
%--------------------------------
\usepackage{fontspec}
\usepackage{xunicode}
\usepackage{xltxtra}
% converts LaTeX specials (quotes, dashes etc.) to Unicode
\defaultfontfeatures{Mapping=tex-text}
\setromanfont [Ligatures={Common}, Numbers={OldStyle}]{$mainfont$}
% Cool ampersand
\newcommand{\amper}{{\fontspec[Scale=.95]{$mainfont$}\selectfont\itshape\&}}

\usepackage{enumitem}
\usepackage{lipsum} % Used for inserting dummy 'Lorem ipsum' text into the template
\usepackage{natbib}
\usepackage{bibentry}
% \usepackage[scale=0.75]{geometry} % Reduce document margins
%\setlength{\hintscolumnwidth}{3cm} % Uncomment to change the width of the dates column
%\setlength{\makecvtitlenamewidth}{10cm} % For the 'classic' style, uncomment to adjust the width of the space allocated to your name

\newcommand{\bibverse}[1]{\begin{verse} \bibentry{#1}. \end{verse} }
%----------------------------------------------------------------------------------------
%	NAME AND CONTACT INFORMATION SECTION
%----------------------------------------------------------------------------------------

\firstname{$name.first$} % Your first name
\middlename{$name.middle$}
\familyname{$name.last$} % Your last name
\email{$name.email$}
% All information in this block is optional, comment out any lines you don't need
\title{Curriculum Vitae}
\address$for(address)${$address$}$endfor$
\mobile{$phone$}



% Uncomment for black and white links
%\AfterPreamble{\hypersetup{colorlinks,urlcolor=black,linkcolor=black}}
%----------------------------------------------------------------------------------------
\begin{document}

% If you want a cover letter to go with it. Note that this cover letter will be included
% in the pagination.
% 
%----------------------------------------------------------------------------------------
%	COVER LETTER
%----------------------------------------------------------------------------------------

% To remove the cover letter, comment out this entire block

\clearpage

\recipient{HR Departmnet}{Corporation\\123 Pleasant Lane\\12345 City, State} % Letter recipient
\date{\today} % Letter date
\opening{Dear Sir or Madam,} % Opening greeting
\closing{Sincerely yours,} % Closing phrase
\enclosure[Attached]{curriculum vit\ae{}} % List of enclosed documents

\makelettertitle % Print letter title

\lipsum[1-3] % Dummy text

\makeletterclosing % Print letter signature

%----------------------------------------------------------------------------------------



% \makecvtitle % Print the CV title

$if(software)$
\section{Authored Software}

\subsection{R Packages}
$for(software.R)$
\cvitemwithcomment{$software.R.role$}{\href{$software.R.url$}{$software.R.name$}}{$software.R.description$}
$endfor$

\subsection{Genomics}
$for(software.Genomics)$
\cvitemwithcomment{$software.Genomics.role$}{\href{$software.Genomics.url$}{$software.Genomics.url$}}{$software.Genomics.description$}
$endfor$

\subsection{Websites}
$for(software.Websites)$
\cvitemwithcomment{$software.Websites.role$}{\href{$software.Websites.url$}{$software.Websites.url$}}{$software.Websites.description$}
$endfor$

$endif$

$if(awards)$
\section{Awards}
$for(awards)$
\cvitem{$awards.year$}{$awards.award$ -- $awards.amount$}
$endfor$
$endif$



% \section{Education}
% 
%----------------------------------------------------------------------------------------
%	EDUCATION SECTION
%----------------------------------------------------------------------------------------

\section{Education}

\cventry{2011--Present}{Ph.D. Plant Pathology}{Oregon State University (OSU)}{Corvallis, OR}{}{Expected 2017}  % Arguments not required can be left empty
\cventry{2007}{B.S. Biology}{Truman State University (TSU)}{Kirksville, MO}{}{Minor: Chemistry}


%
% \section{Employment}
% 
%----------------------------------------------------------------------------------------
%	WORK EXPERIENCE SECTION
%----------------------------------------------------------------------------------------


\cventry{2012--2016}{Thesis Research}{Gr\"unwald Lab}{OSU, Corvallis}{OR}{
	I focused on developing software tools for analyzing the population genetics
	of clonal organisms and demonstrating their applications in a reproducible
	manner.
	\newline{}\newline{}
	Details:
	\begin{itemize}
		\item Authored R package for genetic analysis of organisms with mixed reproduction (sexual/clonal) (\url{https://github.com/grunwaldlab/poppr})
		\item Designed simulation analyses for populations of partially clonal diploid organisms
		\item Isolated, maintained, and extracted DNA of \textit{Phytophthora syringae} for the purposes of Genotyping By Sequencing.
		\item Analyzed the outbreak of the Sudden Oak Death pathogen, \textit{Phytophthora ramorum} in Curry County, OR.
		\item \textbf{Research Advisor: Dr. Niklaus J. Gr\"unwald}
	\end{itemize}
}

\cventry{Aug--Dec 2011}{Rotation}{Jaiswal Lab}{OSU, Corvallis}{OR}{
	Engaged in various research projects combining bioinformatic-based text mining of databases, wet lab, and greenhouse work. \textbf{Research Advisor: Dr. Pankaj Jaiswal}	
}

\cventry{2006--2007}{Undergraduate Research Assistant}{Biology Discipline}{TSU}{Kirksville, MO}{
	As part of a team of undergraduate students, contributed to the annotation of over 2,000 maize genes determined by microarray hybridization analysis to be differentially regulated in the \textit{Zea mays} shoot apical meristem.
	\newline{}\newline{}
	Details:
	\begin{itemize}
		\item Became proficient in performing and interpreting BLAST and InterProScan searches on sequences, identifying and assessing pertinent primary literature, and using a variety of databases to determine the putative function of maize genes.
		\item Collaborated with other researchers on the same project.
		\item \textbf{Research Mentors: Drs. Brent Buckner and Diane Janick-Buckner}
	\end{itemize}
}


%
% \section{Computational Experience}
% 
I am an experienced devops engineer of several packages for population
genetics, epidemiology, and static web publishing in the R programming language
listed below. My goal in software development is to ensure that my applications
are \textbf{open}, and \textbf{accessible} to researchers regardless of
computational skills. Above all, I aim to make sure that my software is robust,
consistent, and correct through careful design and test-driven development
practices.

\subsection{Developer Practices}

\cvitem{}{Version Control (git), Continuous Integration (GitHub Actions), Unit Testing, Memory Profiling}

\subsection{Cloud/HPC Frameworks}

\cvitem{}{AWS, Sun Grid Engine, SLURM workload manager, Docker, Open Science Framework}

\subsection{Languages}

\cvitem{Expert}{\textsc{R}}
\cvitem{Competent}{\textsc{bash}, \textsc{python}, \textsc{XPath}, \textsc{XSLT}, \textsc{C}, \textsc{Make}, \LaTeX, \textsc{perl}}
\cvitem{Can Patch}{\textsc{JavaScript}}
\cvitem{Learning}{\textsc{Rust}, \textsc{SuperCollider}}

%
% \section{Authored Software}
% 
%----------------------------------------------------------------------------------------
%	SOFTWARE SECTION
%----------------------------------------------------------------------------------------




\cvitemwithcomment{Main Author:}{\href{http://cran.r-project.org/package=poppr}{poppr}}{R package for analysis of populations with mixed reproductive modes}

\cvitemwithcomment{}{\href{http://cran.r-project.org/package=popprxl}{popprxl}}{R package for providing data import into poppr from excel spreadsheets.}

\cvitemwithcomment{}{\href{http://cran.r-project.org/package=ezec}{ezec}}{R package for an easy interface to effective concentration calculations}

\cvitem{}{}

\cvitemwithcomment{Contributor:}{\href{http://cran.r-project.org/package=adegenet}{adegenet}}{R package for multivariate analysis of population genetics}

\cvitemwithcomment{}{\href{http://phytophthora-id.org}{phytophthora-id}}{Web application for identification of clonal lineages of two \textit{Phytophthora} species}

\cvitemwithcomment{}{\href{http://microbe-id.org}{microbe-id}}{Template web application for creating species identification web services}

\cvitemwithcomment{}{\href{http://cran.r-project.org/package=mmod}{mmod}}{R package for modern methods of differentiation}

\cvitemwithcomment{}{\href{http://cran.r-project.org/package=heirfstat}{hierfstat}}{R package for estimation of hierarchical F-statistics}

\cvitemwithcomment{}{\href{http://cran.r-project.org/package=vcfR}{vcfR}}{R package for visualization and manipulation of variant call format files}

\cvitemwithcomment{}{\href{http://cran.r-project.org/package=apex}{apex}}{R package for phylogenetic analysis with multiple genes}


%
% \section{Teaching}
% 
%----------------------------------------------------------------------------------------
%	TEACHING EXPERIENCE SECTION
%----------------------------------------------------------------------------------------
\cventry{October 2017}{Reproducible data science for population genetics}{Workshop}{}{}{
	The goal of this workshop was to introduce professionals in the biological sciences to the practices of reproducible population genetic data analysis using R.
	Organized by \textit{PR}Statistics. Course Website: \url{https://goo.gl/amBbph}
	\newline{}\newline{}
	Presented from \textit{2017-10-23} to \textit{2017-10-27} at Margam Discovery Centre, Wales
}
\cventry{Summer 2017}{Introduction to R}{Workshop}{}{}{
	I wrote and instructed a three-hour workshop with Dr. Sydney E. Everhart. The goal of the workshop was to give a basic introduction to the R language that included reading data, writing data, and producing figures. \url{https://everhartlab.github.io/IntroR/}
	\newline{}\newline{}
	Sessions:
	\begin{itemize}
		\item \textit{June 14, 2017} North-Central American Phytopathological Society (NCAPS) 2017 Annual Meeting
		\item \textit{May 24, 2017} University of Nebraska-Lincoln
	\end{itemize}
}

\cventry{Summer 2016}{Reproducible Research in R}{Workshop}{}{}{
	I wrote and instructed a three-hour workshop with Zachary S. L. Foster and Dr. Niklaus Gr\"unwald. The goal of the workshop was to present plant pathologists with the basic tools necessary for performing reproducible science within the R environment. \url{http://grunwaldlab.github.io/Reproducible-science-in-R/}
	\newline{}\newline{}
	Presented on \textit{August 1, 2016} at American Phytopathology Society (APS) 2016 National Conference
}


\cventry{Spring 2016}{Graduate Teaching Assistant}{Botany Dept.}{OSU}{Corvallis, OR}{
	Taught introductory Botany for non-majors focused on emphasizing the role of
	plants in the environment, agriculture, and society. Two labs of $\sim$30 students.
	\newline{}\newline{}
	Responsibilities:
	\begin{itemize}
		\item Developed lectures, prepared materials, and wrote quizzes for labs each week
		\item Proctored all tests and quizzes
		\item Graded assignments and provided students with timely feedback
		\item Held office hours once a week
	\end{itemize}
}

\cventry{2014--2015}{Population Genetics in R}{Workshop}{}{}{
	I wrote and instructed a 4 hour workshop with Drs. Niklaus Gr\"unwald and Sydney Everhart. This workshop introduces tools and concepts that allow researchers to easily perform population genetic analyses in the R statistical environment. \url{http://grunwaldlab.cgrb.oregonstate.edu/popgen}
	\newline{}\newline{}
	Sessions:
	\begin{itemize}
		\item \textit{August 1, 2015} American Phytopathology Society (APS) 2015 National Conference
		\item \textit{August 9, 2014} American Phytopathology Society (APS) 2014 National Conference
		\item \textit{May 17, 2014} Oregon State University
	\end{itemize}
}

\cventry{Winter 2012}{Graduate Teaching Assistant}{Biology Dept.}{OSU}{Corvallis, OR}{
	Lead laboratories of $\sim$48 students in organismal diversity, organ systems, plant and animal physiology, genetics, evolution and ecology.
	\newline{}\newline{}
	Responsibilities:
	\begin{itemize}
		\item Developed introductory presentations for quizzes and labs
		\item Proctored all tests and quizzes
		\item Graded assignments and provided students with timely feedback
		\item Held office hours once a week
	\end{itemize}
}

\cventry{2009--2011}{English Instructor}{Herald NIE}{Joong-Dong}{Daegu, South Korea}{
	Taught basic to intermediate English to Korean students ranging from elementary to middle school with an emphasis on task-based learning techniques.
	\newline{}\newline{}
	Details:
	\begin{itemize}
		\item Took charge of 18 different classes per week
		\item Monitored language acquisition of each student via monthly evaluations based on interviews and speaking tests
		\item Wrote tests, assigned and graded homework pertinent to the level of the students. Initiated and mediated interesting topics for discussion courses
	\end{itemize}
}

\cventry{2008--2009}{English Instructor}{GnB English}{Sangin-2-Dong}{Daegu, South Korea}{
	Taught basic to intermediate English to Korean students ranging from elementary to middle school in tandem with one of the nine Korean English teachers at the academy.
	\newline{}\newline{}
	Details:
	\begin{itemize}
		\item Assisted with at least 30 different classes per week
		\item Monitored language acquisition of students throughout the year
		\item Gained the ability to be prepared for sudden changes in cirriculum and classroom size.
	\end{itemize}
}

\cventry{Fall 2006/07}{Undergraduate Teaching Assistant}{Biology Discipline}{TSU}{Kirksville, MO}{
	Appointed as teaching assistant for undergraduate cell biology course.
	\newline{}\newline{}
	Details:
	\begin{itemize}
		\item Helped prepare instructional labs for students of Dr. Diane Janick-Buckner's Cell Biology class
		\item Responded to student lab questions and referred to professor questions outside of my expertise/knowledge base
	\end{itemize}
}

%
% \section{Spoken Languages}
% 
\cvitemwithcomment{English}{Mother tongue}{}
\cvitemwithcomment{Korean}{Intermediate}{Can manage basic conversation}
%
% \section{Outreach, Service, and Extracurricular Activities}
% 
%----------------------------------------------------------------------------------------
%	OUTREACH SECTION
%----------------------------------------------------------------------------------------


\cventry{2012--2016}{Radio Host}{Inspiration Dissemination}{KBVR FM, OSU}{Corvallis}{Co-created, produced, and hosted a weekly radio show interviewing graduate students in STEM fields about their research and experiences in graduate school.
\newline{}\newline{}
Details:
\begin{itemize}
	\item Provided opportunity for graduate students to present their research in a unique form of outreach.
	\item Actively worked with graduate students to improve their science communication skills.
	\item Assisted undergraduate media students in gaining real world audio post-production experience.
	\item Wrote weekly blog post about the week's show at \url{http://blogs.oregonstate.edu/inspiration}
\end{itemize}
}

\cventry{2012--2016}{Active Contributor}{Bioinformatics Users Group}{OSU}{}{
	Contributed presentations and discussions relevant to use of bioinformatics tools such as workflows in the R statistical environment.
}

% \cventry{2014--Present}{Steward}{Botany and Plant Pathology}{Coalition of Graduate Employees}{}{
% 	Represented and informed Botany and Plant Pathology graduate student workers about 
% }
\cventry{2012--2014}{Treasurer}{Graduate Student Association}{Department of Botany and Plant Pathology}{OSU}{
	Balanced the budget, served on bi-annual travel awards committee, helped organize and coordinate group social functions.
}

\cventry{Summer 2005}{Summer Station Manager}{}{KTRM FM, TSU}{Kirksville, MO}{
	I was the primary authority on personnel decisions, after input from team members. I organized the weekly schedule of DJs, determined the salaries of station directors and balanced a budget.
}

\cventry{2004--2007}{Radio Announcer}{}{KTRM FM, TSU}{Kirksville, MO}{
	I ensured successful operation of the transmitter, covered extra scheduled shifts to ensure KTRM stayed on air, and selected appropriate play-lists for listeners.
}


%
% \section{Awards}
% 
\cvitem{2015}{APS Pacific Division Travel Award -- \$500}
\cvitem{2015}{NESCent Travel award for Population Genetics in R Hackathon -- \$650}
\cvitem{2014}{OSU Botany and Plant Pathology Anita Summers Travel Award -- \$1000}
\cvitem{2014}{Most Innovative [Radio] Program -- Intercollegiate Broadcasting System}
\cvitem{2013}{Seattle Institute For Statistical Genetics Travel Award -- \$450}
\cvitem{2006}{Truman State University Summer Research Stipend -- \$3000}
\cvitem{2003}{Truman State University Presidential Leadership Scholarship -- \$2000}

%
% % Sections in files
% 
\section{Peer Reviewed Publications}

\begin{enumerate}[leftmargin = 14pt]

	\item \textbf{Kamvar ZN}, Larsen MM, Kanaskie AM, Hansen EM, and Gr\"unwald
	NJ. Spatial and temporal population dynamics of the sudden oak death
	epidemic in Oregon forests. \textit{Phytopathology}. \textbf{submitted.}

	\vspace{6pt}

	\item Weiland JE, Garrido PA, \textbf{Kamvar ZN}, Marek SM, Gr\"unwald NJ, and
	Garz\'on CD. Population structure of \textit{Pythium irregulare}, \textit{P.
	sylvaticum}, and \textit{P. ultimum} in forest nursery soils of Oregon and
	Washington. \textit{Phytopathology}. \textbf{in press}.

	\vspace{6pt}

    \item \textbf{Kamvar ZN}, Tabima JF, Gr\"unwald NJ. (2014) \textit{Poppr}: an
	R package for genetic analysis of populations with clonal, partially clonal,
	and/or sexual reproduction. PeerJ \textbf{2}: e281
	\url{http://dx.doi.org/10.7717/peerj.281}
	
	\vspace{6pt}

	\item Buckner B, Beck J, Browning, K, Hoxha E, Grantham L, \textbf{Kamvar
	ZN}, Lough A, Nikolova O, and Schnable PS, Scanlon MJ, and Janick-Buckner D.
	(2007) Involving undergraduates in the annotation and analysis of global
	gene expression studies: creation of a maize shoot apical meristem
	expression database. \textit{Genetics}
	\textbf{176}: 741-747 \url{http://dx.doi.org/10.1534/genetics.106.066472}

\end{enumerate}



\section{Contributed Presentations}


\begin{enumerate}[leftmargin = 14pt]

	\item \textbf{Kamvar ZN}, Tabima JF, Gr\"unwald NJ. (2014) Application of
	the R package poppr for analysis of population genetic data. American
	Phytopathological Society National Conference, Minneapolis, MN.

	\vspace{6pt}

	\item \textbf{Kamvar ZN} (2013) Ph.D. Proposal Seminar: Determination of
	pattern and process in the evolution of the plant pathogen
	\textit{Phytophthora syringae}. Department of Botany and Plant Pathology,
	Oregon State University, Corvallis, OR.

	\vspace{6pt}

	\item \textbf{Kamvar ZN} (2013) \textit{Poppr}: An R package for genetic
	analysis of populations with mixed (clonal/sexual) reproduction. Biology
	Graduate Student Symposium, Hatfield Marine Science Center, Newport OR.

	\vspace{6pt}

	\item \textbf{Kamvar ZN}, Tabima JF, Gr\"unwald NJ (2013) \textit{Poppr}: An
	R package for genetic analysis of populations with mixed (clonal/sexual)
	reproduction. Fungal Genetics Conference, Asilomar, CA.

	\vspace{6pt}

	\item \textbf{Kamvar ZN}, Gr\"unwald NJ (2012) \textit{Poppr}: An R package
	for popualtion genetic analysis. OSU Fall CGRB Conference, Oregon State
	University, Corvallis, OR.

	\vspace{6pt}

	\item Browning K, Fritz A, Hoxha E, and \textbf{Kamvar ZN} (2007) Annotation
	and analysis of global gene expression studies: creation of a maize shoot
	apical meristem expression database, Maize Genetics Conference, St. Charles,
	IL.

	\vspace{6pt}

	\item Browning K, Fritz A, Hoxha E, and \textbf{Kamvar ZN} (2007) Annotation
	and analysis of global gene expression studies: creation of a maize shoot
	apical meristem expression database, Truman Student Research Conference,
	Truman State University, Kirksville, MO.

\end{enumerate}

%----------------------------------------------------------------------------------------
%	COVER LETTER
%----------------------------------------------------------------------------------------

% To remove the cover letter, comment out this entire block

% \clearpage

% \recipient{HR Departmnet}{Corporation\\123 Pleasant Lane\\12345 City, State} % Letter recipient
% \date{\today} % Letter date
% \opening{Dear Sir or Madam,} % Opening greeting
% \closing{Sincerely yours,} % Closing phrase
% \enclosure[Attached]{curriculum vit\ae{}} % List of enclosed documents

% \makelettertitle % Print letter title

% \lipsum[1-3] % Dummy text

% \makeletterclosing % Print letter signature

%----------------------------------------------------------------------------------------

\end{document}

\end{document}
