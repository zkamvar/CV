
%----------------------------------------------------------------------------------------
%	WORKING GROUP SECTION
%----------------------------------------------------------------------------------------

\cventry{2016}{Hackout3}{Hackathon}{Berkeley Institute of Data Science}{June 20--24}{
	 Hackout 3 brought together field epidemiologists, decision makers, modelers
	 and computer scientists to create free, open-source resources for the 
	 real-time monitoring of disease outbreaks. 
\newline{}\newline{}
Highlights:
\begin{itemize}
	\item R Epidemics Consortium: http://www.repidemicsconsortium.org/
	\item Tools for detecting near duplicates in line list data designed for data managers on the ground.
	\item Tools for data cleaning, and disease modeling.
\end{itemize}
}

\cventry{2014}{Population Genetics in R}{Hackathon}{NESCent}{March 16--20}{
	 The event aimed to help foster an interoperating ecosystem of scalable
	 tools and resources for population genetics data analysis in the popular R
	 platform.
\newline{}\newline{}
Highlights:
\begin{itemize}
	\item Website for community contributed tutorials on population genetic analysis in R \url{http://popgen.nescent.org} 
	\item New tools for analyzing multiple gene phylogenies in R \url{http://cran.r-project.org/package=apex}
	\item New contributions to the core data structure of the \textit{adegenet} package
	\item Special issue of Molecular Ecology: \textbf{Population Genomics in R}
\end{itemize}
}
