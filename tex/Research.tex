
%----------------------------------------------------------------------------------------
%	WORK EXPERIENCE SECTION
%----------------------------------------------------------------------------------------


\cventry{2012--Present}{Thesis Research}{Gr\"unwald Lab}{OSU, Corvallis}{OR}{
	My goal is to determine pattern and process in the evolution of the plant 
	pathogen \textit{Phytophthora syringae} by utilizing population genomic 
	tools to analyze genetic differentiation within and among nursery 
	populations.
	\newline{}\newline{}
	Details:
	\begin{itemize}
		\item Designed simulation analyses for populations of partially clonal diploid organisms
		\item Authored R package for genetic analysis of organisms with mixed reproduction (sexual/clonal)\\ (\url{https://github.com/grunwaldlab/poppr})
		\item Isolated, maintained, and extracted DNA of \textit{Phytophthora syringae} for the purposes of Genotyping By Sequencing.
		\item \textbf{Research Advisor: Dr. Niklaus J. Gr\"unwald}
	\end{itemize}
}

\cventry{Aug--Dec 2011}{Rotation}{Jaiswal Lab}{OSU, Corvallis}{OR}{
	Engaged in various research projects combining bioinformatic-based text mining of databases, wet lab, and greenhouse work. \textbf{Research Advisor: Dr. Pankaj Jaiswal}	
}

\cventry{2006--2007}{Undergraduate Research Assistant}{Biology Discipline}{TSU}{Kirksville, MO}{
	As part of a team of undergraduate students, contributed to the annotation of over 2,000 maize genes determined by microarray hybridization analysis to be differentially regulated in the \textit{Zea mays} shoot apical meristem.
	\newline{}\newline{}
	Details:
	\begin{itemize}
		\item Became proficient in performing and interpreting BLAST and InterProScan searches on sequences, identifying and assessing pertinent primary literature, and using a variety of databases to determine the putative function of maize genes.
		\item Collaborated with other researchers on the same project.
		\item \textbf{Research Mentors: Drs. Brent Buckner and Diane Janick-Buckner}
	\end{itemize}
}

