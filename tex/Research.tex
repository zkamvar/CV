
%----------------------------------------------------------------------------------------
%	WORK EXPERIENCE SECTION
%----------------------------------------------------------------------------------------
\cventry{2018--Present}{Postdoctoral Research}{Jombart Lab}{ICL, London}{UK}{
	I develop tools and resources for outbreak analytics in all aspects
	from data validation, analysis, visualization, and reporting.
	\begin{itemize}
    \item Collaborated with MSF (Doctors Without Borders) to produce standardized
      and automated situation report templates for field epidemiologists
		\item Created and contributed to packages for outbreak response
    \item Taught workshops introducing R to field epidemiologists
		\item \textbf{Lab Head: Dr. Thibaut Jombart}
	\end{itemize}
}
\cventry{2017--2018}{Postdoctoral Research}{Everhart Lab}{UNL, Lincoln}{NE}{
	My projects involve understanding the population genetics and genomics of 
	\textit{Sclerotinia sclerotiorum}, the causal agent of white mold on dry bean.
	\begin{itemize}
		\item Assessed genetic diversity of populations of \textit{S. sclerotiorum} across the 
		United States within white mold screening nurseries and producer fields.
		\item Created a pipeline for assembly and filtering of 55 fungicide-exposed isolates of 
		\textit{S. sclerotiorum}. 
		\item Taught and created a workshop introducing students in the agricultural sciences to 
		data analysis in R. 
		\item \textbf{Lab Head: Dr. Sydney E. Everhart}
	\end{itemize}
}

\cventry{2012--2016}{Thesis Research}{Gr\"unwald Lab}{OSU, Corvallis}{OR}{
	I focused on developing software tools for analyzing the population genetics
	of clonal organisms and demonstrating their applications in a reproducible
	manner.
	\begin{itemize}
		\item Authored R package for genetic analysis of organisms with mixed reproduction (sexual/clonal) (\url{https://github.com/grunwaldlab/poppr})
		\item Designed simulation analyses for populations of partially clonal diploid organisms
		\item Isolated, maintained, and extracted DNA of \textit{Phytophthora syringae} for the purposes of Genotyping By Sequencing.
		\item Analyzed the outbreak of the Sudden Oak Death pathogen, \textit{Phytophthora ramorum} in Curry County, OR.
		\item \textbf{Research Advisor: Dr. Niklaus J. Gr\"unwald}
	\end{itemize}
}

\cventry{Aug--Dec 2011}{Rotation}{Jaiswal Lab}{OSU, Corvallis}{OR}{
	Engaged in various research projects combining bioinformatic-based text mining of databases, wet lab, and greenhouse work. \textbf{Research Advisor: Dr. Pankaj Jaiswal}	
}

\cventry{2006--2007}{Undergraduate Research Assistant}{Biology Discipline}{TSU}{Kirksville, MO}{
	As part of a team of undergraduate students, contributed to the annotation of over 2,000 maize genes determined by microarray hybridization analysis to be differentially regulated in the \textit{Zea mays} shoot apical meristem.
	\begin{itemize}
		\item Became proficient in performing and interpreting BLAST and InterProScan searches on sequences, identifying and assessing pertinent primary literature, and using a variety of databases to determine the putative function of maize genes.
		\item Collaborated with other researchers on the same project.
		\item \textbf{Research Mentors: Drs. Brent Buckner and Diane Janick-Buckner}
	\end{itemize}
}

