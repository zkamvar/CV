
%----------------------------------------------------------------------------------------
%	WORK EXPERIENCE SECTION
%----------------------------------------------------------------------------------------
\cventry{2020--2023}{Lesson Infrastructure Developer}{The Carpentries}{}{remote}{
  I redesigned and transitioned the underlying build process for more than 50 open
  source and active lessons, which serve more than 10,000 learners annualy for
  \href{https://carpentries.org}{The Carpentries} volunteer community. This
  culminated in the creation and deployment of
  \href{https://carpentries.github.io/workbench}{The Carpentries Workbench}, a
  portable lesson infrastructure, which separated the content of lessons from
  the build and styling components that has significantly improved the quality
  of life for the more than 100 Lesson Developers and Maintainer volunteers in The
  Carpentries. \\
  \\
  \textbf{Products}
  \begin{itemize}
    \item \href{https://carpentries.github.io/workbench}{The Carpentries
      Workbench} --- A suite of R packages and GitHub Actions to build,
      maintain, validate, and deploy accessible lessons for data and coding
      skills.
    \item \href{https://docs.ropensci.org/tinkr}{tinkr} --- An R package in
      collaboration with \href{https://ropensci.org}{rOpenSci} that transforms
      Markdown to XML and back again. This package is used for validation of
      markdown elements without regular expressions and is the engine behind
      the experimental \href{https://docs.ropensci.org/babeldown}{babeldown} 
      package, which provides automated human language translation of Markdown
      documents using the DeepL API.
    \item \href{https://github.com/carpentries/actions}{Carpentries GitHub
      Actions} --- Tools for securely provisioning and deploying Carpentries
      Lessons with arbitrary code on GitHub.
    \item \href{https://github.com/carpentries/lesson-transition}{Lesson Transition Tool} ---
      Automated and flexible transition between Carpentries-style lesson (built
      with Jekyll) to The Carpentries Workbench. This rearranged folder structure,
      updated Markdown syntax, and removed unneeded tooling and generated output
      to reduce the size of the repository and allow the git history to reflect
      authorship.
  \end{itemize}
  \textbf{Timeline}
  \begin{itemize}
    \item \textbf{R\&D (2020)} --- \textbf{Research} of the fragmented landscape of
      build strategies for lessons and, iterating on feedback from the
      community, \textbf{designed} an initial prototype for the lesson
      infrastructure with deployment strategies.
    \item \textbf{Alpha Testing (2021)} --- \textbf{Successful alpha test} of prototype,
      \textbf{directed} initial design of website frontend, automated safe and
      reproducible rendering of literate programming in lessons, and
      \textbf{implemented} first iteration of workflow to automate transition
      of lessons to new infrastructure.
    \item \textbf{Beta Testing (2022)} --- \textbf{Released} The Workbench to the community,
      acheived full feature pairity with the former infrastructure, gathered
      and \textbf{iterated on feedback} from early adopters,
      \textbf{coordinated} and initiated \textbf{beta testing} phase with official
      lesson Maintainers and Instructors, and \textbf{strategically planned} to
      seamlessly transition more than 50 official lessons by Spring 2023.
    \item \textbf{Rollout and Capacity Building (2023)} --- Finalized beta testing phase, \textbf{coordinated
      the seamless and secure rollout} of The Workbench to all lessons in
      April--May 2023. \textbf{Increased capacity} within The Core Team by
      \textbf{training} three colleagues in R package development and creating
      \href{https://carpentries.github.io/workbench-dev}{comprehensive
      developer documentation} about The Workbench design and testing
      practices.
    \item \textbf{Supervisor: Dr. Robert R. Davey}
  \end{itemize}
}
\cventry{2018--2020}{Research Software Engineer}{Jombart Lab}{ICL, London}{UK}{
  I develop tools and resources for outbreak analytics in all aspects from data
  validation, analysis, visualization, and reporting.
	\begin{itemize}
    \item Collaborated with MSF (Doctors Without Borders) in the R4EPIs project
      to produce standardized and automated situation report templates for
      field epidemiologists
		\item Created and contributed to packages for outbreak response
    \item Taught workshops introducing R to field epidemiologists
		\item \textbf{Lab Head: Dr. Thibaut Jombart}
	\end{itemize}
}
\cventry{2017--2018}{Postdoctoral Research}{Everhart Lab}{UNL, Lincoln}{NE}{
	My projects involve understanding the population genetics and genomics of 
	\textit{Sclerotinia sclerotiorum}, the causal agent of white mold on dry bean.
	\begin{itemize}
		\item Assessed genetic diversity of populations of \textit{S. sclerotiorum} across the 
		United States within white mold screening nurseries and producer fields.
		\item Created a pipeline for assembly and filtering of 55 fungicide-exposed isolates of 
		\textit{S. sclerotiorum}. 
		\item Taught and created a workshop introducing students in the agricultural sciences to 
		data analysis in R. 
		\item \textbf{Lab Head: Dr. Sydney E. Everhart}
	\end{itemize}
}

\cventry{2012--2016}{Thesis Research}{Gr\"unwald Lab}{OSU, Corvallis}{OR}{
	I focused on developing software tools for analyzing the population genetics
	of clonal organisms and demonstrating their applications in a reproducible
	manner.
	\begin{itemize}
		\item Authored R package for genetic analysis of organisms with mixed reproduction (sexual/clonal) (\url{https://github.com/grunwaldlab/poppr})
		\item Designed simulation analyses for populations of partially clonal diploid organisms
		\item Isolated, maintained, and extracted DNA of \textit{Phytophthora syringae} for the purposes of Genotyping By Sequencing.
		\item Analyzed the outbreak of the Sudden Oak Death pathogen, \textit{Phytophthora ramorum} in Curry County, OR.
		\item \textbf{Research Advisor: Dr. Niklaus J. Gr\"unwald}
	\end{itemize}
}

\cventry{Aug--Dec 2011}{Rotation}{Jaiswal Lab}{OSU, Corvallis}{OR}{
	Engaged in various research projects combining bioinformatic-based text mining of databases, wet lab, and greenhouse work. \textbf{Research Advisor: Dr. Pankaj Jaiswal}	
}

\cventry{2006--2007}{Undergraduate Research Assistant}{Biology Discipline}{TSU}{Kirksville, MO}{
	As part of a team of undergraduate students, contributed to the annotation of over 2,000 maize genes determined by microarray hybridization analysis to be differentially regulated in the \textit{Zea mays} shoot apical meristem.
	\begin{itemize}
		\item Became proficient in performing and interpreting BLAST and InterProScan searches on sequences, identifying and assessing pertinent primary literature, and using a variety of databases to determine the putative function of maize genes.
		\item Collaborated with other researchers on the same project.
		\item \textbf{Research Mentors: Drs. Brent Buckner and Diane Janick-Buckner}
	\end{itemize}
}

