I have taken a leading role in the following inter-organizational collaborations
\newline{}

\cventry{2018--Present}{The R4EPIs Project}{with RECON and MSF}{}{}{
  This collaboration between MSF (Doctors Without Borders) and the RECON (R
  Epidemics Consortium) aims to provide a set of situation report templates in
  RMarkdown with standardized data cleaning and summary statistics for reports
  on four common outbreak scenarios and three surveys.
  \begin{itemize}
    \item In a constant, iterative process with field epidemiologists, we
      discussed expectations of input data and appropriate summaries, developed
      tools or found solutions as appropriate. These were tested on simulated
      and real-world data by the end users (field epidemiologists). 
    \item Developed system for reviewing templates developed by the technical
      team and worked with the technical leads to review the templates in a fair
      and equitable manner
    \item Wrote tests for numerical and structural accuracy.
    \item Lead remote collaborative sessions with epidemiologists testing
      templates on their own data, updating templates and code to avoid common
      pitfalls.
    \item Trained contributors in package development and github flow to
      standardize collaboration
    \item Website: \url{https://r4epis.netlify.com}
  \end{itemize}
}

\cventry{2018--Present}{R Epidemics Consortium}{}{}{}{
  This consortium brings together field epidemiologists, software developers, 
  educators, and modellers to help nurture and develop the ecosystem of 
  resources in R for epidemiolgical analysis and modelling. My role as secretary
  and software development coordinator is to ensure that R packages and
  educational resources developed in the consortium are accessible and follow
  best practices. 
  \begin{itemize}
    \item Roles: Secretary and Software Development Coordinator (2018--2020)
    \item Maintainer of: \url{https://reconlearn.org}, \href{https://www.repidemicsconsortium.org/aweek}{aweek}, and \href{https://www.repidemicsconsortium.org/incidence}{incidence}
    \item Expanded \href{https://www.repidemicsconsortium.org/resources/guidelines/}{guidelines for package development and github etiquitte}
    \item Coordinated and ran several workshops for introduction to R
  \end{itemize}
}

\newpage
I was invited to the following collaborative hackathons:
\newline{}

\cventry{2016}{Hackout3}{Hackathon}{Berkeley Institute of Data Science}{June 20--24}{
	 Hackout 3 brought together field epidemiologists, decision makers, modelers
	 and computer scientists to create free, open-source resources for the 
	 real-time monitoring of disease outbreaks. 
  \begin{itemize}
    \item Tools for detecting near duplicates in line list data designed for data managers on the ground.
    \item Tools for data cleaning, and disease modeling.
    \item Creation of the R Epidemics Consortium: \url{http://www.repidemicsconsortium.org/}
  \end{itemize}
}

\cventry{2014}{Population Genetics in R}{Hackathon}{NESCent}{March 16--20}{
	 The event aimed to help foster an interoperating ecosystem of scalable
	 tools and resources for population genetics data analysis in the popular R
	 platform.
  \begin{itemize}
    \item Website for community contributed tutorials on population genetic analysis in R \url{http://popgen.nescent.org} 
    \item New tools for analyzing multiple gene phylogenies in R \url{http://cran.r-project.org/package=apex}
    \item New contributions to the core data structure of the \textit{adegenet} package
    \item Special issue of Molecular Ecology: \textbf{Population Genomics in R}
  \end{itemize}
}
