
%----------------------------------------------------------------------------------------
%	COVER LETTER
%----------------------------------------------------------------------------------------

% To remove the cover letter, comment out this entire block

\clearpage
\begin{flushright}
  Dr. Zhian N. Kamvar\\
  Department of Infectious Disease Epidemiology\\
  Imperial College London, W2 1NY, UK\\
  \today
\end{flushright}
\recipient{Javier Luraschi}{RStudio MLverse team} % Letter recipient
\date{\today} % Letter date
\opening{Dear Javier Luraschi} % Opening greeting
\closing{Sincerely,} % Closing phrase
\enclosure[Attached]{curriculum vit\ae{}, references} % List of enclosed documents



I am applying for the Software Engineer position in the RStudio mlverse team. I
saw your tweet advertizing the position\footnote{https://twitter.com/javierluraschi/status/1188896398789398529},
and realized that this is the position I've been looking for. I am a research
software engineer at Imperial College London developing R packages, report
templates, and trainings for field epidemiologists in outbreak scenarios. My 
experience developing R packages for diverse audiences, discipline in
collaborative software engineering practices, and eagerness to learn and apply
new skillsets makes me an ideal candidate for this position. Having worked as 
both as a scientist and software engineer, I am excited for the opportunity to 
join the mlverse team and find new ways to make machine learning more accessible
to R users. 

\vspace{1ex}

I've been excited about the prospect of working for RStudio because they are
part of the driving force that made R more accessible to everyone. Moreover,
the company culture of remote work, collaboration, and sensible time off is a
welcome respite from moving every two years in the academic sector. The
software engineer position is especially enticing for me because of the
opportunity to streamline a complex 

\vspace{1ex}

My qualifications: 

 - Shipping R packages with C backends (poppr)
 - Worked with data science tools such as docker containers in cloud-based infastructure (Kamvar et al. 2017)
 - I am an autodidact and will learn new approaches/technologies to acheive my goals.
 - Since 2013 I have worked openly and collaboratively on open source software and data analyses. 

    % Experience shipping professional software including CRAN packages.
    % Experience in Scala or C/C++.
    % Experience in data science, machine learning or distributed computing.
    % Ability to work autonomously and independently on difficult problems.


% Desired qualifications

    % Committer in the Spark project.
    % Experience releasing and maintaining CRAN packages.
    % Experience in R, Python or Julia.
    % Experience working in open source projects.

Why you want to hire me:

 - I have six years of experience maintaining open source software, both new and
   legacy code in both R and C.
 - I know how to work with data scientists to write user stories that allow me
   to develop packages that are both user-friendly, robust, and reliable.
 - I have experience working remotely with a diverse team using GitHub flow and 
   continuous integration.
 - I know how to learn new skill sets to solve unique challenges (I learned C
   to address performance issues in R and I learned Make to create a genomic
   data analysis pipeline).


\vspace{1ex}

Collaboaration is an important part of my research in that I have never worked
on a single project without contributing to someone else's. In the case of
\textit{poppr}, I contributed heavily to the \textit{adegenet} package during
and after a 2015 NESCENT Population Genetics in R hackathon. In the case of
Kamvar \textit{et al.} 2017, I was able to track down and fix a
\href{https://github.com/slowkow/ggrepel/issues/72}{hidden bug} in the
\textit{ggrepel} package. Most recently, I am one of the technical leads on a
collaboration between the R Epidemics Consortium (RECON) and M\'{e}decins Sans
Fronti\`{e}res (MSF) called R4EPIs where we work on standardizing analyses and
training for field epidemiologists.  

\vspace{1ex}

As my carreer focus has shifted from investigative research to scientific
software engineering, I believe this project will be immensely beneficial to my
trajectory as a research software engineer. My background in quantitative
science, robust software development, collaborative capacity, and passion for
both equity and fairness in peer review makes me well-suited to join the team
as a software research scientist. Thank you for your time and consideration, I
look forward to hearing from you. 

\vspace{2ex}

Sincerely,

\vspace{5ex}

\textbf{Zhian N. Kamvar}\\
\textit{Attached: curriculum vit\ae{}, references}
\clearpage

%----------------------------------------------------------------------------------------


% EcoHealth Alliance seeks a creative, dedicated, and collaborative software
% research scientist to support a two-year project in launching a new software
% peer-review initiative. The software research scientist will work on the
% Sloan Foundation supported rOpenSci project, with rOpenSci staff and a
% statistical methods editorial board. They will research and develop standards
% and review guidelines for statistical software, publish findings, and develop
% R software to test packages against those standards. The software research
% scientist will work with staff and the board to collaborate broadly with the
% statistical and software communities to gather input, refine and promote the
% standards, and recruit editors and peer reviewers. The candidate must be
% self-motivated, proactive, collaborative and comfortable working openly and
% reproducibly with a broad online community.

% DESCRIPTION AND RESPONSIBILITIES

% - Research standards and protocols for evaluating statistical software
%     quality and correctness, and the extent of their adoption
% - Create new testing frameworks for R packages
% - Assist rOpenSci staff and project board members in drafting new
%     peer-review evaluation standards, guidelines, and documentation
% - Write technical and scientific papers, documentation, and blog posts
% - Assist in organizing peer-review system for scientific software and
%     managing the review board
% - Participate in and represent the rOpenSci project in person and via
%     on-line fora
% - Participate in other projects and tasks as required or assigned by
%     supervisor

% MINIMUM QUALIFICATIONS

% - A Master's degree in statistics, computer science, or a related field with
%     a focus on quantitative methodologies, or equivalent experience in
%     statistical methods evaluation and development
% - Expertise in open-source R package development, including collaborative
%     development using Git and GitHub, testing frameworks, and continuous
%     integration
% - Strong writing skills
% - Experience in collaborative team projects and consensus building
% - A passion for improving scientific reproducibility

% DESIRED QUALIFICATIONS

% - A PhD in statistics, computer science, or a related field with a focus on
%     quantitative methodologies, or equivalent experience in statistical
%     methods evaluation and development
% - Published scientific or technical articles or software documentation



