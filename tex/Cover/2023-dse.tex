
%----------------------------------------------------------------------------------------
%	COVER LETTER
%----------------------------------------------------------------------------------------

% To remove the cover letter, comment out this entire block

\clearpage
\begin{flushright}
  Dr. Zhian N. Kamvar\\
  The Carpentries\\
  (c/o Community Initiatives)\\
  1000 Broadway, Suite \#480\\
  Oakland, CA 94607, USA
  \today
\end{flushright}

\textbf{Program Manager}\\
Eric and Wendy Schmidt Center for Data Science and Environment,

% Responsibilities

%  - Plans, designs, develops, modifies, debugs, deploys and evaluates data
%  science/computational science research software technologies and
%  visualizations.
%  - Works closely with Program Manager, Senior Research Software Engineers, and
%  domain scientists where applicable to perform these responsibilities and uses
%  domain science knowledge where relevant.
%  - Analyzes existing software, scientific/data science code and
%  interactive/static visualizations or works to formulate new logic for existing
%  systems and algorithms.
%  - Works closely with Program Manager, Senior Research Software Engineers and
%  the program team to develop data-enabled software solutions to a given
%  environmental challenge.
%  - Understands and applies open research and development practices, community
%  standards and department policies and procedures relating to work assignments.
%  - May serve as technical lead for a research and development project of
%  moderate scope.
%  - Has the ability to negotiate research and development project plans with
%  interested collaborators, stakeholders and users as applicable.
%  - Works both independently and in collaboration with the program team and
%  stakeholders to design and implement technical solutions to environmental
%  challenges.
%  - Works closely with the team Program Manager to prioritize work needs and
%  opportunities to address given needs.
%  - Documents usage modes, capabilities, characteristics and performance of
%  research codes/software/ data visualizations.
%  - Publishes and presents, based on research and development work, results
%  about data science techniques and tools, performance and algorithm enhancement
%  in research venues to attract usage from domain science communities, or to
%  promote latest technologies within the community.
%  - Communicates technical program details effectively with a wide variety of
%  audiences to ensure understanding and engagement with program stakeholders.
%
% Required Qualifications

%  - Intermediate knowledge of open data science/research software development.
%  - Advanced skills, and demonstrated experience associated with deployment of data science applications, tools, and visualizations.
%  - Demonstrated ability to regularly interface with management.
%  - Demonstrated ability to contribute research and technical content to program outputs.
%
%  - Demonstrated effective communication and interpersonal skills.
%  - Demonstrated ability to communicate technical information to technical and non-technical personnel at various levels in the organization and to external research and education audiences.
%
%  - Proven skills and experience in independently resolving broad computing/data problems using introductory and/or intermediate principles.
%  - Self-motivated and works independently and as part of a team.
%  - Able to learn effectively and meet deadlines.
%  - Proven ability to successfully work on multiple concurrent projects.
%
%  - Proven ability to understand research computing/data needs, mapping use cases to requirements and how systems/software/infrastructure can support those needs and meet the requirements. Demonstrated ability to develop and implement such solutions.
%  - Demonstrated experience and ability to collaborate effectively with all levels of staff; technical, students, faculty and administrators.
%  - Bachelor's degree in related area and/or equivalent experience/training.

% Preferred Qualifications

%  - Experience or interest in contributing to the development of applications with environmental impact.
%  - Experience developing scientific software packages and working in an open software environment.
%  - Experience with the modern Python or R stacks for geospatial/environmental problems, and/or javascript tools for interactive scientific visualization.

\vspace{2ex}

I found the Research Software Engineer/Data Scientist position via
recommendation by Dr. Rich FitzJohn and I am confident that I will excel in
this role because in my \textbf{10 years in interdisciplinary open source
scientific software development}, I have always focused on usability,
reproducibility, and flexibility in my design.
I am a software engineer at The Carpentries where I just finished developing
infrastructure that supports researchers and educators in developing lessons
for data science training. 
My background in \textbf{science communication}, discipline in collaborative
software engineering practices (test driven development, CI/CD, trunk-based
workflows, and project management), and \textbf{eagerness to learn and apply
new skillsets} makes me an ideal candidate for this position. 
I am particularly excited to work in a role that will support researchers in
understand the environmental challenges so they can make critical decisions
that will have real impacts for our world.

\vspace{2ex}

% Qualifications:
%  - RSE development
%  - deployment
%  - communication
%  - interdisciplinary
%  - vendor neutral
%  - environmental science

My skill set lies in the intersection of software development,
reproducible research, open science, and communication. 
\textbf{I have been collaboratively developing open source software on GitHub
since 2013}.
My most recent project is The Carpentries Workbench\footnote{Workbench user manual: \url{https://carpentries.github.io/workbench}}, 
a suite of R packages designed to build, deploy, and audit \textbf{reproducible
data science lessons} built with R Markdown in a \textbf{platform indpendent} manner\footnote{Workbench developer's guide: \url{https://carpentries.github.io/workbench-dev/intro.html\#sec-remote}}.
This was a ground-up redesign of the lesson infrastructure to \textbf{focus on
the needs and working practices of our diverse community of volunteers},
allowing them to focus on the content of their lesson and not the tooling.

% In addition to working as a research software engineer in the fields of
% population genomics, epidemiology, and education, I have training and
% experience in science communication and am able to 

% I am currently finalizing my role developing and deploying reproducible lesson
% infrastructure for The Carpentries that allows lesson authors to focus on
% content, not tools.

% Over my carreer, I have often found myself to be the sole team member with
% software engineering skills. 

% Summary: 
%  - what I am applying for
%  - why I'm a good candidate
%  - why I am excite


\vspace{2ex}

The work I did in academia gave me all the experience to produce 
\textbf{reproducible research}\footnote{Reproducible Research using CI + Docker (Kamvar \textit{et al.}, 2017) doi: \href{https://doi.org/10.7717/peerj.4152}{10.7717/peerj.4152}}
and user-friendly scientific software\footnote{poppr R package (Kamvar \textit{et al.}, 2014) doi: \href{https://doi.org/10.7717/peerj.281}{10.7717/peerj.281}}.
My most successful software project is the R package
\textit{poppr}, which has been \textbf{featured in \textgreater1500
peer-reviewed scientific publications}. I strongly believe this project
continues to be successful because I took a community-centered approach in its
maintenance. With human-centered design, clear documentation, tutorials,
workshops, and diligent forum moderation, I worked to \textbf{significantly
reduce the barrier for reproducible population genetic analysis in R}.

\vspace{2ex}

These alone meet many of the qualifications for this position, but I believe my
work at The Carpentries provides a set of critical skills that will set me
apart from other candidates. At The Carpentries, I was able to hone my
\textbf{skills in communication and DevOps} while developing valuable
\textbf{project management} techniques that allowed me to effectively
coordinate the \textbf{simultaneous development and deployment} of 4 R
packages, a suite of GitHub actions, and the seamless transition\footnote{Automated lesson transition: \url{https://github.com/carpentries/lesson-transition\#readme}}
of > 50 active lessons maintained by > 100 volunteer maintainers, serving >
10,000 learners. anually. 

\vspace{2ex}

% Summarize above
% - indpendent, open source, communication

The experience I have gained over the last decade has given me the technical and
practical experience needed to be a successful research software engineer. I am
excited for the opportunity to work in a team context building tools that will
support researchers with data-enabled solutions to environmental challenges to
make real-world impact on biodiversity and climate resiliance. I would like to
thank the recruitment team for consideration of my application.

\vspace{3ex}

Sincerely,

\vspace{4ex}

\textbf{Zhian N. Kamvar, Ph. D.}\\
{\footnotesize \textit{(Attached: Resum\'{e}, references)}}

\clearpage

%----------------------------------------------------------------------------------------


