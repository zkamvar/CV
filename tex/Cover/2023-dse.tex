
%----------------------------------------------------------------------------------------
%	COVER LETTER
%----------------------------------------------------------------------------------------

% To remove the cover letter, comment out this entire block

\clearpage
\begin{flushright}
  Dr. Zhian N. Kamvar\\
  The Carpentries\\
  (c/o Community Initiatives)\\
  1000 Broadway, Suite \#480\\
  Oakland, CA 94607, USA
  \today
\end{flushright}

\textbf{Program Manager}
Eric and Wendy Schmidt Center for Data Science and Environment,

% Responsibilities

%  - Plans, designs, develops, modifies, debugs, deploys and evaluates data
%  science/computational science research software technologies and
%  visualizations.
%  - Works closely with Program Manager, Senior Research Software Engineers, and
%  domain scientists where applicable to perform these responsibilities and uses
%  domain science knowledge where relevant.
%  - Analyzes existing software, scientific/data science code and
%  interactive/static visualizations or works to formulate new logic for existing
%  systems and algorithms.
%  - Works closely with Program Manager, Senior Research Software Engineers and
%  the program team to develop data-enabled software solutions to a given
%  environmental challenge.
%  - Understands and applies open research and development practices, community
%  standards and department policies and procedures relating to work assignments.
%  - May serve as technical lead for a research and development project of
%  moderate scope.
%  - Has the ability to negotiate research and development project plans with
%  interested collaborators, stakeholders and users as applicable.
%  - Works both independently and in collaboration with the program team and
%  stakeholders to design and implement technical solutions to environmental
%  challenges.
%  - Works closely with the team Program Manager to prioritize work needs and
%  opportunities to address given needs.
%  - Documents usage modes, capabilities, characteristics and performance of
%  research codes/software/ data visualizations.
%  - Publishes and presents, based on research and development work, results
%  about data science techniques and tools, performance and algorithm enhancement
%  in research venues to attract usage from domain science communities, or to
%  promote latest technologies within the community.
%  - Communicates technical program details effectively with a wide variety of
%  audiences to ensure understanding and engagement with program stakeholders.
%
% Required Qualifications

%  - Intermediate knowledge of open data science/research software development.
%  - Advanced skills, and demonstrated experience associated with deployment of data science applications, tools, and visualizations.
%  - Demonstrated ability to regularly interface with management.
%  - Demonstrated ability to contribute research and technical content to program outputs.
%  - Demonstrated effective communication and interpersonal skills.
%  - Demonstrated ability to communicate technical information to technical and non-technical personnel at various levels in the organization and to external research and education audiences.
%  - Proven skills and experience in independently resolving broad computing/data problems using introductory and/or intermediate principles.
%  - Self-motivated and works independently and as part of a team.
%  - Able to learn effectively and meet deadlines.
%  - Proven ability to successfully work on multiple concurrent projects.
%  - Proven ability to understand research computing/data needs, mapping use cases to requirements and how systems/software/infrastructure can support those needs and meet the requirements. Demonstrated ability to develop and implement such solutions.
%  - Demonstrated experience and ability to collaborate effectively with all levels of staff; technical, students, faculty and administrators.
%  - Bachelor's degree in related area and/or equivalent experience/training.

% Preferred Qualifications

%  - Experience or interest in contributing to the development of applications with environmental impact.
%  - Experience developing scientific software packages and working in an open software environment.
%  - Experience with the modern Python or R stacks for geospatial/environmental problems, and/or javascript tools for interactive scientific visualization.

\vspace{1ex}

I found the Research Software Engineer/Data Scientist position via
recommendation by Dr. Rich FitzJohn. 
I am currently finalizing my role developing and deploying reproducible lesson
infrastructure for The Carpentries that allows lesson authors to focus on
content, not tools.
I am confident that I will excel in this role because in my \textbf{10 years in
interdisciplinary open source scientific software development}, I have always
focused on usability, reproducibility, and flexibility in my design.
The thing that excites me most about this position is the opportunity to work
with environmental scientists and reduce their technical burden so they can
dedicate more of their time to answering important questions about the health
of our environment.

% Summary: 
%  - what I am applying for
%  - why I'm a good candidate
%  - why I am excite


\vspace{1ex}

% Qualifications:
%  - RSE development
%  - deployment
%  - communication
%  - interdisciplinary
%  - vendor neutral
%  - environmental science


\vspace{1ex}

% Specific Example:
%
%  - The Carpentries Workbench Deployment

\vspace{1ex}

% Collaboration:
%
% - tinkr, Alpha and Beta Tests
%
\vspace{1ex}

% Summarize above
% - indpendent, open source, communication


\vspace{2ex}

Sincerely,

\vspace{3ex}

\textbf{Zhian N. Kamvar, Ph. D.}
{\footnotesize \textit{(Attached: Resum\'{e}, references)}}

\clearpage

%----------------------------------------------------------------------------------------


% EcoHealth Alliance seeks a creative, dedicated, and collaborative software
% research scientist to support a two-year project in launching a new software
% peer-review initiative. The software research scientist will work on the
% Sloan Foundation supported rOpenSci project, with rOpenSci staff and a
% statistical methods editorial board. They will research and develop standards
% and review guidelines for statistical software, publish findings, and develop
% R software to test packages against those standards. The software research
% scientist will work with staff and the board to collaborate broadly with the
% statistical and software communities to gather input, refine and promote the
% standards, and recruit editors and peer reviewers. The candidate must be
% self-motivated, proactive, collaborative and comfortable working openly and
% reproducibly with a broad online community.

% DESCRIPTION AND RESPONSIBILITIES

% - Research standards and protocols for evaluating statistical software
%     quality and correctness, and the extent of their adoption
% - Create new testing frameworks for R packages
% - Assist rOpenSci staff and project board members in drafting new
%     peer-review evaluation standards, guidelines, and documentation
% - Write technical and scientific papers, documentation, and blog posts
% - Assist in organizing peer-review system for scientific software and
%     managing the review board
% - Participate in and represent the rOpenSci project in person and via
%     on-line fora
% - Participate in other projects and tasks as required or assigned by
%     supervisor

% MINIMUM QUALIFICATIONS

% - A Master's degree in statistics, computer science, or a related field with
%     a focus on quantitative methodologies, or equivalent experience in
%     statistical methods evaluation and development
% - Expertise in open-source R package development, including collaborative
%     development using Git and GitHub, testing frameworks, and continuous
%     integration
% - Strong writing skills
% - Experience in collaborative team projects and consensus building
% - A passion for improving scientific reproducibility

% DESIRED QUALIFICATIONS

% - A PhD in statistics, computer science, or a related field with a focus on
%     quantitative methodologies, or equivalent experience in statistical
%     methods evaluation and development
% - Published scientific or technical articles or software documentation



