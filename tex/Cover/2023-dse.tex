
%----------------------------------------------------------------------------------------
%	COVER LETTER
%----------------------------------------------------------------------------------------

% To remove the cover letter, comment out this entire block

\clearpage
\begin{flushright}
  Dr. Zhian N. Kamvar\\
  The Carpentries\\
  (c/o Community Initiatives)\\
  1000 Broadway, Suite \#480\\
  Oakland, CA 94607, USA
  \today
\end{flushright}

\textbf{Program Manager}
Eric and Wendy Schmidt Center for Data Science and Environment,

% Responsibilities

%  - Plans, designs, develops, modifies, debugs, deploys and evaluates data
%  science/computational science research software technologies and
%  visualizations.
%  - Works closely with Program Manager, Senior Research Software Engineers, and
%  domain scientists where applicable to perform these responsibilities and uses
%  domain science knowledge where relevant.
%  - Analyzes existing software, scientific/data science code and
%  interactive/static visualizations or works to formulate new logic for existing
%  systems and algorithms.
%  - Works closely with Program Manager, Senior Research Software Engineers and
%  the program team to develop data-enabled software solutions to a given
%  environmental challenge.
%  - Understands and applies open research and development practices, community
%  standards and department policies and procedures relating to work assignments.
%  - May serve as technical lead for a research and development project of
%  moderate scope.
%  - Has the ability to negotiate research and development project plans with
%  interested collaborators, stakeholders and users as applicable.
%  - Works both independently and in collaboration with the program team and
%  stakeholders to design and implement technical solutions to environmental
%  challenges.
%  - Works closely with the team Program Manager to prioritize work needs and
%  opportunities to address given needs.
%  - Documents usage modes, capabilities, characteristics and performance of
%  research codes/software/ data visualizations.
%  - Publishes and presents, based on research and development work, results
%  about data science techniques and tools, performance and algorithm enhancement
%  in research venues to attract usage from domain science communities, or to
%  promote latest technologies within the community.
%  - Communicates technical program details effectively with a wide variety of
%  audiences to ensure understanding and engagement with program stakeholders.
%
% Required Qualifications

%  - Intermediate knowledge of open data science/research software development.
%  - Advanced skills, and demonstrated experience associated with deployment of data science applications, tools, and visualizations.
%  - Demonstrated ability to regularly interface with management.
%  - Demonstrated ability to contribute research and technical content to program outputs.
%  - Demonstrated effective communication and interpersonal skills.
%  - Demonstrated ability to communicate technical information to technical and non-technical personnel at various levels in the organization and to external research and education audiences.
%  - Proven skills and experience in independently resolving broad computing/data problems using introductory and/or intermediate principles.
%  - Self-motivated and works independently and as part of a team.
%  - Able to learn effectively and meet deadlines.
%  - Proven ability to successfully work on multiple concurrent projects.
%  - Proven ability to understand research computing/data needs, mapping use cases to requirements and how systems/software/infrastructure can support those needs and meet the requirements. Demonstrated ability to develop and implement such solutions.
%  - Demonstrated experience and ability to collaborate effectively with all levels of staff; technical, students, faculty and administrators.
%  - Bachelor's degree in related area and/or equivalent experience/training.

% Preferred Qualifications

%  - Experience or interest in contributing to the development of applications with environmental impact.
%  - Experience developing scientific software packages and working in an open software environment.
%  - Experience with the modern Python or R stacks for geospatial/environmental problems, and/or javascript tools for interactive scientific visualization.

\vspace{1ex}

I found the Research Software Engineer/Data Scientist position via
recommendation by Dr. Rich FitzJohn and I am confident that I will excel in
this role because in my \textbf{10 years in interdisciplinary open source
scientific software development}, I have always focused on usability,
reproducibility, and flexibility in my design.
I am a software engineer at The Carpentries where I just finished developing
infrastructure that supports researchers and educators in developing lessons
for data science training. 
My background in science communication, \textbf{10 years experience developing
open source software and tutorials for diverse communities}, discipline in
collaborative software engineering practices (test driven development, CI/CD,
trunk-based workflows, and project management), and eagerness to learn and
apply new skillsets makes me an ideal candidate for this position.

In addition to working as a research software engineer in the fields of
population genomics, epidemiology, and education, I have training and
experience in science communication and am able to 

I am currently finalizing my role developing and deploying reproducible lesson
infrastructure for The Carpentries that allows lesson authors to focus on
content, not tools.

Over my carreer, I have often found myself to be the sole team member with
software engineering skills. 

I am excited for the opportunity to work in a team context building tools that
will support researchers with data-enabled solutions to environmental
challenges to make real-world impact on biodiversity and climate resiliance.
% Summary: 
%  - what I am applying for
%  - why I'm a good candidate
%  - why I am excite


\vspace{1ex}

% Qualifications:
%  - RSE development
%  - deployment
%  - communication
%  - interdisciplinary
%  - vendor neutral
%  - environmental science

My skill set lies in the intersection between software development,
reproducible research, open science, and communication. 
\textbf{I have been collaboratively developing open source software on GitHub
since 2013}.
My most recent project is The Carpentries Workbench, a suite of R packages
designed to build, deploy, and audit data science lessons built with R Markdown
in a platform indpendent manner. Taking a community-centered approach, I used
feedback about the former infrastructure to design The Workbench and a series
of GitHub workflows to automate validation and deployment of the lessons.
This met our volunteers where they were, allowing them to focus on the content
of their lesson and not the tooling.

\vspace{1ex}

One clear example of my focus on usability was working with the maintainers of
the Raster and Vector Geospatial Data in R lesson. The old infrastructure would
automatically update packages in the user library when a lesson was run. This
caused new errors and warnings in the lessons when the spaial R packages changed
their API and maintainers were afraid to build the lesson locally. I implemented
a non-invasive package cache where packages could be automatically be pinned and
updated with the ability to audit changes on a monthly basis.
% --- maybe not 


\vspace{1ex}

My most successful software project is the R package
\textit{poppr}, which has been \textbf{featured in \textgreater1500
peer-reviewed scientific publications}. I strongly believe this project 
continues to be successful because I took a community-centered approach. With
human-centered design, clear documentation, tutorials, workshops, and diligent
forum moderation, I worked to \textbf{significantly reduce the barrier
for population genetic analysis in R}.


\vspace{1ex}

% Specific Example:
%
%  - The Carpentries Workbench Deployment

\vspace{1ex}

% Collaboration:
%
% - tinkr, Alpha and Beta Tests
%
\vspace{1ex}

% Summarize above
% - indpendent, open source, communication


\vspace{2ex}

Sincerely,

\vspace{3ex}

\textbf{Zhian N. Kamvar, Ph. D.}
{\footnotesize \textit{(Attached: Resum\'{e}, references)}}

\clearpage

%----------------------------------------------------------------------------------------


