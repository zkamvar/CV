
%----------------------------------------------------------------------------------------
%	COVER LETTER
%----------------------------------------------------------------------------------------

% To remove the cover letter, comment out this entire block

\clearpage
\begin{flushright}
  Dr. Zhian N. Kamvar\\
  Dept. of Infectious Disease Epidemiology\\
  Imperial College London, W2 1NY, UK\\
  \today
\end{flushright}

\textbf{The Carpentries}
(c/o Community Initiatives)\\
1000 Broadway, Suite \#480\\
Oakland, CA 94607, USA

% The Carpentries is committed to “training and fostering an active, inclusive,
% diverse community of learners and instructors who promote and model the
% importance of software and data in research.” We seek an engaged and
% collaborative individual who shares this vision for a full-time position as
% the Lesson Infrastructure Technology Developer for The Carpentries. The
% Lesson Infrastructure Technology Developer has a broad responsibility for
% developing infrastructure and processes to support hosting, curation, and
% discoverability of community-developed lessons.

% As the Lesson Infrastructure Technology Developer, you will create templates,
% infrastructure, and documentation for development of new curricula and
% adaptation of existing curricula to community standards. You will reduce the
% technological barriers to creating and sharing lesson materials by developing
% systems that are accessible to community members from a wide range of
% technical experience levels. The Lesson Infrastructure Technology Developer
% will design systems to enhance discoverability of curricular resources and
% develop clear rating and review systems to help the community assess
% curriculum quality and suitability for their teaching contexts. You will
% develop technical infrastructure to support lesson evaluation, deployment,
% publication, and tracking, and will engage with diverse community members to
% test and iterate on these systems. Depending on your background, skill set
% and interests, you may also contribute to the support of our technical
% infrastructure (e.g., database and systems administration).

% This role is a good fit for someone interested in transitioning from an
% academic environment to a focus on technical work, and who is interested in
% gaining experience in the non-profit space.

% Required:

    % Familiarity with the open-science, open-source community
    % Familiarity with Git and GitHub
    % Familiarity with one or more of the following: Bash, Jekyll, R, Python
    % Familiarity with HTML5, CSS
    % Experience working with the Linux operating system
    % Experience communicating technical subjects to non-technical audiences
    % Demonstrated leadership experience in professional or volunteer settings
    % Excellent written and real-time communication skills
    % Customer-service orientation
    % Ability to work with minimal supervision
    % Excellent time management skills
    % Able to work remotely with a distributed team

% Preferred:

    % Experience with The Carpentries
    % Experience working with a volunteer community
    % Experience contributing to open-source projects
    % Project management skills
    % Experience with continuous integration (e.g. Travis CI)
    % Experience working with metadata schema (e.g. Bioschema)
    % Experience or knowledge of human-centered design and accessibility
    % Experience in a system administrator role
    % Experience with Docker
    % Experience working with web APIs (including GitHub’s)
    % Experience with React or other UI frameworks
    % Experience with RMarkdown or other authoring tools
    % Experience with Amazon Web Services or other cloud providers

\vspace{1ex}
To the recruitment team,

\vspace{1ex}


% 1. In the intro, I'd mention there how impressed you are with the carpentries
% (rather than the end) and try to rephrase a little so that you express that
% not only is the job perfect for you but that you are motivated and would be
% great for the job too. Ie mention an appreciation for carpentry goals.
I saw the position for
\href{https://static.carpentries.org/lesson-infrastructure-technology-developer/}{Lesson
Infastructure Technology Developer} via the ROpenSci slack channel and % I AM EXCITE
I am confident that I will excel in this role because \textbf{the core values of The
Carpentries aligns with my motivations as an scientist and educator.}
I am a research software engineer at Imperial College London developing R
packages, automated workflows, and short-course trainings for field
epidemiologists in outbreak scenarios. My background in science communication,
\textbf{8 years experience developing R packages and tutorials for diverse
communities}, discipline in collaborative software engineering practices (test
driven development, GitHub flow, and continuous integration), and eagerness to
learn and apply new skillsets makes me an ideal candidate for this position.
The work The Carpentries have been doing to make software and data management
in research more accessible is very impressive; as an academic looking to
transition into technical work in the non-profit sector, I'm excited for this
opportunity to apply my skills in open-source software developement,
evidence-based teaching, and communication to the challenge of strenghthening
The Carpentries' infastructure. 

\vspace{1ex}

My skill set lies in the intersection between software development,
reproducible research, open science, and communication.  \textbf{I have been
collaboratively developing open source software on GitHub
since 2013}. My most successful software project is the R package
\textit{poppr}, which has been \textbf{featured in \textgreater500
peer-reviewed scientific publications}. I strongly believe this project 
continues to be successful because I took a community-centered approach. With
human-centered design, clear documentation, tutorials, workshops, and diligent
forum moderation, I worked to \textbf{significantly reduce the barrier
for population genetic analysis in R}.

\vspace{1ex}

My work in lesson and workshop management gives me much of the experience
needed to be effective for this role. \textbf{Over the past 5 years, I have
collaboratively designed 4 introductory workshops} in R programming and
reproducible research to \textbf{non-technical audiences} from the
plant pathology, population genetics, and epidemiology communities. The
majority of the workshop participants reported an increase of confidence with
thier ability to use R for their work.

\vspace{1ex}

I am also \textbf{keenly aware of the infrastructural challenges of curation,
hosting, and discoverability of lessons} via my experience co-maintaining
\url{https://popgen.nescent.org}\footnote{Kamvar, Z.N., López-Uribe, M.M.,
Coughlan, S., Grünwald, N.J., Lapp, H. and Manel, S. (2017), Developing
educational resources for population genetics in R: an open and collaborative
approach. Molecular Ecology Resources, 17: 120-128.
doi:\href{https://doi.org/10.1111/1755-0998.12558}{10.1111/1755-0998.12558}}.
This website hosts \textbf{community-contributed vignettes of population
genetic analyses in R} that are automatically built and deployed through
continuous integration. Our main goal was to ensure that the \textbf{technical
barriers for contributors were minimized or eliminated}.

\vspace{1ex}

For the last year, \textbf{my most rewarding leadership experience} has been as
one of the technical leads on the
R4EPIs\footnote{\url{https://R4EPIs.netlify.com}} project---a collaboration
between the R Epidemics Consortium (RECON) and M\'{e}decins Sans Fronti\`{e}res
(MSF)---\textbf{I have first-hand experience in working remotely with
developers and users} (many of whom were non-technical) to design templates and
eductational materials to get field epidemiologists using R. We used an
iterative human-centered design process to ensure that the templates were built
in a way that epidemiologists were comfortable using. 

\vspace{1ex}

The experience I have gained in the last 5 years have given me the insights and
technical knowledge needed for this position and I am eager for the opportunity
to become part of The Carpentries team. As requested in the instructions, the
country in which I pay taxes is the United States of America. I would like to
thank the recruitment team for considering my application. 

\vspace{2ex}

Sincerely,

\vspace{3ex}

\textbf{Zhian N. Kamvar, Ph. D.}
{\footnotesize \textit{(Attached: Resum\'{e}, references)}}



\clearpage
