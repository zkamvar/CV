
%----------------------------------------------------------------------------------------
%	COVER LETTER
%----------------------------------------------------------------------------------------

% To remove the cover letter, comment out this entire block

\clearpage
\begin{flushright}
  Dr. Zhian N. Kamvar\\
  Dept. of Infectious Disease Epidemiology\\
  Imperial College London, W2 1NY, UK\\
  \today
\end{flushright}

\textbf{The Carpentries}\\
c/o Community Initiatives\\
1000 Broadway, Suite \#480\\
Oakland, CA 94607, USA

% The Carpentries is committed to “training and fostering an active, inclusive,
% diverse community of learners and instructors who promote and model the
% importance of software and data in research.” We seek an engaged and
% collaborative individual who shares this vision for a full-time position as
% the Lesson Infrastructure Technology Developer for The Carpentries. The
% Lesson Infrastructure Technology Developer has a broad responsibility for
% developing infrastructure and processes to support hosting, curation, and
% discoverability of community-developed lessons.

% As the Lesson Infrastructure Technology Developer, you will create templates,
% infrastructure, and documentation for development of new curricula and
% adaptation of existing curricula to community standards. You will reduce the
% technological barriers to creating and sharing lesson materials by developing
% systems that are accessible to community members from a wide range of
% technical experience levels. The Lesson Infrastructure Technology Developer
% will design systems to enhance discoverability of curricular resources and
% develop clear rating and review systems to help the community assess
% curriculum quality and suitability for their teaching contexts. You will
% develop technical infrastructure to support lesson evaluation, deployment,
% publication, and tracking, and will engage with diverse community members to
% test and iterate on these systems. Depending on your background, skill set
% and interests, you may also contribute to the support of our technical
% infrastructure (e.g., database and systems administration).

% This role is a good fit for someone interested in transitioning from an
% academic environment to a focus on technical work, and who is interested in
% gaining experience in the non-profit space.

% Required:

    % Familiarity with the open-science, open-source community
    % Familiarity with Git and GitHub
    % Familiarity with one or more of the following: Bash, Jekyll, R, Python
    % Familiarity with HTML5, CSS
    % Experience working with the Linux operating system
    % Experience communicating technical subjects to non-technical audiences
    % Demonstrated leadership experience in professional or volunteer settings
    % Excellent written and real-time communication skills
    % Customer-service orientation
    % Ability to work with minimal supervision
    % Excellent time management skills
    % Able to work remotely with a distributed team

% Preferred:

    % Experience with The Carpentries
    % Experience working with a volunteer community
    % Experience contributing to open-source projects
    % Project management skills
    % Experience with continuous integration (e.g. Travis CI)
    % Experience working with metadata schema (e.g. Bioschema)
    % Experience or knowledge of human-centered design and accessibility
    % Experience in a system administrator role
    % Experience with Docker
    % Experience working with web APIs (including GitHub’s)
    % Experience with React or other UI frameworks
    % Experience with RMarkdown or other authoring tools
    % Experience with Amazon Web Services or other cloud providers

\vspace{1ex}
To the recruitment team,

\vspace{1ex}


I saw the position for
\href{https://static.carpentries.org/lesson-infrastructure-technology-developer/}{Lesson
Infastructure Technology Developer} via the ROpenSci slack channel and I
realized that this position is exactly what I've been looking for. I am a
research software engineer at Imperial College London (Nationality: USAmerican)
developing R packages, automated workflows, and short-course trainings for
field epidemiologists in outbreak scenarios. 
My background in science communication, \textbf{8 years experience developing R
packages and tutorials} for diverse audiences, discipline in collaborative
software engineering practices (test driven development, GitHub flow, and
continuous integration), and eagerness to learn and apply new skillsets makes
me an ideal candidate for this position. As an academic looking to transition
into technical work in the non-profit sector, I'm excited for this opportunity
to apply my skills in open-source software developement, evidence-based
teaching, and communication to the challenge of strenghthening the Carpentries'
infastructure. 

\vspace{1ex}

    % Familiarity with the open-science, open-source community
    % Familiarity with Git and GitHub
    % Familiarity with one or more of the following: Bash, Jekyll, R, Python
    % Familiarity with HTML5, CSS

    % Experience working with the Linux operating system
    % Experience communicating technical subjects to non-technical audiences
    % Demonstrated leadership experience in professional or volunteer settings
    % Excellent written and real-time communication skills
    % Customer-service orientation

    % Ability to work with minimal supervision
    % Excellent time management skills
    % Able to work remotely with a distributed team



My skill set lies in the intersection between software development,
reproducible research, open science, and communication.  I have been
collaboratively developing \textbf{open source} software on GitHub
\textbf{since 2013}. My most successful software project is the R package
\textit{poppr}, which has been \textbf{featured in \textgreater500
peer-reviewed scientific publications}. I strongly believe this project was
successful because \textbf{I took a `customer-service' approach}.  With
human-centered design, clear documentation, tutorials, workshops, and diligent
forum moderation, this approach \textbf{significantly reduced the barrier for
population genetic analysis in R}.

\vspace{1ex}

My experience in lesson and workshop management gives me much of the experience
needed to be effective for this role.  Over the past 5 years, I have
\textbf{collaboratively designed 4 introductory workshops} in R programming and
reproducible research to \textbf{non-technical audiences} with backgrounds in
plant pathology, population genetics, and epidemiology. In 2017, I worked with
Dr. Sydney Everhart to create an intro to R lesson as a template that was
\textbf{designed to be easily modifiable} and has been shared with members in
the plant pathology community\footnote{Initial workshop, 2017:
\url{https://twitter.com/number_three/status/867468977651539968}}\footnote{Presented
by Dr. Katlin Gold, 2018:
\url{https://twitter.com/kaitlinmgold/status/1055833741418000386}}\footnote{Presented
by Dr. Lucky Mehra, 2019:
\url{https://twitter.com/lmehra/status/1191373603771691008}}.

\vspace{1ex}

For the last year, as one of the technical leads on the
\href{https://r4epis.netlify.com}{R4EPIs} project---a collaboration between the
R Epidemics Consortium (RECON) and M\'{e}decins Sans Fronti\`{e}res (MSF)---\textbf{I
have first-hand experience in working closely with developers and users} (many
of whom were not proficient in R) to design templates and eductational materials
to get field epidemiologists start in using R. We used continuous integration,
and code review \textbf{on GitHub} to ensure the quality of contributed code.



\vspace{2ex}

Sincerely,

\vspace{5ex}

\textbf{Zhian N. Kamvar, Ph. D.}\\
\textit{Attached: Resum\'{e}, references}





% My qualifications: 

%  - Shipping R packages with C backends (poppr)
%  - Worked with data science tools such as docker containers in cloud-based infastructure (Kamvar et al. 2017)
%  - I am an autodidact and will learn new approaches/technologies to acheive my goals.
%  - Since 2013 I have worked openly and collaboratively on open source software and data analyses. 
%  - I have worked in two distributed computing frameworks (SGE and SLURM) and am knowledgable of how to properly use shared resources.

% Why you want to hire me:

%  - I have six years of experience maintaining open source software, both new and
%    legacy code in both R and C.
%  - I know how to work with data scientists to write user stories that allow me
%    to develop packages that are both user-friendly, robust, and reliable.
%  - I have experience working remotely with a diverse team using GitHub flow and 
%    continuous integration.
%  - I know how to learn new skill sets to solve unique challenges (I learned C
%    to address performance issues in R and I learned Make to create a genomic
%    data analysis pipeline).
%  - I have several years experience of collaborative working.
%  - I have strong technical writing skills.





\clearpage
