
%----------------------------------------------------------------------------------------
%	COVER LETTER
%----------------------------------------------------------------------------------------

% To remove the cover letter, comment out this entire block

\clearpage
\begin{flushright}
  Dr. Zhian N. Kamvar\\
  Dept. of Infectious Disease Epidemiology\\
  Imperial College London, W2 1NY, UK\\
  \today
\end{flushright}

\textbf{10X Genomics}\\
6230 Stoneridge Mall Road\\
Pleasanton, CA 94588-3260

% Required:

    % Familiarity with the open-science, open-source community
    % Familiarity with Git and GitHub
    % Familiarity with one or more of the following: Bash, Jekyll, R, Python
    % Familiarity with HTML5, CSS
    % Experience working with the Linux operating system
    % Experience communicating technical subjects to non-technical audiences
    % Demonstrated leadership experience in professional or volunteer settings
    % Excellent written and real-time communication skills
    % Customer-service orientation
    % Ability to work with minimal supervision
    % Excellent time management skills
    % Able to work remotely with a distributed team

% Preferred:

    % Experience with The Carpentries
    % Experience working with a volunteer community
    % Experience contributing to open-source projects
    % Project management skills
    % Experience with continuous integration (e.g. Travis CI)
    % Experience working with metadata schema (e.g. Bioschema)
    % Experience or knowledge of human-centered design and accessibility
    % Experience in a system administrator role
    % Experience with Docker
    % Experience working with web APIs (including GitHub’s)
    % Experience with React or other UI frameworks
    % Experience with RMarkdown or other authoring tools
    % Experience with Amazon Web Services or other cloud providers

\vspace{1ex}
To the recruitment team,

\vspace{1ex}


I saw the position for
\href{https://static.carpentries.org/lesson-infrastructure-technology-developer/}{Lesson
Infastructure Technology Developer} via the ROpenSci slack channel and I
realized that this position is exactly what I've been looking for. I am a
research software engineer at Imperial College London (I pay taxes in the USA)
developing R packages, automated workflows, and short-course trainings for
field epidemiologists in outbreak scenarios. 
My background in science
communication, \textbf{8 years experience developing R packages and tutorials}
for diverse audiences, discipline in collaborative software engineering
practices (test driven development, GitHub flow, and continuous integration),
and eagerness to learn and apply new skillsets makes me an ideal candidate for
this position.  As an academic looking to transition into technical work in the
non-profit sector, I'm excited for this opportunity to apply my skills in
open-source software developement, evidence-based teaching, and communication
to the challenge of strenghthening the Carpentries' infastructure. 

\vspace{1ex}

    % Familiarity with the open-science, open-source community
    % Familiarity with Git and GitHub
    % Familiarity with one or more of the following: Bash, Jekyll, R, Python
    % Familiarity with HTML5, CSS

    % Experience working with the Linux operating system
    % Experience communicating technical subjects to non-technical audiences
    % Demonstrated leadership experience in professional or volunteer settings
    % Excellent written and real-time communication skills
    % Customer-service orientation

    % Ability to work with minimal supervision
    % Excellent time management skills
    % Able to work remotely with a distributed team

I have all of the desired qualifications for this position.


I have been collaboratively developing \textbf{open source} software on GitHub \textbf{since 2013}.


My most recent first-author analytical work,
\href{https://peerj.com/articles/4152/}{Kamvar \textit{et al.} 2017}, is fully
reproducible in a Docker container hosted on
DockerHub.

\textbf{I have a solid foundation in automated
pipelines} and all of my published papers have the code and data needed to
reproduce the results.  

\vspace{1ex}

For the last year, as one of the technical leads on the
\href{https://r4epis.netlify.com}{R4EPIs} project---a collaboration between the
R Epidemics Consortium (RECON) and M\'{e}decins Sans Fronti\`{e}res (MSF)---\textbf{I
have first-hand experience in working closely with developers and users} (many
of whom were not proficient in R) to define verification tests against
pathologies unique to epidemiological data. We used 
continuous integration, and code review \textbf{on GitHub} to ensure the quality of
contributed code.



\vspace{2ex}

Sincerely,

\vspace{5ex}

\textbf{Zhian N. Kamvar, Ph. D.}\\
\textit{Attached: Resum\'{e}, references}





% My qualifications: 

%  - Shipping R packages with C backends (poppr)
%  - Worked with data science tools such as docker containers in cloud-based infastructure (Kamvar et al. 2017)
%  - I am an autodidact and will learn new approaches/technologies to acheive my goals.
%  - Since 2013 I have worked openly and collaboratively on open source software and data analyses. 
%  - I have worked in two distributed computing frameworks (SGE and SLURM) and am knowledgable of how to properly use shared resources.

% Why you want to hire me:

%  - I have six years of experience maintaining open source software, both new and
%    legacy code in both R and C.
%  - I know how to work with data scientists to write user stories that allow me
%    to develop packages that are both user-friendly, robust, and reliable.
%  - I have experience working remotely with a diverse team using GitHub flow and 
%    continuous integration.
%  - I know how to learn new skill sets to solve unique challenges (I learned C
%    to address performance issues in R and I learned Make to create a genomic
%    data analysis pipeline).
%  - I have several years experience of collaborative working.
%  - I have strong technical writing skills.





\clearpage
