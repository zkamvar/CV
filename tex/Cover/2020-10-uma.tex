
%----------------------------------------------------------------------------------------
%	COVER LETTER
%----------------------------------------------------------------------------------------

% To remove the cover letter, comment out this entire block

\clearpage
\begin{flushright}
  Dr. Zhian N. Kamvar\\
  The Carpentries\\
  (c/o Community Initiatives)\\
  1000 Broadway, Suite \#480\\
  Oakland, CA 94607, USA
  \today
\end{flushright}

\textbf{To whom it may concern}

\vspace{2ex}

% I found the _____ position via
% _____ and 
I am confident that I will excel in
this role because in my \textbf{10 years in interdisciplinary open source
scientific software development}, I have always focused on usability,
reproducibility, and flexibility in my design.
I am a software engineer at The Carpentries where I just finished developing
infrastructure that supports researchers and educators in developing lessons
for data science training. 
My background in \textbf{science communication}, discipline in collaborative
software engineering practices (test driven development, CI/CD, trunk-based
workflows, and project management), and \textbf{eagerness to learn and apply
new skillsets} makes me an ideal candidate for this position. 
I am particularly excited to work in a role that will support public health
researchers rapidly share and validate models that can help inform policies
and actions that will help circumvent future pandemics.

\vspace{2ex}

My skill set lies in the intersection of software development,
reproducible research, open science, and communication. 
\textbf{I have been collaboratively developing open source software on GitHub
since 2013}.
My most recent project is The Carpentries Workbench\footnote{Workbench user manual: \url{https://carpentries.github.io/workbench}}, 
a suite of R packages designed to build, deploy, and audit \textbf{reproducible
data science lessons} built with R Markdown in a \textbf{platform indpendent} manner\footnote{Workbench developer's guide: \url{https://carpentries.github.io/workbench-dev/intro.html\#sec-remote}}.
This was a ground-up redesign of the lesson infrastructure to \textbf{focus on
the needs and working practices of our diverse community of volunteers},
allowing them to focus on the content of their lesson and not the tooling.

\vspace{2ex}

The work I did in academia gave me all the experience to produce 
\textbf{reproducible research}\footnote{Reproducible Research using CI + Docker (Kamvar \textit{et al.}, 2017) doi: \href{https://doi.org/10.7717/peerj.4152}{10.7717/peerj.4152}}
and user-friendly scientific software\footnote{poppr R package (Kamvar \textit{et al.}, 2014) doi: \href{https://doi.org/10.7717/peerj.281}{10.7717/peerj.281}}.
My most successful software project is the R package
\textit{poppr}, which has been \textbf{featured in \textgreater1500
peer-reviewed scientific publications}. I strongly believe this project
continues to be successful because I took a community-centered approach in its
maintenance. With human-centered design, clear documentation, tutorials,
workshops, and diligent forum moderation, I worked to \textbf{significantly
reduce the barrier for reproducible population genetic analysis in R}.

\vspace{2ex}

These alone meet many of the qualifications for this position, but I believe my
work at The Carpentries provides a set of critical skills that will set me
apart from other candidates. At The Carpentries, I was able to hone my
\textbf{skills in communication and DevOps} while developing valuable
\textbf{project management} techniques that allowed me to effectively
coordinate the \textbf{simultaneous development and deployment} of 4 R
packages, a suite of GitHub actions, and the seamless transition\footnote{Automated lesson transition: \url{https://github.com/carpentries/lesson-transition\#readme}}
of > 50 active lessons maintained by > 100 volunteer maintainers, serving >
10,000 learners. anually. 

\vspace{2ex}

The experience I have gained over the last decade has given me the technical and
practical experience needed to be a successful research software engineer. I am
excited for the opportunity to work in a team context building tools that will
support researchers with data-enabled solutions to environmental challenges to
make real-world impact on biodiversity and climate resiliance. I would like to
thank the recruitment team for consideration of my application.

\vspace{3ex}

Sincerely,

\vspace{4ex}

\textbf{Zhian N. Kamvar, Ph. D.}\\
{\footnotesize \textit{(Attached: Resum\'{e}, references)}}

\clearpage

%----------------------------------------------------------------------------------------


