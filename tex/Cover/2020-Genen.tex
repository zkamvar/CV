
%----------------------------------------------------------------------------------------
%	COVER LETTER
%----------------------------------------------------------------------------------------

% To remove the cover letter, comment out this entire block

\clearpage
\begin{flushright}
  Dr. Zhian N. Kamvar\\
  1339 Poppy Way\\
  Cupertino, CA 95014\\
  \today
\end{flushright}

\textbf{Genentech Pharmaceuticals}\\
Data Science and Statistical Computing Group
South San Francisco, CA
% As as a senior bioinformatics software engineer within the Data Science and
% Statistical Computing group in the Bioinformatics and Computational Biology
% department of Genentech Research, you will work with computational
% biologists, bench scientists and other engineers to develop solutions to
% previously unsolved problems.  You will lead the development of scientific
% databases, visualizations, and/or scalable analytical methods, according to
% your interests and skills. You will provide pragmatic engineering leadership
% in collaboration with scientists and other engineers solving open
% bioinformatics problems, such as single cell expression analysis, structural
% variant calling, and HLA genotyping. Your infrastructure and tools will
% enable our scientists, across a range of computational expertise, to discover
% drug targets and biomarkers from high-throughput data and ultimately deliver
% life saving therapies to our patients

% You should be passionate about sustainable software engineering practices,
% the transformative potential of health-related data, and our mission to
% improve the lives of patients. You should have a flexible and learning
% mindset, be able to work in a fluid and dynamic environment, be comfortable
% leading globally distributed development teams and have a strong desire to
% pursue creative solutions to challenging problems.

% Responsibilities


%     Work with business analysts and computational scientists to understand and conceptualize the complex, emerging needs of our scientists, whether they are working at the keyboard or the bench.
%     Collaboratively and pragmatically solve scientific software engineering challenges encountered at the forefront of genomics, particularly those related to the scalable storage, collaborative analysis and interactive visualization of large, multi-omics datasets.

%     Plan and prioritize technically and socially complex scientific software projects in conjunction with collaborators, steering committees and other stakeholders.

%     Lead local and off-shore engineering teams to support your software development efforts.

%     Effectively communicate strategies, ideas, goals and progress to departmental, cross-functional and senior management.
%     Contribute to the broader scientific community through open-source software development.

% Requirements

%     BS or higher in statistics, bioinformatics, computer science or related field.
%     5+ years experience (including any graduate school) developing tools for data analysis. Seniority of position will depend on experience and other factors.
%     Experience supporting data science activities.
%     Adept at object-oriented programming, with proficiency in Python,C++ or Java.
%     Familiar with a popular high-level language used in data science, such as R, Python or Julia.
%     Familiar with operating on large data, such as data stored in relational and non-relational databases, array stores, HDF5 files or parquet files.
%     Demonstrated adherence to best practices in software engineering,particularly usability, testing, and appropriate use of abstraction.
%     Demonstrated ability to lead heterogeneous engineering teams and interface with domain experts and users.
%     Demonstrated ability to effectively communicate about complex bioinformatics problems to peers, users and leadership.
%     Biological domain knowledge and basic data analysis skills are desirable but not required.
%     Familiarity with formal build/release/deploy and continuous integration frameworks (e.g., Jenkins) is a plus.

% In addition,

%     You are enthusiastic about working in a scientific environment, especially one that is related to drug discovery and development.
%     You are a quick learner, are curious about new areas and the opportunity to build expertise, and courageously and creatively take initiative to see your ideas implemented.
You are attracted by the challenges of developing software that solves universal problems in bioinformatics.
%     You are able to perform at a high level in a fast changing and demanding environment.
%     You are pragmatic about the tradeoffs between features, quality, and timeliness.
\vspace{1ex}
To Michael and the recruitment team,

\vspace{1ex}

I was recommended to apply to the Bioinformatics Software Engineer III
position\footnote{https://www.gene.com/careers/detail/201908-124116/Bioinformatics-Software-Engineer-III}
by a colleage of mine, Paweł Piątkowski. He mentioned that the position aligned
well with my skill set and, after reading the job description, I must agree. I
am a research software engineer at Imperial College London developing R
packages, automated workflows, and short-course trainings for field
epidemiologists in outbreak scenarios. My background in bionformatic data
analysis, \textbf{8 years experience developing R packages} for diverse
audiences, discipline in collaborative software engineering practices (object orientation, test
driven development, GitHub flow, and continuous integration), and eagerness to
learn and apply new skillsets makes me an ideal candidate for this position. I
am excited for this opportunity to join the Genentech team to bring creative 
solutions to the challenging problems of drug discovery and other bioinformatics
challenges to improve human health.

\vspace{1ex}

I have all of the desired qualifications for this position. Not only do I have
\href{https://github.com/zkamvar/read-processing}{experience in building
bioinformatic pipelines}, but I have been developing in R and C since 2012
and currently maintain 8 packages on CRAN. All of these packages are tested
under continuous integration and one
(\href{https://grunwaldlab.github.io/poppr}{poppr}) is featured in
\textgreater500 scientific publications. \textbf{I have a solid foundation in automated
pipelines} and all of my published papers have the code and data needed to
reproduce the results.  My most recent first-author analytical work,
\href{https://peerj.com/articles/4152/}{Kamvar \textit{et al.} 2017}, is fully
reproducible in a Docker container hosted on DockerHub.

\vspace{1ex}

The idea of working as a software engineer in bioinformatics is quite appealing to me
because this sits at the intersection of two topics I've been passionate about
for the past six years: reproducible research and robust software applications.
I am keenly aware of the value of sustainable software engineering practices to support improvement of new and
legacy codebases. \href{https://github.com/grunwaldlab/poppr/issues/16}{One
concrete example} is from 2015 when a
\href{https://github.com/thibautjombart/adegenet/commit/add256cd11f37e4865e4278e6bf3ffa81bcd0f63}{major
version update} to a dependency for poppr caused much of its functionality to
fail, my test suite allowed me to safely refactor to adapt to the update. \textbf{I am passionate about interdisciplinary collaborations}. For
the last year, as one of the technical leads on the
\href{https://r4epis.netlify.com}{R4EPIs} project---a collaboration between the
R Epidemics Consortium (RECON) and M\'{e}decins Sans Fronti\`{e}res (MSF)---\textbf{I
have first-hand experience in working closely with developers and users} (many
of whom were not proficient in R) to define verification tests against
pathologies unique to epidemiological data. We used validation tests,
continuous integration, and code review on GitHub to ensure the quality of
contributed code.

\vspace{1ex}

As my carreer focus has shifted from investigative research to scientific
software engineering, this position will be immensely beneficial to my
trajectory as a research software engineer. My background in quantitative
science, robust software development, and experience as both a developer and 
user makes me well-suited to join the Genentech team as a bioinformatics software engineer. Thank you for your time and consideration, I look forward to
hearing from you. 

\vspace{2ex}

Sincerely,

\vspace{5ex}

\textbf{Zhian N. Kamvar, Ph. D.}\\
\textit{Attached: Resum\'{e}, references}





% My qualifications: 

%  - Shipping R packages with C backends (poppr)
%  - Worked with data science tools such as docker containers in cloud-based infastructure (Kamvar et al. 2017)
%  - I am an autodidact and will learn new approaches/technologies to acheive my goals.
%  - Since 2013 I have worked openly and collaboratively on open source software and data analyses. 
%  - I have worked in two distributed computing frameworks (SGE and SLURM) and am knowledgable of how to properly use shared resources.

% Why you want to hire me:

%  - I have six years of experience maintaining open source software, both new and
%    legacy code in both R and C.
%  - I know how to work with data scientists to write user stories that allow me
%    to develop packages that are both user-friendly, robust, and reliable.
%  - I have experience working remotely with a diverse team using GitHub flow and 
%    continuous integration.
%  - I know how to learn new skill sets to solve unique challenges (I learned C
%    to address performance issues in R and I learned Make to create a genomic
%    data analysis pipeline).
%  - I have several years experience of collaborative working.
%  - I have strong technical writing skills.





\clearpage
