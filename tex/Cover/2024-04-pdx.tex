
%----------------------------------------------------------------------------------------
%	COVER LETTER
%----------------------------------------------------------------------------------------

% To remove the cover letter, comment out this entire block

\clearpage
\begin{flushright}
  Zhian N. Kamvar, Ph. D.\\
  1653 NW Midlake Ln\\
  Beaverton, OR 97006\\
  \today
\end{flushright}

\textbf{To whom it may concern}

\vspace{2ex}

I found the DevOps Lead position through a post 
Mastodon\footnote{https://hachyderm.io/@skinnylatte/112288252467110280} 
I am confident that I will excel in
this role because in my \textbf{10 years in interdisciplinary open source
scientific software development}, I have always focused on usability,
reproducibility, and flexibility in my design.
In my previous role as a software engineer at The Carpentries, I developed
infrastructure that supports researchers and educators in developing lessons
for data science training.
My background in \textbf{science communication}, discipline in collaborative
software engineering practices (test driven development, CI/CD, trunk-based
workflows, and project management), and \textbf{eagerness to learn and apply
new skillsets} makes me an ideal candidate for this position. 
I am particularly excited to work in a role that will support researchers in
understand the environmental challenges so they can make critical decisions
that will have real impacts for our world.


% details describing your education, training and/or experience, and where
% obtained, which clearly reflects your qualifications 

\vspace{2ex}

% Experience in DevOps, software development, or IT operations, with a proven
%   track record in automation, CI/CD, and operational excellence.
Since 2013, I have been developing scientific software, relying heavily on
automated testing, CI/CD pipelines, and quality controls at multiple levels.
In my previous role, I developed an automated, secure and platform-independent deployment system for our community-maintained lesson infrastructure that has been operational since 2021\footnote{Explanation of Deployment System: https://carpentries.github.io/workbench-dev/remote/intro.html}. I employed unit and integration tests on CI that would run weekly\footnote{Explanation of Testing Strategies: https://carpentries.github.io/workbench-dev/testing.html} testing pre-release versions combined with a detailed release process ensured the robustness of our infrastructure\footnote{Explanation of Release Strategy: https://carpentries.github.io/workbench-dev/releases.html}.

\vspace{2ex}

% Knowledge of DevOps tools and practices (e.g., Jenkins, Git, Docker), cloud
%   platforms (AWS, Azure, GCP), scripting (Python, Bash), and security
%   fundamentals.
DevOps tooling and practices have been absolutely essential to my work,
especially in my previous role where I: used advanced features of Git,
continuously deployed on AWS, Created modified, and maintained scripts in R,
BASH, and Python for daily data aggregation, and included failsafes to prevent
malicious code entering a community-maintained repository. 


\vspace{2ex}

% Experience leading and mentoring teams in a technological environment,
%   including utilizing strong project management skills.
From 2018 to 2020 I was the Director of Software Development at the R Epidemics Consortium where I lead development of software for use in field epidemiological situations. In my most recent role, after coordinating a major transition of public-facing infrastructure that serves tens of thousands of people each year, I mentored
three of my colleagues unfamiliar with the software development cycle or the R language in the maintenance and upkeep of the software stack I created. This included regular training sessions and one-on-one personal project sessions. 

\vspace{2ex}

% Ability to exercise excellent communication skills, both verbally and in
%   writing, including conveying complex technical ideas and processes to a
%   variety of audiences.
I have been teaching people to work with code, data, and perform
reproducible research since 2014. As a certified Carpentries Instructor
Trainer, experienced in evidence-based active learning principles, I incorporate several forms of feedback within my instruction and modify it to the audience, ensuring that understanding of the material is achieved. I have written several tutorials, manual pages, and reference guides.

\vspace{3ex}

Sincerely,

\vspace{4ex}

\textbf{Zhian N. Kamvar, Ph. D.}\\
{\footnotesize \textit{(Attached: Resum\'{e}, references)}}

\clearpage

%----------------------------------------------------------------------------------------


