
%----------------------------------------------------------------------------------------
%	COVER LETTER
%----------------------------------------------------------------------------------------

% To remove the cover letter, comment out this entire block

\clearpage
\begin{flushright}
  Zhian N. Kamvar, PhD\\
  1653 NW Midlake Ln\\
  Beaverton, OR 97006\\
  \today
\end{flushright}

\textbf{To: Jaclyn Snyder, Senior Recruiter}\\

\vspace{2ex}

% We invite candidates who are passionate about leveraging technology to
% enhance public service delivery, and who are eager to contribute to a culture
% of innovation and excellence within the City of Portland's digital landscape.

I found the DevOps Lead position through a post Mastodon\footnote{\url{https://hachyderm.io/@skinnylatte/112288252467110280}}. 
% THIS NEEDS WORK
With \textbf{more than 10 years in interdisciplinary open-source scientific software development}; my background in \textbf{science communication}, discipline in collaborative DevOps tools and practices, and \textbf{eagerness to learn and apply new skillsets} makes me an ideal candidate for this position. 
I was previously a software engineer for The Carpentries (a non-profit in education with a large and distributed volunteer community) where I \textbf{developed an automated, secure, and platform-independent deployment system} for their community-maintained lesson infrastructure that has been in \textbf{continuous operation since 2021}\footnote{Deployment System: \url{https://carpentries.github.io/workbench-dev/remote/intro.html}}.
I am particularly excited for the chance to work in a role that will improve the digital experience for Portlanders.
% My goal in software development is to ensure that my applications are \textbf{robust} and \textbf{accessible} to users regardless of computational skills. 
% Above all, I aim to ensure my software is usable, efficient, consistent, and correct through best practices in software development and reproducibility.


% details describing your education, training and/or experience, and where
% obtained, which clearly reflects your qualifications 

\vspace{2ex}

% \textbf{Experience in DevOps, software development, or IT operations, with a proven track record in automation, CI/CD, and operational excellence.}\\
% As an \textbf{open source software enginer}, I have a \textbf{proven track record in automation and operational excellence}.
My skill set lies in the intersection of software development, reproducible research, open science, and communication. 
\textbf{I have been collaboratively developing open source software on GitHub since 2013}.
My most recent project was The Carpentries Workbench\footnote{Workbench user manual: \url{https://carpentries.github.io/workbench}}, 
a suite of R packages and CI workflows designed to build, deploy, and audit \textbf{reproducible
data science lessons} in a \textbf{secure\footnote{Pull Request Security: \url{https://carpentries.github.io/sandpaper-docs/pull-request.html\#risk-management}} and platform indpendent} manner.
This was a ground-up redesign of the lesson infrastructure to \textbf{focus on
the needs and working practices of our diverse community of volunteers},
allowing them to focus on the content of their lesson and not the tooling.
% % I employed unit and integration tests on CI that would run weekly\footnote{Testing Strategies: \url{https://carpentries.github.io/workbench-dev/testing.html}} testing pre-release versions combined with a detailed release process ensured the robustness and security of our infrastructure\footnote{Release Strategy: \url{https://carpentries.github.io/workbench-dev/releases.html}}.

\vspace{2ex}

% \textbf{Knowledge of DevOps tools and practices (e.g., Jenkins, Git, Docker), cloud platforms (AWS, Azure, GCP), scripting (Python, Bash), and security fundamentals.}\\
The work I did in academia (2012--2020) taught me many of the DevOps practices I have now, from \textbf{containerization}\footnote{Reproducible Research using CI + Docker (Kamvar \textit{et al.}, 2017) doi: \href{https://doi.org/10.7717/peerj.4152}{10.7717/peerj.4152}}, to \textbf{Python and BASH scripting}, 
and user-friendly scientific software\footnote{poppr R package (Kamvar \textit{et al.}, 2014) doi: \href{https://doi.org/10.7717/peerj.281}{10.7717/peerj.281}}.
My most successful software project is the R package
\textit{poppr}, which has been \textbf{featured in \textgreater2000
peer-reviewed publications}. I strongly believe this project
continues to be successful because I took a community-centered approach in its
maintenance. With \textbf{human-centered design, clear documentation, tutorials,
workshops, and diligent forum moderation}, I worked to significantly
reduce barriers for researchers and improve user experience.


% Above all, in all of my projects, the \textbf{quality of the user experience has driven my designs}.



\vspace{2ex}

% \textbf{Experience leading and mentoring teams in a technological environment, including utilizing strong project management skills.}\\

My leadership work in the non-profit space in the R4Epis project (2018--2019) and The Carpentries (2020--2023) provides a set of critical skills in communcation, DevOps, and mentorship.
At The Carpentries, I was able to hone my \textbf{skills in communication and DevOps} while developing valuable \textbf{project management} techniques that allowed me to effectively coordinate the \textbf{simultaneous development and deployment} of 4 R packages, a suite of GitHub Actions, and the seamless transition of all community-facing lessons. 

\vspace{2ex}

% \textbf{Ability to exercise excellent communication skills, both verbally and in writing, including conveying complex technical ideas and processes to a variety of audiences.}\\
Lastly, my skills in scientific communication will set me apart from other candidates. 
I am the co-founder of an award-winning science communication podcast and
\textbf{I have been teaching people to work with data and code since 2014} through workshops, tutorials, manuals, and guides using \textbf{evidence-based active learning principles.} 
This includes training \textbf{three of my former colleagues in DevOps, automation, accessibility, and maintenance} when we found that my funding was to end.

\vspace{2ex}
The experience I have gained in over a decade of work has given me the technical and practical experience needed to be a successful DevOps Leader. 
I am excited for the opportunity to work with a team that is passionate about making a difference through technology.
Having developed free and open tools throughout my carreer, I am particularly excited for a chance to work in public service technology. 
I would like to thank the recruitment team for consideration of my application.

\vspace{3ex}

Sincerely,

\vspace{4ex}

\textbf{Zhian N. Kamvar, PhD}\\
{\footnotesize \textit{(Attached: Resum\'{e}, references)}}

\clearpage

%----------------------------------------------------------------------------------------


