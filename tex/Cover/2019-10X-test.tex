
%----------------------------------------------------------------------------------------
%	COVER LETTER
%----------------------------------------------------------------------------------------

% To remove the cover letter, comment out this entire block

\clearpage
\begin{flushright}
  Dr. Zhian N. Kamvar\\
  Department of Infectious Disease Epidemiology\\
  Imperial College London, W2 1NY, UK\\
  \today
\end{flushright}

\textbf{10X Genomics}\\
6230 Stoneridge Mall Road\\
Pleasanton, CA 94588-3260

% Responsibilities

% -   Plan design, develop, and automate test plans against software releases.
% -   Work with software engineers and computational biologists to develop
%     tests, define test metrics and troubleshoot/resolve issues.
% -   Report test issues and results in a consistent and timely manner.
% -   Support troubleshooting of customer issues.
% -   Test installations over a range of operating systems.

% Desired Skills and Background

% -   BS/MS in computer science, bioinformatics, computational biology, or
%     equivalent.
% -   Good knowledge of Python, Bash, Linux, and Git.
% -   Experience in designing and developing validation, verification, and
%     regression tests.
% -   Experience with issue tracking systems (Jira).
% -   2+ years bioinformatics development/testing or equivalentis highly
%     desirable.
% -   Experience in bioinformatics applications and DNA sequencing is highly
%     desirable.
% -   Experience with continuous integration systems (Jenkins, Travis) is
%     desirable.
% -   Experience with image processing tools and algorithms is desirable.
% -   Experience in Go, Rust, JavaScript and/or R is a plus.
% -   Experience with Amazon Web Services and/or VMWare is a plus.

\vspace{1ex}
To the recruitment team,

\vspace{1ex}

I saw the position for
\href{https://boards.greenhouse.io/10xgenomics/jobs/1769868?gh_jid=1769868#application}{Software
Engineer in Test} via
\href{https://twitter.com/sjackman/status/1189977010908454912?s=20}{Shawn
Jackman on Twitter} and I realized that this position is exactly what I've been
looking for. I am a research software engineer at Imperial College London
developing R packages, report templates, and trainings for field
epidemiologists in outbreak scenarios. My background in bionformatic data
analysis, seven years experience developing R packages for diverse audiences,
discipline in collaborative software engineering practices (TDD, GitHub flow,
and CI), and eagerness to learn and apply new skillsets makes me an ideal
candidate for this position. Having worked as both as a scientist and software
engineer, I am excited for the opportunity to join the 10X genomics team
developing robust and maintainable test suites for automated bioinformatic
pipelines.

\vspace{1ex}

I have all of the desired qualifications for this position. Not only do I have
\href{https://github.com/zkamvar/read-processing}{experience working with 
bioinformatic workflows}, but I have been developing in R and C since 2012
and currently maintain five packages on CRAN. All of these packages are tested
under continuous integration and one
(\href{https://grunwaldlab.github.io/poppr}{poppr}) is featured in
\textgreater500 scientific publications. I have a solid foundation in automated
workflows and all of my published papers have the code and data needed to
reproduce the results. My most recent first-author analytical work,
\href{https://peerj.com/articles/4152/}{Kamvar \textit{et al.} 2017}, is fully
reproducible in a Docker container hosted on DockerHub.

\vspace{1ex}

The idea of working as a software engineer in test is quite appealing to me
because this sits at the intersection of two topics I've been passionate about
for the past six years: reproducible research and robust software applications.
I am keenly aware of the value of tests to prevent regression in behavior. One
concrete example is from 2015 when an R package I was dependent on updated to a
new major version and caused many of my tests to fail due to a new data
structure: \url{https://github.com/grunwaldlab/poppr/issues/16}. For the last
year, as one of the technical leads on the
\href{https://r4epis.netlify.com}{R4EPIs} project---a collaboration between the
R Epidemics Consortium (RECON) and M\'{e}decins Sans Fronti\`{e}res (MSF)---I
have first-hand experience in working with both developers and users (many of
whom were not proficient in R) to define verification tests against pathologies
unique to epidemiological data. We used validation tests, continuous
integration, and code review on GitHub to ensure the quality of contributed
code.

\vspace{1ex}

As my carreer focus has shifted from investigative research to scientific
software engineering, I believe this position will be immensely beneficial to my
trajectory as a research software engineer. My background in quantitative
science, robust software development, and experience as both a developer and 
user makes me well-suited to join the 10X genomics team as a software engineer
in testing. Thank you for your time and consideration, I look forward to
hearing from you. 

\vspace{2ex}

Sincerely,

\vspace{5ex}

\textbf{Zhian N. Kamvar}\\
\textit{Attached: curriculum vit\ae{}, references}





% My qualifications: 

%  - Shipping R packages with C backends (poppr)
%  - Worked with data science tools such as docker containers in cloud-based infastructure (Kamvar et al. 2017)
%  - I am an autodidact and will learn new approaches/technologies to acheive my goals.
%  - Since 2013 I have worked openly and collaboratively on open source software and data analyses. 
%  - I have worked in two distributed computing frameworks (SGE and SLURM) and am knowledgable of how to properly use shared resources.

% Why you want to hire me:

%  - I have six years of experience maintaining open source software, both new and
%    legacy code in both R and C.
%  - I know how to work with data scientists to write user stories that allow me
%    to develop packages that are both user-friendly, robust, and reliable.
%  - I have experience working remotely with a diverse team using GitHub flow and 
%    continuous integration.
%  - I know how to learn new skill sets to solve unique challenges (I learned C
%    to address performance issues in R and I learned Make to create a genomic
%    data analysis pipeline).
%  - I have several years experience of collaborative working.
%  - I have strong technical writing skills.





\clearpage
