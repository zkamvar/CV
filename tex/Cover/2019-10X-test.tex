
%----------------------------------------------------------------------------------------
%	COVER LETTER
%----------------------------------------------------------------------------------------

% To remove the cover letter, comment out this entire block

\clearpage
\begin{flushright}
  Dr. Zhian N. Kamvar\\
  Department of Infectious Disease Epidemiology\\
  Imperial College London, W2 1NY, UK\\
  \today
\end{flushright}
\recipient{10X Genomics}{
6230 Stoneridge Mall Road\\
Pleasanton, CA 94588-3260}

\date{\today} % Letter date
\opening{To the hiring comittee,}
\closing{Sincerely,} % Closing phrase
\enclosure[Attached]{curriculum vit\ae{}, references} % List of enclosed documents


I saw the position for Software Engineer in
Test\footnote{https://boards.greenhouse.io/10xgenomics/jobs/1769868?gh_jid=1769868#application}
via Shawn Jackman on
Twitter\footnote{https://twitter.com/sjackman/status/1189977010908454912?s=20}.
I realized that this position is exactly what I've been looking for. I am a
research software engineer at Imperial College London developing R packages,
report templates, and trainings for field epidemiologists in outbreak
scenarios. My background in bionformatic data analaysis, seven years experience
developing R packages for diverse audiences, discipline in collaborative
software engineering practices (TDD, GitHub flow, and CI), and eagerness to
learn and apply new skillsets makes me an ideal candidate for this position.
Having worked as both as a scientist and software engineer, I am excited for
the opportunity to join he 10X genomics team developing robust and maintainable
test suites for automated bioinformatic pipelines.

\vspace{1ex}

% The idea of working as a software engineer in test is quite appealing to me
% because this sits at the intersection of two things I've been passionate about
% for the past six years: reproducible research and robust software applications.
% Working at 10X genomics would give me the opportunity to put my specific skill
% set in bioinformatics and software development to work.  Moreover, as I am
% looking to leave academia and settle down, the family-oriented company culture
% of 10X Genomics suits my needs quite well. 

% \vspace{1ex}
% Responsibilities

%     Plan design, develop, and automate test plans against software releases.
%     Work with software engineers and computational biologists to develop tests, define test metrics and troubleshoot/resolve issues.
%     Report test issues and results in a consistent and timely manner.
%     Support troubleshooting of customer issues.
%     Test installations over a range of operating systems.

The idea of working as a software engineer in test is quite appealing to me
because this sits at the intersection of two things I've been passionate about
for the past six years: reproducible research and robust software applications.
For the last two years, it has been my job to ensure that software used to
report statistics from epidemiological data was robust and trustworthy---which
is especially important given the fact that it is used in humanitarian
settings. As one of the technical leads on the
\href{https://r4epis.netlify.com}{R4EPIs} project---a collaboration between the
R Epidemics Consortium (RECON) and M\'{e}decins Sans Fronti\`{e}res (MSF)---I
have first-hand experience in working with both developers and users (many of
whom were not proficient in R) to define verification tests against pathologies
unique to epidemiological data. We used continuous integration on Travis and
code review on GitHub to ensure the quality of contributed code.

\vspace{1ex}

To give a more concrete picutre of my skillset, I have been developing in R and
C since 2012 (version control since 2013, tests and continous integration since 
2014) and currently maintain five packages on CRAN. All of these packages are
tested under continuous integration and one
(\href{https://grunwaldlab.github.io/poppr}{\textit{poppr}}) is featured in
\textgreater500 scientific publications. All of my published papers have the
code and data needed to reproduce the results. My most recent first-author 
analytical work, \href{https://peerj.com/articles/4152/}{Kamvar \textit{et al.}
2017}, is fully reproducible in a Docker container under continuous integration. 

\vspace{1ex}


Perhaps my biggest accomplishment has been the
\href{https://grunwaldlab.github.io/poppr}{\{poppr\} R package} for population
genetic analysis. It has \textgreater500 citations and \textgreater85,000
downloads primarily because it was centered around the users. In addition to the
vast amount of documentation, I have maintained an active user 
forum\footnote{https://groups.google.com/group/poppr} where I field questions,
feature requests, and bug reports. This constant cycle of feedback helped me
shape the package and its tests to improve the user experience and confidence.

\vspace{1ex}


Collaboaration is an important part of my research in that I have never worked
on a single project without contributing to someone else's. In the case of
\{poppr\}, I contributed heavily to the \{adegenet\} package during
the 2015 NESCENT Population Genetics in R hackathon. The author of \{adegenet\}
wanted to refactor a critical data structure in the package and I encouraged the
author to use unit tests before refactoring. After the hackathon, I had 
contributed more code and tests to the package to ensure its stability. 


\vspace{1ex}




% Desired Skills and Background

%     BS/MS in computer science, bioinformatics, computational biology, or equivalent.
%     Good knowledge of Python, Bash, Linux, and Git.
%     Experience in designing and developing validation, verification, and regression tests.
%     Experience with issue tracking systems (Jira).
%     2+ years bioinformatics development/testing or equivalentis highly desirable.
%     Experience in bioinformatics applications and DNA sequencing is highly desirable.
%     Experience with continuous integration systems (Jenkins, Travis) is desirable.
%     Experience with image processing tools and algorithms is desirable.
%     Experience in Go, Rust, JavaScript and/or R is a plus.
%     Experience with Amazon Web Services and/or VMWare is a plus.



My qualifications: 

 - Shipping R packages with C backends (poppr)
 - Worked with data science tools such as docker containers in cloud-based infastructure (Kamvar et al. 2017)
 - I am an autodidact and will learn new approaches/technologies to acheive my goals.
 - Since 2013 I have worked openly and collaboratively on open source software and data analyses. 
 - I have worked in two distributed computing frameworks (SGE and SLURM) and am knowledgable of how to properly use shared resources.

    % Experience shipping professional software including CRAN packages.
    % Experience in Scala or C/C++.
    % Experience in data science, machine learning or distributed computing.
    % Ability to work autonomously and independently on difficult problems.


% Desired qualifications

    % Committer in the Spark project.
    % Experience releasing and maintaining CRAN packages.
    % Experience in R, Python or Julia.
    % Experience working in open source projects.

Why you want to hire me:

 - I have six years of experience maintaining open source software, both new and
   legacy code in both R and C.
 - I know how to work with data scientists to write user stories that allow me
   to develop packages that are both user-friendly, robust, and reliable.
 - I have experience working remotely with a diverse team using GitHub flow and 
   continuous integration.
 - I know how to learn new skill sets to solve unique challenges (I learned C
   to address performance issues in R and I learned Make to create a genomic
   data analysis pipeline).
 - I have several years experience of collaborative working.
 - I have strong technical writing skills.


\vspace{1ex}

As my carreer focus has shifted from investigative research to scientific
software engineering, I believe this project will be immensely beneficial to my
trajectory as a research software engineer. My background in quantitative
science, robust software development, collaborative capacity, and passion for
both equity and fairness in peer review makes me well-suited to join the team
as a software research scientist. Thank you for your time and consideration, I
look forward to hearing from you. 

\vspace{2ex}

Sincerely,

\vspace{5ex}

\textbf{Zhian N. Kamvar}\\
\textit{Attached: curriculum vit\ae{}, references}
\clearpage

%----------------------------------------------------------------------------------------


% EcoHealth Alliance seeks a creative, dedicated, and collaborative software
% research scientist to support a two-year project in launching a new software
% peer-review initiative. The software research scientist will work on the
% Sloan Foundation supported rOpenSci project, with rOpenSci staff and a
% statistical methods editorial board. They will research and develop standards
% and review guidelines for statistical software, publish findings, and develop
% R software to test packages against those standards. The software research
% scientist will work with staff and the board to collaborate broadly with the
% statistical and software communities to gather input, refine and promote the
% standards, and recruit editors and peer reviewers. The candidate must be
% self-motivated, proactive, collaborative and comfortable working openly and
% reproducibly with a broad online community.

% DESCRIPTION AND RESPONSIBILITIES

% - Research standards and protocols for evaluating statistical software
%     quality and correctness, and the extent of their adoption
% - Create new testing frameworks for R packages
% - Assist rOpenSci staff and project board members in drafting new
%     peer-review evaluation standards, guidelines, and documentation
% - Write technical and scientific papers, documentation, and blog posts
% - Assist in organizing peer-review system for scientific software and
%     managing the review board
% - Participate in and represent the rOpenSci project in person and via
%     on-line fora
% - Participate in other projects and tasks as required or assigned by
%     supervisor

% MINIMUM QUALIFICATIONS

% - A Master's degree in statistics, computer science, or a related field with
%     a focus on quantitative methodologies, or equivalent experience in
%     statistical methods evaluation and development
% - Expertise in open-source R package development, including collaborative
%     development using Git and GitHub, testing frameworks, and continuous
%     integration
% - Strong writing skills
% - Experience in collaborative team projects and consensus building
% - A passion for improving scientific reproducibility

% DESIRED QUALIFICATIONS

% - A PhD in statistics, computer science, or a related field with a focus on
%     quantitative methodologies, or equivalent experience in statistical
%     methods evaluation and development
% - Published scientific or technical articles or software documentation



