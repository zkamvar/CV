
%----------------------------------------------------------------------------------------
%	COVER LETTER
%----------------------------------------------------------------------------------------

% To remove the cover letter, comment out this entire block

\clearpage
\begin{flushright}
  Dr. Zhian N. Kamvar\\
  Department of Infectious Disease Epidemiology\\
  Imperial College London, W2 1NY, UK\\
  \today
\end{flushright}
% \recipient{HR Departmnet}{EcoHealth Alliance\\460 West 34th Street - 17th Floor\\New York, NY 10001-2320} % Letter recipient
% \date{\today} % Letter date
% \opening{Dear Dr. Ross, members of the search committee} % Opening greeting
\closing{Sincerely,} % Closing phrase
\enclosure[Attached]{curriculum vit\ae{}, references} % List of enclosed documents

% \makelettertitle % Print letter title
\errmessage{! PUT A FORMAL SALUTATION HERE }
\end

\vspace{1ex}

I am applying for the position of Software Research Scientist. My
training as a computational biologist, passion for robust test-driven software
developement, and practical experience in cross-discipline collaboration and
communication makes me the ideal candidate for this position. As a developer
of software for analysis of both population genetic and epidemiological data, I
am keenly aware of how the lack of standards impacts scientists\footnote{one
recent example:
\url{https://twitter.com/reneecatullo/status/1133900861397016578}}. I have both
the understanding to know how a well-tested package can still give incorrect
results and the technical expertise to collaborate with a team to deliver
robust and useable software that produces correct results. Having worked as
both a research scientist and research software engineer, I am excited for the
opportunity to collaborate with the rOpenSci team. 

\vspace{1ex}

I have all of the preferred skills and qualifications necessary for this
position including a degree in a quantitative field (population genetics),
expertise in open-source R package development\footnote{examples: population
genetics---\href{https://grunwaldlab.github.io/poppr}{poppr},
epidemiology---\href{https://www.repidemicsconsortium.org/aweek}{aweek}, \href{https://www.repidemicsconsortium.org/incidence}{incidence}, and
\href{https://github.com/R4EPI/sitrep}{sitrep}}, and strong technical writing
skills\footnote{scientific: \href{https://peerj.com/articles/4152/}{Kamvar
\textit{et al.} 2017}, technical:
\href{https://grunwaldlab.github.io/poppr/reference/psex.html}{Documentation
for \texttt{poppr::psex()}}}.
As the co-creator of an
\href{https://blogs.oregonstate.edu/inspiration}{award-winning sci-comm radio
program}, I have unique experience in interdisciplinary communication.  
Moreover I have experience in leading collaborative projects\footnote{RECON + MSF:
\href{https://R4EPIs.netlify.com}{The R4EPIs project}} which employ respectful
peer review\footnote{\href{https://github.com/R4EPI/sitrep/pull/76}{Example
code review for the sitrep package}} with collaborators of varying expertise.
Most importantly, I have a passion for robustness in statistical
software\footnote{\href{https://zkamvar.netlify.com/post/2017-09-23-squish/squish}{Blog
post about correcting a statistical error in my own package}}.

\vspace{1ex}


To give a more concrete picutre of my skillset, I have been developing in R and
C since 2012 (version control since 2013, tests and continous integration since 
2014) and currently maintain five packages on CRAN. All of these packages are
tested under continuous integration and one
(\href{https://grunwaldlab.github.io/poppr}{\textit{poppr}}) is featured in
\textgreater500 scientific publications. All of my published papers have the
code and data needed to reproduce the results. My most recent first-author 
analytical work, \href{https://peerj.com/articles/4152/}{Kamvar \textit{et al.}
2017}, is fully reproducible in a Docker container under continuous integration. 

\vspace{1ex}

Collaboaration is an important part of my research in that I have never worked
on a single project without contributing to someone else's. In the case of
\textit{poppr}, I contributed heavily to the \textit{adegenet} package during
and after a 2015 NESCENT Population Genetics in R hackathon. In the case of
Kamvar \textit{et al.} 2017, I was able to track down and fix a
\href{https://github.com/slowkow/ggrepel/issues/72}{hidden bug} in the
\textit{ggrepel} package. Most recently, I am one of the technical leads on a
collaboration between the R Epidemics Consortium (RECON) and M\'{e}decins Sans
Fronti\`{e}res (MSF) called R4EPIs where we work on standardizing analyses and
training for field epidemiologists.  

\vspace{1ex}

As my carreer focus has shifted from investigative research to scientific
software engineering, I believe this project will be immensely beneficial to my
trajectory as a research software engineer. My background in quantitative
science, robust software development, collaborative capacity, and passion for
both equity and fairness in peer review makes me well-suited to join the team
as a software research scientist. Thank you for your time and consideration, I
look forward to hearing from you. 

\vspace{2ex}

Sincerely,

\vspace{5ex}

\textbf{Zhian N. Kamvar}\\
\textit{Attached: curriculum vit\ae{}, references}
\clearpage

%----------------------------------------------------------------------------------------


% EcoHealth Alliance seeks a creative, dedicated, and collaborative software
% research scientist to support a two-year project in launching a new software
% peer-review initiative. The software research scientist will work on the
% Sloan Foundation supported rOpenSci project, with rOpenSci staff and a
% statistical methods editorial board. They will research and develop standards
% and review guidelines for statistical software, publish findings, and develop
% R software to test packages against those standards. The software research
% scientist will work with staff and the board to collaborate broadly with the
% statistical and software communities to gather input, refine and promote the
% standards, and recruit editors and peer reviewers. The candidate must be
% self-motivated, proactive, collaborative and comfortable working openly and
% reproducibly with a broad online community.

% DESCRIPTION AND RESPONSIBILITIES

% - Research standards and protocols for evaluating statistical software
%     quality and correctness, and the extent of their adoption
% - Create new testing frameworks for R packages
% - Assist rOpenSci staff and project board members in drafting new
%     peer-review evaluation standards, guidelines, and documentation
% - Write technical and scientific papers, documentation, and blog posts
% - Assist in organizing peer-review system for scientific software and
%     managing the review board
% - Participate in and represent the rOpenSci project in person and via
%     on-line fora
% - Participate in other projects and tasks as required or assigned by
%     supervisor

% MINIMUM QUALIFICATIONS

% - A Master's degree in statistics, computer science, or a related field with
%     a focus on quantitative methodologies, or equivalent experience in
%     statistical methods evaluation and development
% - Expertise in open-source R package development, including collaborative
%     development using Git and GitHub, testing frameworks, and continuous
%     integration
% - Strong writing skills
% - Experience in collaborative team projects and consensus building
% - A passion for improving scientific reproducibility

% DESIRED QUALIFICATIONS

% - A PhD in statistics, computer science, or a related field with a focus on
%     quantitative methodologies, or equivalent experience in statistical
%     methods evaluation and development
% - Published scientific or technical articles or software documentation



