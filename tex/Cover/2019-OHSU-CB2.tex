
%----------------------------------------------------------------------------------------
%	COVER LETTER
%----------------------------------------------------------------------------------------

% To remove the cover letter, comment out this entire block

\clearpage
\begin{flushright}
  Dr. Zhian N. Kamvar\\
  Dept. of Infectious Disease Epidemiology\\
  Imperial College London, W2 1NY, UK\\
  \today
\end{flushright}

\textbf{Computational Biology Program}\\
3181 S.W. Sam Jackson Park Road\\
Portland, OR 97239


% THe Computational Biologist 2 will support investigators in the conduct of
% original research in service to OHSU’s mission to improve human health,
% building and leveraging platform based tools and biological understanding
% while working in a team oriented data processing, data analysis and software
% development environment, working with high-dimensional data (omic, imaging,
% phenotypic), following best practices such as literature currency, tool
% benchmarking and code sharing through GitHub. Duties include teamwork, client
% interaction, cross-core collaboration, code and data management, maintenance
% and development of tools and pipelines, and support of investigators’
% discoveries and scholarly work.
% Required Qualifications

% - Bachelor’s degree with major courses in field of research. Additional
%   experience in lieu of degree may be considered.
% - One year of relevant experience
% - Demonstrated ability in research related to genomic analysis, machine
%   learning, or image analysis, which may include experience with next-gen
%   sequencing or medical imaging
% - Demonstrated skills in a high level programming language, preferably Python
%   or R, preferably in Linux or Unix, optionally with Java or C/C++
% - Understanding of basic statistics and some of its applications to
%   biomedical science
% - Experience with structured data storage and with multi-component data
%   processing systems
% - Experience with architecture or tools for managing “omics” or imaging data
% - Ability to prioritize multiple tasks at one time
% - Excellent communication, analytical and organizational skills, both written
%   and verbal
% - Ability to work independently and as part of a team while being
%   collaborative in resolving problems
% - A desire to change the world and contribute to the elimination of human
%   disease

% Preferred Qualifications

% - Masters with major courses in field of research
% - 5 years of relevant experience, including one year of hands-on experience
%   with generation or analysis of omic or imaging data
% - Some work experience in a professional, team-oriented computational biology
%   environment
% - High performance computing experience
% - A passion for open-access innovation



\vspace{1ex}
To the recruitment team,

\vspace{1ex}

I am applying for the position of Computational Biologist. I am a research
software engineer at Imperial College London developing R packages, automated
workflows, and short-course trainings for field epidemiologists in outbreak
scenarios. My background in bionformatic data analysis, \textbf{8 years
experience developing R packages} for diverse audiences, discipline in
\textbf{team-oriented and collaborative software engineering practices} (test
driven development, GitHub flow, and continuous integration), and eagerness to
learn and apply new skillsets makes me an ideal candidate for this position.
Having worked as both as a scientist and software engineer, I am excited for
the opportunity to join the OHSU Computational Biology team to support research
that will improve human health.

\vspace{1ex}

I have all of the desired qualifications for this position. Not only do I have
experience in building bioinformatic pipelines\footnote{Automated Illumina assembly pipeline:\url{https://github.com/zkamvar/read-processing}}, but I have been developing in R and C since 2012
and currently maintain 5 packages on CRAN. All of these packages are tested
under continuous integration and one
(poppr\footnote{R package `poppr': \url{https://grunwaldlab.github.io/poppr}}) is featured in
\textgreater500 scientific publications. \textbf{I have a solid foundation in automated
pipelines} and all of my published papers have the code and data needed to
reproduce the results. My most recent first-author analytical work\footnote{\textbf{Kamvar ZN}, Amaradasa BS, Jhala R, McCoy S, Steadman JR,
  Everhart SE. (2017) Population structure and phenotypic variation of
  \textit{Sclerotinia sclerotiorum} from dry bean (\textit{Phaseolus vulgaris})
  in the United States. \textit{PeerJ} \textbf{5}:e4152 doi: \href{https://doi.org/10.7717/peerj.4152}{10.7717/peerj.4152}}
is fully reproducible in a Docker container hosted on DockerHub.

\vspace{1ex}


For the last year, as one of the technical leads on the
R4EPIs project\footnote{\url{https://r4epis.netlify.com}}---a collaboration between the
R Epidemics Consortium (RECON) and M\'{e}decins Sans Fronti\`{e}res
(MSF)---\textbf{I have first-hand experience in working closely with developers
and users} (many of whom were not proficient in R) to design code and tests
against pathologies unique to epidemiological data. We used validation tests,
continuous integration, and code review on GitHub to ensure the quality of
contributed code.

\vspace{1ex}

This position will be immensely beneficial to my growth as both a
bioinformatician and reserach software engineer by giving me the challenges of 
working on problems that can have a positive impact human health. My background
in quantitative science, robust software development, and experience as both a
developer and user makes me well-suited to join the OHSU team as a
computational biologist.  Thank you for your time and consideration, I look
forward to hearing from you. 

\vspace{2ex}

Sincerely,

\vspace{5ex}

\textbf{Zhian N. Kamvar, Ph. D.}\\
\textit{Attached: Resum\'{e}, references}





% My qualifications: 

%  - Shipping R packages with C backends (poppr)
%  - Worked with data science tools such as docker containers in cloud-based infastructure (Kamvar et al. 2017)
%  - I am an autodidact and will learn new approaches/technologies to acheive my goals.
%  - Since 2013 I have worked openly and collaboratively on open source software and data analyses. 
%  - I have worked in two distributed computing frameworks (SGE and SLURM) and am knowledgable of how to properly use shared resources.

% Why you want to hire me:

%  - I have six years of experience maintaining open source software, both new and
%    legacy code in both R and C.
%  - I know how to work with data scientists to write user stories that allow me
%    to develop packages that are both user-friendly, robust, and reliable.
%  - I have experience working remotely with a diverse team using GitHub flow and 
%    continuous integration.
%  - I know how to learn new skill sets to solve unique challenges (I learned C
%    to address performance issues in R and I learned Make to create a genomic
%    data analysis pipeline).
%  - I have several years experience of collaborative working.
%  - I have strong technical writing skills.





\clearpage
