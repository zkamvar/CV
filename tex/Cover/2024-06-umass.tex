
%----------------------------------------------------------------------------------------
%	COVER LETTER
%----------------------------------------------------------------------------------------

% To remove the cover letter, comment out this entire block

\clearpage
\begin{flushright}
  Zhian N. Kamvar, PhD\\
  1653 NW Midlake Ln\\
  Beaverton, OR 97006\\
  \today
\end{flushright}

\textbf{To: Reich Lab Hiring Mangager}\\

\vspace{2ex}

% We invite candidates who are passionate about leveraging technology to
% enhance public service delivery, and who are eager to contribute to a culture
% of innovation and excellence within the City of Portland's digital landscape.

I found the DevOps Lead position from a post on Mastodon\footnote{\url{https://hachyderm.io/@skinnylatte/112288252467110280}}. 
% THIS NEEDS WORK
With \textbf{more than 10 years in interdisciplinary open-source scientific software development} %my background in \textbf{science communication}, 
with a strong focus on \textbf{collaborative DevOps tools and practices}, %established track record of \textbf{project management and leadership}, and \textbf{eagerness to learn and apply new skillsets} makes me 
I am an ideal candidate for this position. 
I was a software engineer for The Carpentries (a non-profit in education with a large and distributed volunteer community) where I developed \textit{The Carpentries Workbench}, \textbf{an automated, secure}, and platform-independent deployment system for their community-maintained lesson \textbf{infrastructure that has been in continuous operation since 2021}\footnote{\textit{The Carpentries Workbench}: \url{https://carpentries.github.io/workbench}}.%\footnote{Deployment System: \url{https://carpentries.github.io/workbench-dev/remote/intro.html}}.
Having developed free and open tools throughout my carreer, \textbf{I am particularly excited for a chance to work in public service technology}. 
% My goal in software development is to ensure that my applications are \textbf{robust} and \textbf{accessible} to users regardless of computational skills. 
% Above all, I aim to ensure my software is usable, efficient, consistent, and correct through best practices in software development and reproducibility.


% details describing your education, training and/or experience, and where
% obtained, which clearly reflects your qualifications 

\vspace{2ex}

% \textbf{Experience in DevOps, software development, or IT operations, with a proven track record in automation, CI/CD, and operational excellence.}\\
% As an \textbf{open source software enginer}, I have a \textbf{proven track record in automation and operational excellence}.
\underline{\textbf{DevOps experience and operational excellence:}}
I have been collaboratively developing open source software on GitHub \textbf{since 2013} in the disciplines of population genetics, epidemiology, and education.
Most recently, I was the lead for \textit{The Carpentries Workbench}, 
a suite of R packages and \textbf{CI/CD workflows} designed to build, deploy, and audit reproducible data science lessons in a \textbf{secure and platform indpendent} manner.
This was a ground-up redesign of the lesson infrastructure to \textbf{focus on
the needs and working practices of our diverse community of \textgreater2000 volunteers},
allowing them to focus on the content of their lesson and not the tooling.
% % I employed unit and integration tests on CI that would run weekly\footnote{Testing Strategies: \url{https://carpentries.github.io/workbench-dev/testing.html}} testing pre-release versions combined with a detailed release process ensured the robustness and security of our infrastructure\footnote{Release Strategy: \url{https://carpentries.github.io/workbench-dev/releases.html}}.

\vspace{2ex}

% \textbf{Knowledge of DevOps tools and practices (e.g., Jenkins, Git, Docker), cloud platforms (AWS, Azure, GCP), scripting (Python, Bash), and security fundamentals.}\\
\underline{\textbf{DevOps knowledge of tools and practices:}}
I operate under a \textbf{growth mindset} and am \textbf{always learning}.
The work I did in academia (2012--2020) taught me many of the DevOps practices I have now.
As a grad worker, I developed \textbf{project management skills} creating user-friendly scientific software\footnote{poppr R package (Kamvar \textit{et al.}, 2014) doi: \href{https://doi.org/10.7717/peerj.281}{10.7717/peerj.281} featured in \textbf{\textgreater2000 peer-reviewed publications}} on \textbf{Linux} with \textbf{Git} and \textbf{CI} and 
performed complex simulation analyses using \textbf{automation with Python and BASH}.
In my postdoctoral work, I used \textbf{Docker} for automated reproducible research\footnote{Automated Research using Docker + CircleCI (Kamvar \textit{et al.}, 2017) doi: \href{https://doi.org/10.7717/peerj.4152}{10.7717/peerj.4152}}, 
and was an early adopter of \textbf{GitHub Actions} for CI/CD to manage deployments with limited resources.
My work at The Carpentries (2020--2023) gave me the \textbf{learning opportunity} to \textbf{scale my DevOps skills} by allowing me to work with \textbf{monitoring frameworks and deployments on AWS} and \textbf{securely managing personally identifiable information}. 
% My most successful software project is the R package
% \textit{poppr}, which has been \textbf{featured in \textgreater2000
% peer-reviewed publications}. I strongly believe this project
% continues to be successful because I took a community-centered approach in its
% maintenance. With \textbf{human-centered design, clear documentation, tutorials,
% workshops, and diligent forum moderation}, I worked to significantly
% reduce barriers for researchers and improve user experience.


% Above all, in all of my projects, the \textbf{quality of the user experience has driven my designs}.



\vspace{2ex}

% \textbf{Experience leading and mentoring teams in a technological environment, including utilizing strong project management skills.}\\
\underline{\textbf{Leadership and Mentoring:}}
I have \textbf{\textgreater5 years of leadership experience} through my work in the non-profit space in the R4Epis project (2018--2019) and The Carpentries (2020--2023).
At R4Epis, I \textbf{coordinated testing, development, and deployment} of softwware for field epidemiologists with limited computing resources.
My role in The Carpentries started with \textbf{vendor and tool management} for their infrastructure, including \textbf{hiring and evaluating contractors} for accessible UI design work.
I put my \textbf{strong mentorship skills} to work in 2023 when I \textbf{trained 3 novice colleages in DevOps, automation, and maintenance} of our infrastructure.

% At The Carpentries, I was able to hone my \textbf{skills in communication and DevOps} while developing valuable \textbf{project management} techniques that allowed me to effectively coordinate the \textbf{simultaneous development and deployment} of 4 R packages, a suite of GitHub Actions, and the seamless transition of all community-facing lessons. 

\vspace{2ex}

% \textbf{Ability to exercise excellent communication skills, both verbally and in writing, including conveying complex technical ideas and processes to a variety of audiences.}\\
\underline{\textbf{Communication skills:}}
I am the co-founder of an award-winning \textbf{science communication podcast} where I coached graduate workers to \textbf{explain complex technical ideas} to a general audience (2012--2016).
As a certified Carpentries Instructor Trainer, I have been \textbf{teaching people to work with data and code since 2014} using \textbf{evidence-based active learning principles.} 
And finally, much of the success for my projects lies with my ability to \textbf{write clear and concice documentation tailored to the audience}.


\vspace{2ex}
The experience I have gained in over a decade has given me the experience needed to be a successful DevOps Leader. 
I am excited for the opportunity to work with a team that is passionate about making a difference through technology
and particularly excited for the chance to work in a role that will improve the digital experience for Portlanders.
I would like to thank the recruitment team for consideration of my application.

\vspace{3ex}

Sincerely,

\vspace{4ex}

\textbf{Zhian N. Kamvar, PhD}\\
{\footnotesize \textit{(Attached: Resum\'{e}, references)}}

\clearpage

%----------------------------------------------------------------------------------------


