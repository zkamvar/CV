
%----------------------------------------------------------------------------------------
%	COVER LETTER
%----------------------------------------------------------------------------------------

% To remove the cover letter, comment out this entire block

\clearpage
\begin{flushright}
  Zhian N. Kamvar, PhD\\
  1653 NW Midlake Ln\\
  Beaverton, OR 97006\\
  \today
\end{flushright}

\textbf{To: Reich Lab Hiring Mangager}\\

\vspace{2ex}

With \textbf{more than 10 years in interdisciplinary open-source scientific software development} and a strong focus on \textbf{user support, documentation, and testing}, I am an ideal candidate for this position. 
I was a software engineer for The Carpentries (a non-profit in education with a large and distributed volunteer community) where I developed \textit{The Carpentries Workbench}, \textbf{an automated, secure}, and platform-independent deployment system for their community-maintained lesson \textbf{infrastructure that has been in continuous operation since 2021}\footnote{\textit{The Carpentries Workbench}: \url{https://carpentries.github.io/workbench}}.
I am particularly excited to work in a role that will support public health
researchers rapidly share and validate models that can help inform policies
and actions that will help circumvent future pandemics.



\vspace{2ex}

\underline{\textbf{Development experience:}}
I have been collaboratively \textbf{developing, documenting, and testing} open source software \textbf{on GitHub since 2013} in the disciplines of population genetics, epidemiology, and education.
Most recently, I was the lead for \textit{The Carpentries Workbench}, 
a suite of R packages and \textbf{CI/CD workflows} designed to build, deploy, and audit reproducible data science lessons in a \textbf{secure and platform indpendent} manner.
This was a ground-up redesign of the lesson infrastructure to \textbf{focus on
the needs and working practices of our diverse community of \textgreater2000 volunteers},
allowing them to focus on the content of their lesson and not the tooling.

\vspace{2ex}

\underline{\textbf{Organizational Skills}}
I operate under a \textbf{growth mindset} and am \textbf{always learning}.
The work I did in academia (2012--2020) taught me many of the organizational practices I have now.
As a grad worker, I developed \textbf{project management skills} creating user-friendly scientific software\footnote{poppr R package (Kamvar \textit{et al.}, 2014) doi: \href{https://doi.org/10.7717/peerj.281}{10.7717/peerj.281} featured in \textbf{\textgreater2000 peer-reviewed publications}} on \textbf{Linux} with \textbf{Git} and \textbf{CI} and 
performed complex simulation analyses using \textbf{automation with Python and BASH}.
In my postdoctoral work, I used \textbf{Docker} for automated reproducible research\footnote{Automated Research using Docker + CircleCI (Kamvar \textit{et al.}, 2017) doi: \href{https://doi.org/10.7717/peerj.4152}{10.7717/peerj.4152}}, 
and was an early adopter of \textbf{GitHub Actions} for CI/CD to manage deployments with limited resources.

\vspace{2ex}

\underline{\textbf{Leadership, Support, and Collaborations:}}
I have \textbf{\textgreater5 years of leadership experience} through my work in the non-profit space in the R4Epis project and The Carpentries.
At R4Epis (2018--2019), I \textbf{coordinated testing, development, and deployment} of software for field epidemiologists with limited computing resources.
By providing intial documentation for technical configuration via offline resources along with \textbf{on-demand remote support} via synchronous and asynchronous communications, I was able to set them up for success.
My work at The Carpentries (2020--2023) gave me the opportunity to hone my skills \textbf{supporting a user base through a technology transition}. Through webinars, presentations, and hands-on tutorials, I was able to \textbf{facilitate community growth} that left users feeling empowered.
% My role in The Carpentries started with \textbf{vendor and tool management} for their infrastructure, including \textbf{hiring and evaluating contractors} for accessible UI design work.
In 2023, as my funding was coming to an end, I \textbf{guided 3 novice colleages in code review, automation, and maintenance} of our infrastructure.


\vspace{2ex}

\underline{\textbf{Communication skills:}}
I am the co-founder of an award-winning \textbf{science communication podcast} where I coached graduate workers to \textbf{explain complex technical ideas} to a general audience (2012--2016).
As a certified Carpentries Instructor Trainer, I have been \textbf{teaching people to work with data and code since 2014} using \textbf{evidence-based active learning principles.} 
And finally, much of the success for my projects lies with my ability to \textbf{write clear and concice documentation tailored to the audience}.


\vspace{2ex}
The experience I have gained in over a decade has given me the experience needed to be a valuable asset to the lab. 
I am particularly excited for the opportunity to work in the public health sector again.
I would like to thank the recruitment team for consideration of my application.

\vspace{3ex}

Sincerely,

\vspace{4ex}

\textbf{Zhian N. Kamvar, PhD}\\
{\footnotesize \textit{(Attached: Resum\'{e}, references)}}

\clearpage

%----------------------------------------------------------------------------------------


