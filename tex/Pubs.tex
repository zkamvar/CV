\section{Peer Review}

Molecular Ecology, Methods in Ecology and Evolution

\section{Peer Reviewed Publications}

\begin{enumerate}[leftmargin = 14pt]

	\item Tabima JF, Everhart SE, Larsen MM, Weisberg AJ, Kamvar ZN, Tancos MA,
	Smart CD, Chang JH, Gr\"unwald NJ. (2016) Microbe-ID: an open source toolbox
	for microbial genotyping and species identification. PeerJ \textbf{4}:e2279
	\href{https://doi.org/10.7717/peerj.2279}{10.7717/peerj.2279}
	
	\item Jombart T, Archer F, Schliep K, \textbf{Kamvar Z}, Harris R, Paradis
	E, Goudet J, Lapp H (2016). apex: phylogenetics with multiple genes.
	Molecular Ecology Resources. doi:
	\href{http://dx.doi.org/10.1111/1755-0998.12567}{10.1111/1755-0998.12567}

	\item \textbf{Kamvar ZN}, L\'opez-Uribe MM, Coughlan S, Gr\"unwald NJ, Lapp
	H, Manel S (2016). Developing educational resources for population genetics
	in R: an open and collaborative approach. Molecular ecology resources. doi:
	\href{http://dx.doi.org/10.1111/1755-0998.12558}{10.1111/1755-0998.12558}

	\item Gr\"unwald NJ, Larsen MM, \textbf{Kamvar ZN}, Reeser PW, Kanaskie A,
	Laine J, Wiese R (2015) First report of the EU1 clonal lineage of
	\textit{Phytophthora ramorum} on tanoak in an OR forest. 
	\textit{Plant Disease}. 100:5, 1024-1024. doi:
	\href{http://dx.doi.org/10.1094/PDIS-10-15-1169-PDN}{10.1094/PDIS-10-15-1169-PDN}

	\vspace{6pt}

	\item \textbf{Kamvar ZN}, Brooks JC and Gr\"unwald NJ (2015) Novel R tools for
	analysis of genome-wide population genetic data with emphasis on clonality.
	\textit{Front. Genet.} \textbf{6}: 208. doi: 
	\href{http://dx.doi.org/10.3389/fgene.2015.00208}{10.3389/fgene.2015.00208}

	\vspace{6pt}

	\item \textbf{Kamvar ZN}, Larsen MM, Kanaskie AM, Hansen EM, and Gr\"unwald
	NJ. (2015) Spatial and temporal analysis of populations of the sudden oak
	death pathogen in Oregon forests. \textit{Phytopathology}. \textbf{105}:
	982-989. doi: 
	\href{http://dx.doi.org/10.1094/PHYTO-12-14-0350-FI}{10.1094/PHYTO-12-14-0350-FI}.
	
	\vspace{6pt}

	\item Weiland JE, Garrido PA, \textbf{Kamvar ZN}, Marek SM, Gr\"unwald NJ, and
	Garz\'on CD. (2015) Population structure of \textit{Pythium irregulare}, \textit{P.
	sylvaticum}, and \textit{P. ultimum} in forest nursery soils of Oregon and
	Washington. \textit{Phytopathology}. \textbf{105}: 684-694. doi: 
	\href{http://dx.doi.org/10.1094/PHYTO-05-14-0147-R}{10.1094/PHYTO-05-14-0147-R}

	\vspace{6pt}

    \item \textbf{Kamvar ZN}, Tabima JF, Gr\"unwald NJ. (2014) \textit{Poppr}: an
	R package for genetic analysis of populations with clonal, partially clonal,
	and/or sexual reproduction. PeerJ \textbf{2}: e281.
	doi: \href{http://dx.doi.org/10.7717/peerj.281}{10.7717/peerj.281}
	
	\vspace{6pt}

	\item Buckner B, Beck J, Browning, K, Hoxha E, Grantham L, \textbf{Kamvar
	ZN}, Lough A, Nikolova O, and Schnable PS, Scanlon MJ, and Janick-Buckner D.
	(2007) Involving undergraduates in the annotation and analysis of global
	gene expression studies: creation of a maize shoot apical meristem
	expression database. \textit{Genetics}
	\textbf{176}: 741-747. doi: 
	\href{http://dx.doi.org/10.1534/genetics.106.066472}{10.1534/genetics.106.066472}

\end{enumerate}



\section{Contributed Presentations}


\begin{enumerate}[leftmargin = 14pt]

	\item \textbf{Kamvar ZN}, Larsen MM, Kanaskie AM, Hansen EM, Gr\"unwald NJ.
	(2015) Evidence for at least two introductions of the sudden oak death
	pathogen into Oregon forests. American Phytopathological Society National
	Conference, Pasadena, CA.

	\vspace{6pt}

	\item \textbf{Kamvar ZN}, Tabima JF, Gr\"unwald NJ. (2014) Application of
	the R package poppr for analysis of population genetic data. American
	Phytopathological Society National Conference, Minneapolis, MN.

	\vspace{6pt}

	\item \textbf{Kamvar ZN} (2013) Ph.D. Proposal Seminar: Determination of
	pattern and process in the evolution of the plant pathogen
	\textit{Phytophthora syringae}. Department of Botany and Plant Pathology,
	Oregon State University, Corvallis, OR.

	\vspace{6pt}

	\item \textbf{Kamvar ZN} (2013) \textit{Poppr}: An R package for genetic
	analysis of populations with mixed (clonal/sexual) reproduction. Biology
	Graduate Student Symposium, Hatfield Marine Science Center, Newport OR.

	\vspace{6pt}

	\item \textbf{Kamvar ZN}, Tabima JF, Gr\"unwald NJ (2013) \textit{Poppr}: An
	R package for genetic analysis of populations with mixed (clonal/sexual)
	reproduction. Fungal Genetics Conference, Asilomar, CA.

	\vspace{6pt}

	\item \textbf{Kamvar ZN}, Gr\"unwald NJ (2012) \textit{Poppr}: An R package
	for popualtion genetic analysis. OSU Fall CGRB Conference, Oregon State
	University, Corvallis, OR.

	\vspace{6pt}

	\item Browning K, Fritz A, Hoxha E, and \textbf{Kamvar ZN} (2007) Annotation
	and analysis of global gene expression studies: creation of a maize shoot
	apical meristem expression database, Maize Genetics Conference, St. Charles,
	IL.

	\vspace{6pt}

	\item Browning K, Fritz A, Hoxha E, and \textbf{Kamvar ZN} (2007) Annotation
	and analysis of global gene expression studies: creation of a maize shoot
	apical meristem expression database, Truman Student Research Conference,
	Truman State University, Kirksville, MO.

\end{enumerate}

%----------------------------------------------------------------------------------------
%	COVER LETTER
%----------------------------------------------------------------------------------------

% To remove the cover letter, comment out this entire block

% \clearpage

% \recipient{HR Departmnet}{Corporation\\123 Pleasant Lane\\12345 City, State} % Letter recipient
% \date{\today} % Letter date
% \opening{Dear Sir or Madam,} % Opening greeting
% \closing{Sincerely yours,} % Closing phrase
% \enclosure[Attached]{curriculum vit\ae{}} % List of enclosed documents

% \makelettertitle % Print letter title

% \lipsum[1-3] % Dummy text

% \makeletterclosing % Print letter signature

%----------------------------------------------------------------------------------------

\end{document}