
\section{Peer Review Service}

\textit{I have reviewed for the following journals, including articles describing
original research and software applications:}

The Journal of Open Source Software, Molecular Plant-Microbe Interactions,
Molecular Ecology, Methods in Ecology and Evolution, Phytopathology, Ecology
and Evolution, Tropical Plant Pathology, Journal of Aquaculture Research, 
rOpenSci

\section{Peer Reviewed Publications}

\begin{enumerate}[leftmargin = 14pt]

  \item Thompson, RN, Stockwin, JE, van Gaalen, RD, Polonsky, JA,
    \textbf{Kamvar, ZN}, Demarsh, PA, Dahlqwist, E, Li, S, Miguel, E, Jombart,
    T, Lessler, J, Cauchemez, S, Cori, A (2019). Improved inference of
    time-varying reproduction numbers during infectious disease outbreaks.
    \textit{Epidemics}, 100356. 
  doi: \href{https://doi.org/10.1016/j.epidem.2019.100356}{10.1016/j.epidem.2019.100356}

	\vspace{3pt}
  
  \item Polonsky JA, Baidjoe A, \textbf{Kamvar ZN}, Cori A, Durski K, Edmunds
    WJ, Eggo RM, Funk S, Kaiser L, Keating P, Polain de Waroux O, Marks M,
    Moraga P, Morgan O, Nouvellet P, Ratnayake R, Roberts CH, Whitworth J,
    Jombart T.  (2019) Outbreak analytics: A developing data science for
    informing the response to emerging pathogens. \textit{Philosophical
    Transactions of the Royal Society B: Biological Sciences},
    \textit{374}(1776), 20180276. 
  doi: \href{https://doi.org/10.1098/rstb.2018.0276}{10.1098/rstb.2018.0276}

	\vspace{3pt}
  
  \item \textbf{Kamvar ZN}, Cai J, Pulliam JRC, Schumacher J, Jombart T.
    Epidemic curves made easy using the R package \textit{incidence} [version
    1; peer review: 1 approved, 2 approved with reservations]. F1000Research 2019, 8:139. (doi:
    \href{https://doi.org/10.12688/f1000research.18002.1}{10.12688/f1000research.18002.1})\\
	\rule[0.25\baselineskip]{0.25\textwidth}{0.5pt}\\
  software: \href{https://github.com/reconhub/incidence#readme}{https://github.com/reconhub/incidence}\\
  doi:\phantom{ware:} \href{https://doi.org/10.5281/zenodo.1404718}{10.5281/zenodo.1404718}

	\vspace{3pt}
  
  \item Miorini TJJ, \textbf{Kamvar ZN}, Higgins R, Raetano CG, Steadman JR,
    Everhart SE. (2019) Variation in pathogen aggression and cultivar
    performance against \textit{Sclerotinia sclerotiorum} in soybean and dry
    bean from Brazil and the U.S. \textit{Tropical Plant Pathology} \textbf{44:}73--81 doi:
    \href{https://doi.org/10.1007/s40858-018-00273-w}{10.1007/s40858-018-00273-w}\\
	\rule[0.25\baselineskip]{0.25\textwidth}{0.5pt}\\
  data/analysis: \href{https://github.com/everhartlab/SscPhenoProj#readme}{https://github.com/everhartlab/SscPhenoProj}\\
  doi:\phantom{t/analysis:} \href{https://doi.org/10.17605/OSF.IO/2X7FC}{10.17605/OSF.IO/2X7FC} 

	\vspace{3pt}
  
  \item Pannullo A, \textbf{Kamvar ZN}, Miorini TJJ, Steadman JR, Everhart SE.
    (2019) Genetic variation and structure of \textit{Sclerotinia sclerotiorum}
    populations from soybean in Brazil. \textit{Tropical Plant Pathology} \textbf{44:}53--64 doi:
    \href{https://doi.org/10.1007/s40858-018-0266-5}{10.1007/s40858-018-0266-5}\\
	\rule[0.25\baselineskip]{0.25\textwidth}{0.5pt}\\
  data/analysis: \href{https://github.com/everhartlab/brazil-sclerotinia-2017#readme}{https://github.com/everhartlab/brazil-sclerotinia-2017}\\
  doi:\phantom{t/analysis:}  \href{https://doi.org/10.17605/OSF.IO/E4UPH}{10.17605/OSF.IO/E4UPH} 

	\vspace{3pt}
  
  \item \textbf{Kamvar ZN}, Everhart SE. (2019) Something in the agar does not
    compute: on the discriminatory power of mycelial compatibility in
    \textit{Sclerotinia sclerotiorum}. \textit{Tropical Plant Pathology} \textbf{44:}32--40  doi:
    \href{https://doi.org/10.1007/s40858-018-0263-8}{10.1007/s40858-018-0263-8}\\
    \href{https://github.com/everhartlab/sclerotinia-review-2017/raw/master/manuscript/review.pdf}{Download the un-formatted PDF}\\
	\rule[0.25\baselineskip]{0.25\textwidth}{0.5pt}\\
  simulations: \href{https://github.com/everhartlab/sclerotinia-review-2017#readme}{https://github.com/everhartlab/sclerotinia-review-2017}\\
    doi:\phantom{mlatons:} \href{https://doi.org/10.17605/OSF.IO/B8VD3}{10.17605/OSF.IO/B8VD3} 

	\vspace{3pt}
  
	\item \textbf{Kamvar ZN}, Amaradasa BS, Jhala R, McCoy S, Steadman JR,
	Everhart SE. (2017) Population structure and phenotypic variation of
	\textit{Sclerotinia sclerotiorum} from dry bean (\textit{Phaseolus vulgaris})
	in the United States. \textit{PeerJ} \textbf{5}:e4152 doi: \href{https://doi.org/10.7717/peerj.4152}{10.7717/peerj.4152}\\
	\rule[0.25\baselineskip]{0.25\textwidth}{0.5pt}\\
	data/analysis: \href{https://github.com/everhartlab/sclerotinia-366#readme}{https://github.com/everhartlab/sclerotinia-366}\\
	doi:\phantom{t/analysis:}
	\href{https://doi.org/10.17605/OSF.IO/EJB5Y}{10.17605/OSF.IO/EJB5Y}

	\vspace{3pt}

	\item Gr\"unwald NJ, Everhart SE, Knaus BJ, \textbf{Kamvar ZN}. (2017)
	Best practices for population genetic analyses. \textit{Phytopathology}
	\textbf{107}:9 doi: \href{http://doi.org/10.1094/PHYTO-12-16-0425-RVW}{10.1094/PHYTO-12-16-0425-RVW}

	\vspace{3pt}

	\item Tabima JF, Everhart SE, Larsen MM, Weisberg AJ, \textbf{Kamvar ZN}, Tancos MA,
	Smart CD, Chang JH, Gr\"unwald NJ. (2016) Microbe-ID: an open source toolbox
	for microbial genotyping and species identification. \textit{PeerJ} \textbf{4}: e2279
	doi: \href{https://doi.org/10.7717/peerj.2279}{10.7717/peerj.2279}

	\vspace{3pt}

	\item Jombart T, Archer F, Schliep K, \textbf{Kamvar ZN}, Harris R, Paradis
	E, Goudet J, Lapp H (2016). apex: phylogenetics with multiple genes.
	\textit{Molecular Ecology Resources}. \textbf{17}:1 19-26 doi:
	\href{http://doi.org/10.1111/1755-0998.12567}{10.1111/1755-0998.12567}

	\vspace{3pt}

	\item \textbf{Kamvar ZN}, L\'opez-Uribe MM, Coughlan S, Gr\"unwald NJ, Lapp
	H, Manel S (2016). Developing educational resources for population genetics
	in R: an open and collaborative approach. \textit{Molecular Ecology Resources}.
	\textbf{17}:1 120-128 doi:
	\href{http://doi.org/10.1111/1755-0998.12558}{10.1111/1755-0998.12558}

	\vspace{3pt}

	\item Gr\"unwald NJ, Larsen MM, \textbf{Kamvar ZN}, Reeser PW, Kanaskie A,
	Laine J, Wiese R (2015) First report of the EU1 clonal lineage of
	\textit{Phytophthora ramorum} on tanoak in an OR forest.
	\textit{Plant Disease}. \textbf{100}:5, 1024-1024. doi:
	\href{http://doi.org/10.1094/PDIS-10-15-1169-PDN}{10.1094/PDIS-10-15-1169-PDN}

	\vspace{3pt}

	\item \textbf{Kamvar ZN}, Brooks JC and Gr\"unwald NJ (2015) Novel R tools for
	analysis of genome-wide population genetic data with emphasis on clonality.
	\textit{Front. Genet.} \textbf{6}: 208. doi: \\
	\href{http://doi.org/10.3389/fgene.2015.00208}{10.3389/fgene.2015.00208}\\
	\rule[0.25\baselineskip]{0.25\textwidth}{0.5pt}\\
	data/analysis:
	\href{https://github.com/grunwaldlab/supplementary-poppr-2.0#readme}{https://github.com/grunwaldlab/supplementary-poppr-2.0}\\
	doi:\phantom{t/analysis:}
	\href{https://doi.org/10.5281/zenodo.17424}{10.5281/zenodo.17424}

	\vspace{3pt}

	\item \textbf{Kamvar ZN}, Larsen MM, Kanaskie AM, Hansen EM, and Gr\"unwald
	NJ. (2015) Spatial and temporal analysis of populations of the sudden oak
	death pathogen in Oregon forests. \textit{Phytopathology}. \textbf{105}:7
	982-989. doi:
	\href{http://doi.org/10.1094/PHYTO-12-14-0350-FI}{10.1094/PHYTO-12-14-0350-FI}.\\
	\rule[0.25\baselineskip]{0.25\textwidth}{0.5pt}\\
	data/analysis:
	\href{https://github.com/zkamvar/Sudden_Oak_Death_in_Oregon_Forests#readme}{https://github.com/zkamvar/Sudden\_Oak\_Death\_in\_Oregon\_Forests}\\
	doi:\phantom{t/analysis:}
	\href{https://doi.org/10.5281/zenodo.13007}{10.5281/zenodo.13007}

	\vspace{3pt}

	\item Weiland JE, Garrido PA, \textbf{Kamvar ZN}, Marek SM, Gr\"unwald NJ, and
	Garz\'on CD. (2015) Population structure of \textit{Pythium irregulare}, \textit{P.
	sylvaticum}, and \textit{P. ultimum} in forest nursery soils of Oregon and
	Washington. \textit{Phytopathology}. \textbf{105}:5 684-694. doi: \\
	\href{http://doi.org/10.1094/PHYTO-05-14-0147-R}{10.1094/PHYTO-05-14-0147-R}

	\vspace{3pt}

  \item \textbf{Kamvar ZN}, Tabima JF, Gr\"unwald NJ. (2014) \textit{Poppr}: an
	R package for genetic analysis of populations with clonal, partially clonal,
	and/or sexual reproduction. PeerJ \textbf{2}: e281. doi: \\
	\href{http://doi.org/10.7717/peerj.281}{10.7717/peerj.281}

	\vspace{3pt}

	\item Buckner B, Beck J, Browning, K, Hoxha E, Grantham L, \textbf{Kamvar
	ZN}, Lough A, Nikolova O, and Schnable PS, Scanlon MJ, and Janick-Buckner D.
	(2007) Involving undergraduates in the annotation and analysis of global
	gene expression studies: creation of a maize shoot apical meristem
	expression database. \textit{Genetics}
	\textbf{176}: 741-747. doi:
	\href{http://doi.org/10.1534/genetics.106.066472}{10.1534/genetics.106.066472}

\end{enumerate}

% \section{Current Preprints}

% \begin{enumerate}[leftmargin = 14pt]

% \end{enumerate}


\section{Contributed Presentations}


\begin{enumerate}[leftmargin = 14pt]

  \item Amrish Baidjoe, Elburg van Boetzelaar, Raphael Brechard, Antonio Isidro
    Carrión Martín, Kate Doyle, Christopher Ian Jarvis, Thibaut Jombart,
    \textbf{Zhian Kamvar}, Patrick Keating, Anna Kuhne, Annick Lenglet, Pete
    Masters, Dirk Schumacher, Rosamund Southgate, Carolyn Tauro, Alex Spina,
    Maria Verdecchia, Larissa Vernier. (2019) Advancing data analytics for
    field epidemiologists using R: the R4epis innovation project. UseR! 2019,
    Toulouse, FR\\
    slides: \href{https://bit.ly/2YNAZLx}{https://bit.ly/2YNAZLx}

  \vspace{3pt}
    
	\item \textbf{Kamvar ZN}, Amaradasa BS, Jhala R, McCoy S, Steadman JR,
	Everhart SE. (2018) Population structure and phenotypic variation of
	\textit{Sclerotinia sclerotiorum} from dry bean in the USA. National Sclerotinia Initiative Conference, Bloomington, MN.\\
	doi: \href{https://doi.org/10.6084/m9.figshare.5791713}{10.6084/m9.figshare.5791713}

	\vspace{3pt}

	\item \textbf{Kamvar ZN}, Everhart SE, Gr\"unwald NJ. (2017) I Think We're
	A Clone Now: Factors Influencing Inference of Clonality In Diploid
	Populations. American Phytopathological Society National Conference,
	San Antonio, TX.

	\vspace{3pt}

	\item \textbf{Kamvar ZN}. (2017) Development and Application of Tools
	for Genetic Analysis of Clonal Populations. Department of Plant Pathology
	Seminar Series, University of Nebraska-Lincoln, Lincoln, NE.

	\vspace{3pt}

	\item \textbf{Kamvar ZN}. (2017) \textbf{Ph. D. Defense Seminar:} Development and
	Application of Tools for Genetic Analysis of Clonal Populations. Department
	of Botany and Plant Pathology, Oregon State University, Corvallis, OR.

	\vspace{3pt}

	\item \textbf{Kamvar ZN}, Brooks J, Gr\"unwald NJ. (2016) Tools for analysis
	of clonal population genetic data in R. American Phytopathological Society
	National Conference, Tampa, FL.\\
	url: \href{https://github.com/grunwaldlab/poppr-poster-aps-2016#readme}{https://github.com/grunwaldlab/poppr-poster-aps-2016}

	\vspace{3pt}

	\item \textbf{Kamvar ZN}, Larsen MM, Kanaskie AM, Hansen EM, Gr\"unwald NJ.
	(2015) Evidence for at least two introductions of the sudden oak death
	pathogen into Oregon forests. American Phytopathological Society National
	Conference, Pasadena, CA. \\
	url: \href{https://github.com/zkamvar/Presentation-APS2015#readme}{https://github.com/zkamvar/Presentation-APS2015}

	\vspace{3pt}

	\item \textbf{Kamvar ZN}, Tabima JF, Gr\"unwald NJ. (2014) Application of
	the R package poppr for analysis of population genetic data. American
	Phytopathological Society National Conference, Minneapolis, MN.

	\vspace{3pt}

	\item \textbf{Kamvar ZN} (2013) \textbf{Ph.D. Proposal Seminar:} Determination of
	pattern and process in the evolution of the plant pathogen
	\textit{Phytophthora syringae}. Department of Botany and Plant Pathology,
	Oregon State University, Corvallis, OR.

	\vspace{3pt}

	\item \textbf{Kamvar ZN} (2013) \textit{Poppr}: An R package for genetic
	analysis of populations with mixed (clonal/sexual) reproduction. Biology
	Graduate Student Symposium, Hatfield Marine Science Center, Newport OR.

	\vspace{3pt}

	\item \textbf{Kamvar ZN}, Tabima JF, Gr\"unwald NJ (2013) \textit{Poppr}: An
	R package for genetic analysis of populations with mixed (clonal/sexual)
	reproduction. Fungal Genetics Conference, Asilomar, CA.

	\vspace{3pt}

	\item \textbf{Kamvar ZN}, Gr\"unwald NJ (2012) \textit{Poppr}: An R package
	for popualtion genetic analysis. OSU Fall CGRB Conference, Oregon State
	University, Corvallis, OR.

	\vspace{3pt}

	\item Browning K, Fritz A, Hoxha E, and \textbf{Kamvar ZN} (2007) Annotation
	and analysis of global gene expression studies: creation of a maize shoot
	apical meristem expression database, Maize Genetics Conference, St. Charles,
	IL.

	\vspace{3pt}

	\item Browning K, Fritz A, Hoxha E, and \textbf{Kamvar ZN} (2007) Annotation
	and analysis of global gene expression studies: creation of a maize shoot
	apical meristem expression database, Truman Student Research Conference,
	Truman State University, Kirksville, MO.

\end{enumerate}

%----------------------------------------------------------------------------------------
%	COVER LETTER
%----------------------------------------------------------------------------------------

% To remove the cover letter, comment out this entire block

% \clearpage

% \recipient{HR Departmnet}{Corporation\\123 Pleasant Lane\\12345 City, State} % Letter recipient
% \date{\today} % Letter date
% \opening{Dear Sir or Madam,} % Opening greeting
% \closing{Sincerely yours,} % Closing phrase
% \enclosure[Attached]{curriculum vit\ae{}} % List of enclosed documents

% \makelettertitle % Print letter title

% \lipsum[1-3] % Dummy text

% \makeletterclosing % Print letter signature

%----------------------------------------------------------------------------------------
