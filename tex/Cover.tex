
%----------------------------------------------------------------------------------------
%	COVER LETTER
%----------------------------------------------------------------------------------------

% To remove the cover letter, comment out this entire block

\clearpage

\recipient{HR Departmnet}{EcoHealth Alliance\\460 West 34th Street - 17th Floor\\New York, NY 10001-2320} % Letter recipient
\date{\today} % Letter date
\opening{To whom it may concern,} % Opening greeting
\closing{Sincerely,} % Closing phrase
\enclosure[Attached]{curriculum vit\ae{}, references} % List of enclosed documents

\makelettertitle % Print letter title

I am applying for the position of Software Research Scientist, which I saw
advertised in mid-July of 2019. Currently, I work as a Research Software
Engineer at Imperial College London developing user-friendly and robust tools
for outbreak analyitcs. My formal training is in the population genetics of
plant pathogens, where I developed the widely-used R package \textit{poppr}
that implemented population genetic analyses partially clonal populations.
\textbf{The overarching goal in my work is to implement methods for analysis
that are clear, accurate, consistent, and easy to use.}

Having worked as both a research scientist and research software engineer, I am
excited for the opportunity to collaborate with the rOpenSci team. I have all of
the preferred skills and qualifications necessary for this position including
a degree in a quantitative field (population genetics), expertise in
open-source R package development\footnote{1:
\url{https://grunwaldlab.github.io/poppr}}, strong technical writing
skills\footnote{2: \url{https://peerj.com/articles/4152/}} \footnote{3:
\url{https://grunwaldlab.github.io/poppr/reference/psex.html}}, experience in
collaborative projects\footnote{4: \url{https://R4EPIs.netlify.com}} with
respectful peer review\footnote{5:
\url{https://github.com/R4EPI/sitrep/pull/76}}, and passion for robustness in
statistical software\footnote{6: \url{https://zkamvar.netlify.com/post/2017-09-23-squish/squish}}.

I believe that my background in quantitative science, robust software
development, collaborative capacity, and passion for both equity and 
fairness in peer review makes me well-suited to join the team as a software
research scientist. Thank you for your time and consideration,
I look forward to hearing from you. 

% MINIMUM QUALIFICATIONS

% - A Master's degree in statistics, computer science, or a related field with
%     a focus on quantitative methodologies, or equivalent experience in
%     statistical methods evaluation and development
% - Expertise in open-source R package development, including collaborative
%     development using Git and GitHub, testing frameworks, and continuous
%     integration
% - Strong writing skills
% - Experience in collaborative team projects and consensus building
% - A passion for improving scientific reproducibility

% DESIRED QUALIFICATIONS

% - A PhD in statistics, computer science, or a related field with a focus on
%     quantitative methodologies, or equivalent experience in statistical
%     methods evaluation and development
% - Published scientific or technical articles or software documentation

% - Research standards and protocols for evaluating statistical software
%     quality and correctness, and the extent of their adoption
% - Create new testing frameworks for R packages
% - Assist rOpenSci staff and project board members in drafting new
%     peer-review evaluation standards, guidelines, and documentation
% - Write technical and scientific papers, documentation, and blog posts
% - Assist in organizing peer-review system for scientific software and
%     managing the review board
% - Participate in and represent the rOpenSci project in person and via
%     on-line fora
% - Participate in other projects and tasks as required or assigned by
%     supervisor



% - Finish with summary sentence including qualifications
% - Add contatct info
% - Give thanks
\makeletterclosing % Print letter signature

%----------------------------------------------------------------------------------------


% EcoHealth Alliance seeks a creative, dedicated, and collaborative software
% research scientist to support a two-year project in launching a new software
% peer-review initiative. The software research scientist will work on the
% Sloan Foundation supported rOpenSci project, with rOpenSci staff and a
% statistical methods editorial board. They will research and develop standards
% and review guidelines for statistical software, publish findings, and develop
% R software to test packages against those standards. The software research
% scientist will work with staff and the board to collaborate broadly with the
% statistical and software communities to gather input, refine and promote the
% standards, and recruit editors and peer reviewers. The candidate must be
% self-motivated, proactive, collaborative and comfortable working openly and
% reproducibly with a broad online community.

% DESCRIPTION AND RESPONSIBILITIES

% - Research standards and protocols for evaluating statistical software
%     quality and correctness, and the extent of their adoption
% - Create new testing frameworks for R packages
% - Assist rOpenSci staff and project board members in drafting new
%     peer-review evaluation standards, guidelines, and documentation
% - Write technical and scientific papers, documentation, and blog posts
% - Assist in organizing peer-review system for scientific software and
%     managing the review board
% - Participate in and represent the rOpenSci project in person and via
%     on-line fora
% - Participate in other projects and tasks as required or assigned by
%     supervisor

% MINIMUM QUALIFICATIONS

% - A Master's degree in statistics, computer science, or a related field with
%     a focus on quantitative methodologies, or equivalent experience in
%     statistical methods evaluation and development
% - Expertise in open-source R package development, including collaborative
%     development using Git and GitHub, testing frameworks, and continuous
%     integration
% - Strong writing skills
% - Experience in collaborative team projects and consensus building
% - A passion for improving scientific reproducibility

% DESIRED QUALIFICATIONS

% - A PhD in statistics, computer science, or a related field with a focus on
%     quantitative methodologies, or equivalent experience in statistical
%     methods evaluation and development
% - Published scientific or technical articles or software documentation



