
%----------------------------------------------------------------------------------------
%	COVER LETTER
%----------------------------------------------------------------------------------------

% To remove the cover letter, comment out this entire block

\clearpage
\begin{flushright}
  Dr. Zhian N. Kamvar\\
  Department of Infectious Disease Epidemiology\\
  Imperial College London, W2 1NY, UK\\
  \phantom{menace}\\
  \today
\end{flushright}
% \recipient{HR Departmnet}{EcoHealth Alliance\\460 West 34th Street - 17th Floor\\New York, NY 10001-2320} % Letter recipient
% \date{\today} % Letter date
% \opening{Dear Dr. Ross, members of the search committee} % Opening greeting
\closing{Sincerely,} % Closing phrase
\enclosure[Attached]{curriculum vit\ae{}, references} % List of enclosed documents

% \makelettertitle % Print letter title
Dear Dr. Ross, members of the search committee,\\
\phantom{tollbooth}

% I am applying for the position of Software Research Scientist, which I saw
% advertised in mid-July of 2019. Currently, I work as a Research Software
% Engineer at Imperial College London developing user-friendly and robust tools
% for outbreak analyitcs. My formal training is in the population genetics of
% plant pathogens, where I developed the widely-used R package \textit{poppr}
% that implemented population genetic analyses partially clonal populations.
% The overarching goal in my work is to implement methods for analysis
% that are clear, accurate, consistent, and easy to use.
I am applying for the position of Software Research Scientist with rOpenSci. My
training as a computational biologist, passion for robust test-driven software
developement, and practical experience in cross-discipline communication  makes
me the ideal candidate for this position. As a developer of software for
analysis of population genetic data, I am keenly aware of how the lack of
standards impacts scientists\footnote{one recent example:
\url{https://twitter.com/reneecatullo/status/1133900861397016578}}. Preventing
problems like these can only be done with a strong collaboration between
scientists, statisticians, and software developers. I have direct experience in
managing projects with contributors with varying skill sets and am eager to work
together to bring better tools, standards, and training for statistical software
development and use. 

To give a more concrete picutre of my skillset, I have been developing in R and
C since 2012 (version control since 2013, tests and continous integration since 
2014) and currently maintain five packages on CRAN, one of which (poppr)
is featured in \textgreater400 publications.  

In these five
years, I have not only kept up-to-date with current workflow practices (testing,
version control, continuous integration), but also had the opportunity to 

 over six years experience developing analytical tools in R, experience in coordinating groups with varying
techincal experience, and a strong passion for test-driven development, I
believe that I am an ideal candidate for this position

Having worked as both a research scientist and research software engineer, I am
excited for the opportunity to collaborate with the rOpenSci team. I have all
of the preferred skills and qualifications necessary for this position
including a degree in a quantitative field (population genetics), expertise in
open-source R package development\footnote{examples: population genetics---\href{https://grunwaldlab.github.io/poppr}{poppr}, epidemiology---\href{https://www.repidemicsconsortium.org/aweek}{aweek}, and
\href{https://github.com/R4EPI/sitrep}{sitrep}}, strong technical writing
skills\footnote{scientific: \url{https://peerj.com/articles/4152/} , technical:
\href{https://grunwaldlab.github.io/poppr/reference/psex.html}{Documentation
for \texttt{poppr::psex()}}}, experience in leading collaborative
projects with varying technological expertise\footnote{RECON + MSF: \href{https://R4EPIs.netlify.com}{The R4EPIs
project}} and respectful peer
review\footnote{\href{https://github.com/R4EPI/sitrep/pull/76}{Example code
review for the sitrep package}}, and passion for robustness in statistical
software\footnote{\href{https://zkamvar.netlify.com/post/2017-09-23-squish/squish}{Blog
post about correcting a statistical error in my own package}}.

I believe that my background in quantitative science, robust software
development, collaborative capacity, and passion for both equity and 
fairness in peer review makes me well-suited to join the team as a software
research scientist. Thank you for your time and consideration,
I look forward to hearing from you. 

\makeletterclosing % Print letter signature

\clearpage

%----------------------------------------------------------------------------------------


% EcoHealth Alliance seeks a creative, dedicated, and collaborative software
% research scientist to support a two-year project in launching a new software
% peer-review initiative. The software research scientist will work on the
% Sloan Foundation supported rOpenSci project, with rOpenSci staff and a
% statistical methods editorial board. They will research and develop standards
% and review guidelines for statistical software, publish findings, and develop
% R software to test packages against those standards. The software research
% scientist will work with staff and the board to collaborate broadly with the
% statistical and software communities to gather input, refine and promote the
% standards, and recruit editors and peer reviewers. The candidate must be
% self-motivated, proactive, collaborative and comfortable working openly and
% reproducibly with a broad online community.

% DESCRIPTION AND RESPONSIBILITIES

% - Research standards and protocols for evaluating statistical software
%     quality and correctness, and the extent of their adoption
% - Create new testing frameworks for R packages
% - Assist rOpenSci staff and project board members in drafting new
%     peer-review evaluation standards, guidelines, and documentation
% - Write technical and scientific papers, documentation, and blog posts
% - Assist in organizing peer-review system for scientific software and
%     managing the review board
% - Participate in and represent the rOpenSci project in person and via
%     on-line fora
% - Participate in other projects and tasks as required or assigned by
%     supervisor

% MINIMUM QUALIFICATIONS

% - A Master's degree in statistics, computer science, or a related field with
%     a focus on quantitative methodologies, or equivalent experience in
%     statistical methods evaluation and development
% - Expertise in open-source R package development, including collaborative
%     development using Git and GitHub, testing frameworks, and continuous
%     integration
% - Strong writing skills
% - Experience in collaborative team projects and consensus building
% - A passion for improving scientific reproducibility

% DESIRED QUALIFICATIONS

% - A PhD in statistics, computer science, or a related field with a focus on
%     quantitative methodologies, or equivalent experience in statistical
%     methods evaluation and development
% - Published scientific or technical articles or software documentation



