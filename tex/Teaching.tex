
%----------------------------------------------------------------------------------------
%	TEACHING EXPERIENCE SECTION
%----------------------------------------------------------------------------------------
\cventry{2018--2019}{Introduction to R for Field Epidemiologists}{Workshop}{RECON}{}{
  This series of RECON workshops used classic epidemiological case studies to guide
  field epidemiologists (who had previously worked in STATA) through a workflow
  in R, emphasizing good data practices and reproducibility.
  \newline{}\newline{}
  Sessions:
  \begin{itemize}
    \item \textit{October 27--28, 2019} at TEPHINET global conference in Atlanta, GA, USA
    \item \textit{November 19--20, 2018} at the 2018 ESCAIDE conference in Mater Dei, Malta (60 participants)
    \item \textit{October 8--11, 2018} with EPIET Alumni Network in Sofia, Bulgaria (18 participants)
  \end{itemize}
}

\cventry{October 2017}{Reproducible data science for population genetics}{Workshop}{}{}{
	The goal of this workshop was to introduce professionals in the biological sciences to the practices of reproducible population genetic data analysis using R.
	Organized by \textit{PR}Statistics. Course Website: \url{https://goo.gl/amBbph}
	\newline{}\newline{}
	Presented from \textit{2017-10-23} to \textit{2017-10-27} at Margam Discovery Centre, Wales
}

\cventry{Summer 2017}{Introduction to R}{Workshop}{}{}{
	I wrote and instructed a three-hour workshop with Dr. Sydney E. Everhart. The goal of the workshop was to give a basic introduction to the R language that included reading data, writing data, and producing figures. \url{https://everhartlab.github.io/IntroR/}
	\newline{}\newline{}
	Sessions:
	\begin{itemize}
		\item \textit{June 14, 2017} North-Central American Phytopathological Society (NCAPS) 2017 Annual Meeting
		\item \textit{May 24, 2017} University of Nebraska-Lincoln
	\end{itemize}
}

\cventry{Summer 2016}{Reproducible Research in R}{Workshop}{}{}{
	I wrote and instructed a three-hour workshop with Zachary S. L. Foster and Dr. Niklaus Gr\"unwald. The goal of the workshop was to present plant pathologists with the basic tools necessary for performing reproducible science within the R environment. \url{http://grunwaldlab.github.io/Reproducible-science-in-R/}
	\newline{}\newline{}
	Presented on \textit{August 1, 2016} at American Phytopathology Society (APS) 2016 National Conference
}


\cventry{Spring 2016}{Graduate Teaching Assistant}{Botany Dept.}{OSU}{Corvallis, OR}{
	Taught introductory Botany for non-majors focused on emphasizing the role of
	plants in the environment, agriculture, and society. Two labs of $\sim$30 students.
	\newline{}\newline{}
	Responsibilities:
	\begin{itemize}
		\item Developed lectures, prepared materials, and wrote quizzes for labs each week
		\item Proctored all tests and quizzes
		\item Graded assignments and provided students with timely feedback
		\item Held office hours once a week
	\end{itemize}
}

\cventry{2014--2015}{Population Genetics in R}{Workshop}{}{}{
	I wrote and instructed a 4 hour workshop with Drs. Niklaus Gr\"unwald and Sydney Everhart. This workshop introduces tools and concepts that allow researchers to easily perform population genetic analyses in the R statistical environment. \url{http://grunwaldlab.cgrb.oregonstate.edu/popgen}
	\newline{}\newline{}
	Sessions:
	\begin{itemize}
		\item \textit{August 1, 2015} American Phytopathology Society (APS) 2015 National Conference
		\item \textit{August 9, 2014} American Phytopathology Society (APS) 2014 National Conference
		\item \textit{May 17, 2014} Oregon State University
	\end{itemize}
}

\cventry{Winter 2012}{Graduate Teaching Assistant}{Biology Dept.}{OSU}{Corvallis, OR}{
	Lead laboratories of $\sim$48 students in organismal diversity, organ systems, plant and animal physiology, genetics, evolution and ecology.
	\newline{}\newline{}
	Responsibilities:
	\begin{itemize}
		\item Developed introductory presentations for quizzes and labs
		\item Proctored all tests and quizzes
		\item Graded assignments and provided students with timely feedback
		\item Held office hours once a week
	\end{itemize}
}

\cventry{2009--2011}{English Instructor}{Herald NIE}{Joong-Dong}{Daegu, South Korea}{
	Taught basic to intermediate English to Korean students ranging from elementary to middle school with an emphasis on task-based learning techniques.
	\newline{}\newline{}
	Details:
	\begin{itemize}
		\item Took charge of 18 different classes per week
		\item Monitored language acquisition of each student via monthly evaluations based on interviews and speaking tests
		\item Wrote tests, assigned and graded homework pertinent to the level of the students. Initiated and mediated interesting topics for discussion courses
	\end{itemize}
}

\cventry{2008--2009}{English Instructor}{GnB English}{Sangin-2-Dong}{Daegu, South Korea}{
	Taught basic to intermediate English to Korean students ranging from elementary to middle school in tandem with one of the nine Korean English teachers at the academy.
	\newline{}\newline{}
	Details:
	\begin{itemize}
		\item Assisted with at least 30 different classes per week
		\item Monitored language acquisition of students throughout the year
		\item Gained the ability to be prepared for sudden changes in cirriculum and classroom size.
	\end{itemize}
}

\cventry{Fall 2006/07}{Undergraduate Teaching Assistant}{Biology Discipline}{TSU}{Kirksville, MO}{
	Appointed as teaching assistant for undergraduate cell biology course.
	\newline{}\newline{}
	Details:
	\begin{itemize}
		\item Helped prepare instructional labs for students of Dr. Diane Janick-Buckner's Cell Biology class
		\item Responded to student lab questions and referred to professor questions outside of my expertise/knowledge base
	\end{itemize}
}
